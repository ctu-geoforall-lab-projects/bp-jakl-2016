\documentclass[czech,11pt,a4paper]{article}
\usepackage[utf8]{inputenc}
\usepackage{a4wide}
\usepackage[pdftex,breaklinks=true,colorlinks=true,urlcolor=blue,
  pagecolor=black,linkcolor=black]{hyperref}
\usepackage[czech]{babel}

\pagestyle{empty}

\begin{document}

\begin{center}
  {\Large --- Posudek vedoucího bakalářské práce ---}
\end{center}

\vspace{.2cm}

\noindent \begin{tabular}{rp{.9\textwidth}}
  {\bf Název:} & Začleňování geografických datových sad do OpenStreetMap \\
  {\bf Student:} & Martin Jákl \\
  {\bf Vedoucí:} & Ing. Martin Landa, Ph.D. \\
  {\bf Fakulta:} & Fakulta stavební ČVUT v Praze \\ 
  {\bf Katedra:} & Katedra geomatiky \\
  {\bf Oponent:} & Ing. Jáchym Čepický \\
  {\bf Pracoviště oponenta:} & OpenGeoLabs s.r.o. \\
\end{tabular}

\vspace{1cm}

Téma práce vychází ze zájmu studenta o projekt OpenStreetMap (dále
OSM), do něhož přispíval již dříve jako aktivní \uv{maper}. Práce se
věnuje problematice začlenění dat, která v~dubnu 2015 uvolnil Institut
plánování a rozvoje hlavního města Prahy (dále IPR) jako
tzv. \uv{otevřená data}. Kromě seznámení se s novými technologiemi a
postupy se musel student věnovat i licenční otázce, díky níž nelze v
současné době data IPR do projektu OSM fyzicky začlenit. Vzhledem k
tomu nemohl být tento proces dotažen do finálního stavu (což by bylo
pravděpodobně i nad rámec práce tohoto formátu). Cílem bylo vytvořit
metodiku, kterou by bylo možné využít ve chvíli, kdy bude licenční
problém na straně IPR vyřešen. \newline

Úvodní část práce poskytuje nutný vhled do projektu OSM, jeho
historie, zdrojů dat a~především hromadných importů, kterých se do
značné míry dotýká. Dále jsou popsány datové sady, které IPR uvolnil v
režimu otevřených dat, přičemž je tento termín podrobně vysvětlen v
kapitole 2.2. Kromě toho jsou zmíněny technologie, které byly použity
pro zpracování praktické části. Tu lze rozdělit do několika sekcí. IPR
vystavuje datové sady v souborových formátech jako je Esri Shapefile
či DXF. Data poskytuje většinou ve dvou souřadni\-cových systémech --
S-JTSK a WGS-84. Pro strojové zpravování poskytuje tzv. ATOM
feed. Prvotním cílem bylo vybrané datové sady stáhnout a naimportovat
do geodatabáze PostGIS, která byla pro tento účel vytvořena na
oborovém serveru geo102. Tento úkol vyřešil student navržením
vlastního nástroje v programovacím jazyce Python, který umožňuje
z~příkazové řádky filtrovat poskytovaná data, dávkově je stahovat a
importovat do geodatabáze PostGIS. Tento nástroj považuji za zásadní
praktický výstup práce. V dalším kroku student vybral vhodná data pro
import a~navrhl pro ně metodiku začlenění do databáze OSM. Výsledek
můžeme považovat za úvodní rešerši či nástin metodiky. Nejlépe student
zpracoval začlení datové sady \uv{3D budov}, jejímž výsledkem jsou i
reprodukovatelné SQL dávky. Výsledek hodnotím jako průměrný. \newline

Student docházel na pravidelné schůzky připraven a plnil úkoly dle
domluvy. Bohužel vývoj nástroje pro stahování dat IPR zabralo více
času než bylo původně plánováno. Na podstatnou druhou část návrhu
metodiky začleňování dat IPR do databáze OSM nezbylo dostatek
času. To se na celkovém praktickém výstupu podepsalo. Celou práci
ještě snižuje necitlivá práce autora s českým jazykem. Přes všechny
výtky je celkové hodnocení velmi mírné, a to především díky snaze
studenta podat své maximum. \newline

\newpage

Na základě výše uvedeného doporučuji předloženou práci k obhajobě a
hodnotím ji klasifikačním stupněm

\begin{center}
  {\bf --- B (velmi dobře) --- }
\end{center}

\vskip 2cm

\begin{tabular}{lp{.2\textwidth}r}
& & \ldots\ldots\ldots\ldots\ldots\ldots\ldots \\
V~Praze dne 24. června 2016 & & Ing. Martin Landa, Ph.D. \\
& & Fakulta stavební, ČVUT v Praze \\
\end{tabular}

\end{document}
