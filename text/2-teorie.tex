\chapter{Teorie}
\label{2-Teorie}

\section{OpenStreetMap}
\label{OpenStreetMap}

\subsection{Vznik}
\label{vznik}
OpenStreetMap (OSM) je projekt, jež vznikl s cílem vytvoření a sběru 
volně dostupných geografických dat a následně jejich možné vizualizace
do topografických map. Projekt založil Steve Coast v červenci roku 
2004 v Anglii. Jako inspiraci mu posloužil projekt Wikipedia. 

Zprvu projekt využívalo jen pár nadšenců, ale postupem času získal 
projekt popularitu. S nárůstem počtu uživatelů narůstal i objem dat. 
Bylo tedy nutné zvyšovat kapacitu serverů. 

V dubnu roku 2006 byla založena nadace OpenStreetMap fundantion pro financováni 
samotného OSM (zaměstnanců, běhu serverů atd.). \cite{wikiOSM}


\subsection{Struktura dat}
\label{struktura dat}

OSM data jsou ukládána ve formátu XML (verze 1.0). Jeho výhoda je jasná 
struktura, snadná orientace v kódu pro člověka. Nevýhodou je ovšem větší objem 
dat, který lze ale snížit kompresí. Každý prvek má vlastní XML soubor. 

Třídy prvků v OSM jsou rozděleny na uzel (node), cesta (way) a 
relace (relation).

Příklad XML souboru pro cestu:

{\scriptsize
\begin{lstlisting}
<osm version="0.6" 
generator="CGImap 0.4.0 (32632 thorn-01.openstreetmap.org)" copyright="OpenStreetMap and contributors" attribution="http://www.openstreetmap.org/copyright" license="http://opendatacommons.org/licenses/odbl/1-0/">
   <way id="87249754" visible="true" version="2" changeset="34489106" timestamp="2015-10-07T11:52:41Z" user="Petr Dlouhý" uid="17615">
       <nd ref="1014526199"/>
       <nd ref="1014525941"/>
       <nd ref="1014526337"/>
       <nd ref="1014526022"/>
       <nd ref="1014526277"/>
       <nd ref="1014525984"/>
       <tag k="highway" v="path"/>
       <tag k="source" v="bing:ortofoto"/>
   </way>
</osm>
\end{lstlisting}
}


Uzel je definován jedinečným identifikátorem ({\tt node id=}). Jeho 
souřadnice jsou ve WGS~­84. Je také ukládána verze ({\tt version= }) a kdy 
byl do databáze přidán a v jaké změně to bylo provedeno ({\tt changeset=~}). 
Dále k bodu je možné připojit různé atributy s klíčem a hodnotou ({\tt tag~k=~v=~}). 

Linie je spojení dvou a více uzlů a má taky svůj identifikátor ({\tt way~id=~}).
Hlavičku XML souboru má shodnou s bodem. Ale dále obsahuje také seznam id uzlů, 
jež ji tvoří. Liniím lze jí taktéž přiřadit atributy.  

Linie lze ještě rozdělit na neuzavřené a uzavřené. Uzaveným liniím lze připojit 
i atributy určené jen pro plochy. Například les ({\tt landuse=forest}).
V případě, když přidává-li se atribut, jenž je v základu pro linie, ale chceme
ho použít pro plochy, přidává se atribut {\tt area=yes}.

Speciálním příkladem jsou relace ({\tt relation= }), do které lze zahrnout 
jeden a více prvků. Lze do nich spojit prvky stejné nebo odlišné třídy 
nebo i jiné relace. Pro příklad dálnice je tvořena mnoha liniemi a ty 
jsou zahrnuty do společné relace Dálnice D1. Nebo turistická trasa 
(od~KČT) je relace sdružující linie (cesty, pěšiny) tak i uzly 
(rozcestníky, vyhlídky, apod. ).

U atributů popsaných na osmwiki je vždy uvedeno, k jaké třídě je 
vhodné a nevhodné je použít (dle komunity uživatelů OSM). Pokud se stane, že je použit 
na~jinou než povolenou třídu, nemusí to hlásit chybu, ale může být následně 
problém v některých vykreslovačích. \cite{OSMfeatures}

\subsection{Licence}
\label{licence}

Původní data OSM byla na distribuována pod licencí 
Creative~Commons~Allribution­Share~Alike 2.0 (CC~BY~­SA~2.0). 


Tato licence umožňovala užití (distribuci ale i editaci) díla pod podmínkou, 
že bude uveden zdroj OpenStreetMap.org ve viditelné části 
vytvořených mapových dlaždic \cite{OSMlicence} 

  \begin{figure}[hbt]
    \centering
      \includegraphics{./pictures/attribution_example.png}
      \caption{attribution example}
      \label{fig:attribution_example}
  \end{figure} 

Roku 2012 byla licence publikovaných dat změněna na Open Data Commons 
Open Database Licence (OdbL). Tato změna licence přinesla problém 
s~daty, které byly poskytnuty projektu v předchozí licencí 
(CC­~BY~SA~2.0). Bylo nutné se dotázat každého z dřívějších 
přispěvatelů dat, ať už právnických osob tak i fyzických osob, jestli 
s touto změnou souhlasí a je možné s jejich daty i s novou licencí 
nakládat. U přispěvatelů, kteří nedovolili užívání jejich díla 
pod novou licencí, nebo u těch, co se nevyjádřili, bylo nutné 
z~databáze vymazat. Tato situace nastala pouze ve zlomku případů. 
Nejvíce tato změna licencí ohrozila data v zemích jako je Polsko a Nový Zéland.
\cite {OSMlicenceIssue} 

\subsection{Opendata}
\label{opendata}
Základní myšlenka otevřených dat vznikla v USA z iniciativy vlády Barracka Obamy. 
Tedy pokud vzniknou geografická data z veřejných peněz, měla by tedy být 
přístupná veřejně. Mělo to kladný efekt na tamní ekonomiku. Díky tomuto byly 
zprvu k~dispozici satelitní snímky povrchu Země a digitální model terénu 
s~rozlišením 30x30 m od NASA (pro pozdější vykreslení vrstevnic). 

Tento trend se začal rozšiřovat zprvu do zemí západních Evropy,
jmenovitě Anglie, Francie, ale i jiné země mimo Evropu..  

V ČR se tomuto věnuje fond otevrenadata.cz, který založil Otakar 
Motejl. Teto fond spolupracuje s nadací Sociery Fund Praha. V rámci 
těchto uskupení je vyvíjen tlak na zveřejňování smluv a dat 
státních institucí, jelikož jejich získání a údržba bylo placeno 
z~veřejných zdrojů.

\subsection{Zdroje dat}
\label{opendata}

Jak bylo zmíněno, byla snaha, aby mapová data tvořili jedinci vlastním 
sběrem dat. Sběr dat ve smyslu měřením vlastní GNSS (GPS, Glonass) 
přijímačem a znalost místních poměrů (uzavřené silnice, stezky atd.). 
Toto mapové dílo z těchto dat mohli volně užívat k vlastním užití. 
Komunita přispěvatelů se zprvu pomalu, ale později rychle rozrostla a 
dnes čítá 3,7 milionů registrovaných uživatelů s alespoň jednou 
vytvořenou změnou v OSM a 2,7 milionů účtů aktivních přispěvatelů. \cite{OSMstats}

Tyto mapové podklady byly velice vhodné i pro další projekt, dnes již 
velice rozšířený a známý jako Geocashing (GC). Projekt GC začal mapové 
podklady od OSM užívat a zároveň jeho uživatelé začali sami tvořit a 
přispívat do OSM. 

Přispěvatelé dat do OSM musí respektovat, že OSM je pod licencí OdbL. 
Tudíž i jejich zdroj dat musí splňovat tuto licenci. Proto by měli 
všechny svoje změny, které v OSM vytvoří, řádně ozdrojovat atributem 
s~klíčem 
{\tt source= }
V případě vlastním sběrem dat se vyplňuje hodnotou 
{\tt source=survey}
případně uvést zdroj, odkud čerpali. Pokud tuto povinnost poruší a 
použijí zdroj, jež není kompatibilní z politikou OSM, tak samotné OSM 
jejich změnu, aby předešel sporům, sám vymaže. Bohužel musí vymazat 
celou sadu změn, byť by v něm byl jeden prvek, jenž toto poruší. 

Druhým významným zdrojem dat jsou soukromé subjekty (společnosti). 
V~tomto případě jde většinou o podkladové zdroje dat. Pro obkreslování 
silničních síti z~leteckých nebo satelitních rastrů. V~jejich případě 
řešeno písemným svolením, nebo smlouvou. Jako významným zdrojem byla 
společnost Bing, jež nabídla k~dispozici letecké snímky většiny 
obydlené pevniny. 

Třetím zdrojem a zároveň postupně dominujícím, co do obsahu dat, jsou 
databáze ze státního sektoru. Tyto databáze jsou nejvhodnějším zdrojem 
dat.

\subsection{Vykreslovače}
\label{vykreslovače}
Na hlavní stránce OSM je pět „základních“ přednastavených vrstev vykreslených 
z~dat z OSM. Využívá aplikaci OpenLayers založenou na konceptu AJAX.

\begin{itemize}

  \item Standardní vrstva - vykresluje všechny prvky přiměřeně.
  \item Cyklomapa - vykresluje cyklostezky, výškopis. 
  \item Dopravní mapa - vykresluje silniční a železniční sítě.
  \item MapQuest Open vznikla jako podkladová mapa právě pro potřeby 
Geocaching.
  \item Humanitární mapa, která vykresluje služby (restaurace, banky, muzea, 
  školy, kostely...)  a potlačuje ostatní prvky. 

\end{itemize}

Existují další stránky jež se zabývají vlastní kompozicí a vykreslením
různých dat. Například mapu turistických a cyklistických tras vykresluje
 mtbmap.cz a to pro celou Evropu. 
 
Zajímavými projekty jsou, které k 2D mapě přidává „třetí“ rozměr a 
vytváří tzv. 2.5D mapu. Většinou jde o 3D zobrazení budov, mostů (dle 
atributů) popřípadě i stromů. 



\section{Importy}
\label{Importy}
Výraz import v tomto případe znamená začlenění většího množství datových sad 
z~jednoho datového úložiště (databáze serveru) na jiný. Při velkých objemech dat
se využívá výkonu výpočetní techniky z důvodu její bezchybovosti, a také 
z~důvodu časové náročnosti. Na člověku poté ale zbývá zvolit postup importu
a naprogramovat skript nebo program, podle něhož technika samotný proces 
provede. 

Datové importy z veřejných databází do databáze OSM jsou velmi cenné. 
Otevřené geografické, ale i jiné, databáze státu a jeho veřejných institucí 
financovaných státem jsou komplexní. Komplexní v tom smyslu, že obsahují celistvý
soubor dat, protože je daná instituce vyžaduje ke svém chodu. Jistá nevýhoda tu 
ale může být, a to, že data nemusí být vždy úplně aktuální. Některá data mohou 
být sbírána a zveřejněna i větším časovým odstupem.

V rámci České Republiky proběhlo již několik hromadných datových importů. Jak 
již bylo řečeno větší časová náročnost zabere samotná příprava na import. A to 
v~případě importu do OSM nejen napsání skriptu, ale i nutná diskuze tohoto záměru
na~diskuzní konferenci Talk-cz. 

\subsection{Talk-cz}
\label{Talk-cz}
Tato diskuze probíhá přes posílání emailových zpráv do společné konference. 
Uživatelům chodí emaily z probíhající diskuze a pokud na nějaký chtějí reagovat,
tak pošlou email na adresu serveru, na kterém diskuze běží, a musí do Předmětu 
napsat Re: a předmět zprávy, na kterou reagují. Server tyto zprávy pomocí 
předmětu a času řadí. Diskuze je poté k dispozici na webových stránkách 

{\tt http://lists.openstreetmap.org/listinfo/talk-cz} .

\section{IPR Praha}
\label{IPR Praha}
IPR Praha je zkratka názvu Institut plánování a rozvoje a hlavního města Prahy. 
Tento institut se věnuje urbanismu, architektury a rozvoje města Prahy. Hlavním
úkolem IPR je tvorba územního plánu Prahy a významným úkolem IPR je zajišťovat
zpracování geografických informací. Spravuje data a mapy města Prahy. Od roku 
2002 poskytuje na svých stránkách zdarma webové aplikace bez limitu využití. 
Po~rozvoji Pražského geoportálu došlo k jejich většímu využívání.  Na základě 
platných Pravidel pro poskytování dat a  výstupů z datových souborů a datového 
skladu Geografického byla od dne 1.~4.~2015 zveřejněny datové soubory a další 
webové služby. Tato data byla uveřejněna pod licencí CC-BY SA 3.0 \cite{IPR}


\section{PosgreSQL}
\label{PostgreSQL}
.."že Je to nadstavba SQL"


\subsection{PostGIS}
\label{PostGIS}
.."nadstavba PostgreSQL"..

\section{Python}
\label{Python}
Programovací jazyk Python v vznikl v roce 1991. Navrhlo ho Guido van Rossum. 
Je vyvíjen jako open source projekt. Inspiroval se hlavně programovacím jazykem
ABS, který byl přímo vytvořen pro~vyuku začatečníků v programování. Python je 
proto jeden z nejvýhodnějších programovacích jazyků pro začínající programátory,
ale i přesto ho lze použít pro~praktické programování. 

Jedná se tedy o jednoduchý programovací jazyk podporující různá programovací 
paradigmata, hlavně objektové, imperiativní, procedurální a omezené míře i 
fun- kcionální. Je multiplatformní, lze jej tedy použít na různých operačních 
systémech. Klade velký důraz na syntaxi psaného kodu, využívá hlavně prázdné 
znaky v~psaném kodu.  

Programy nebo skripty lze psát v textovém editoru ale je vhodné použít 
standartu PEP~8, který kontroluje prázdné znaky a správnou strukturu skriptu. 

V současné době je vydána stabilní verze 3.5.7, která vůči předchozí verzi 2.x
mění některá, již "zažitá" syntaxe. Například nejviditelnější je změna funkce 
{\tt print } pro verzi 2.x stačilo  {\tt print "Ahoj"}  a pro verzi 3.x je již
nutné string vložit do závorek  {\tt print("Ahoj")}  . Tyto dvě dnes používané 
verze 2.x a 3.x navzájem nekompatibilní.
\cite{python} 
\cite{wikiPython} 
  
  
\subsection{knihovna argparse}
\label{argparse} 
Knihovna argparse, jejíž autor je Tshepang Lekhonkhobe, je knihovna 
do~programovacícho jazyku Pyhon. Řeší konzolový vstup do aplikace a zpracovává 
ho pro další použití v programu. 


\subsection{knihovna xmltodict}
\label{xmltodict} 
Tato knihovna umí číst soubory ve formátu XML a zpracovává je do snažší formu 
dat, která je jednoduší číst pro samotný program. Data z XML zpracuje do formy 
datové řady (array), kde každý člen této řady může být opět řada dat. 
\cite{xmltodict}
příklad parserování dat ve XML formátu 

{\scriptsize
\begin{lstlisting}
<head>
  <many>ele1</many>
  <many>ele2</many>
</head>
\end{lstlisting}
}

Po otevření a parserování se v Python(u) provede příkazem.

{\scriptsize
\lstset{language=Python}
\begin{lstlisting}
with open('example.xml') as fd: 
  doc = xmltodict.parse(fd.read()) 
\end{lstlisting}
}

Objekt {\tt doc} má poté podobu.

{\scriptsize
\lstset{language=Python}
\begin{lstlisting}
OrderedDict([(u'head', OrderedDict([(u'many', [u'ele1', u'ele2'])]))]) }
\end{lstlisting}
}

V této datové řadě lze již snadno procházet. 

{\scriptsize
\lstset{language=Python}
\begin{lstlisting}
doc['head']['many'][0]
ele1
\end{lstlisting}
}


\subsection{knihovna GDAL/OGR}
\label{GDAL/OGR}
Jedná se o knihovnu pro práci s geografickými dat a umožňuje jejich čtení a zápis.
Je psána v programovacím jazyce C++. Je považován za jeden z hlavních 
open~source projektů a je hojně používán ve sféře GIS. Pracuje s vektorovými a 
rastrovými datovými formáty. Podporuje velkou škálu souborových formátů a to 
pro vektorová data ale i pro rastrová. Obsahuje databázi definic kartografických
projekcí používaných na celém světě a databázi jejich známých transformací. 
Umožňuje tedy komplexní operace s geografickými daty. \cite{GDAL}


\section{QGIS}
\label{QGIS}
QGIS je program vytvořen za účelem prohlížení, editaci



