\chapter{Praktická část}
\label{4-Praktická část}

\section{Geodatabáze PostGIS}
\label{Server PostGIS} 
V rámci této bakalářské práce byla zřízena geodatabáze PostGIS na školním
serveru {\em geo102}. Data z OSM jsou do školní databáze nahrávána pomocí programu {\tt osm2pgsql}.\footnote{dostupné z \url{http://wiki.openstreetmap.org/wiki/Osm2pgsql}}
%%% ML: to neni pravda, shellovy skript (viz git) pracuje primo s daty CR
%Ten nejprve stáhne aktuální data pro celou Evropu a následně je ořeže
%jen na území ČR.
%%% ML: neskodilo by zminit, ze tyto aktualnize zajistuje shellovy skript, ktyery je soucasti prilohy na cd
Obsah databáze je každý den pravidelně aktualizován podle aktuálního stavu OSM.

%%% ML: my jsme to zakladni schema ale upravovali (height)
Tabulky v databázi byly vytvořeny podle základního schématu. Pro~každou třídu prvků (uzel, cesta
a relace) je vytvořena samostatná tabulka. Pro~uzly je vytvořena
tabulka {\tt czech\_point}, pro~cesty tabulka {\tt czech\_line} a pro~plošné prvky (uzavřené cesty s atributy
pro~plochy) tabulka {\tt czech\_polygon}. Relace jsou
%%% ML: plochami a ne plochamy! (to uz me prekvapilo, rikal jste, ze text projde jazykovou korekci)
řešeny tak, že pokud jsou tvořeny plochami, vytvoří z~nich program
{\tt osm2pgsql} multipolygon a ten vloží do~tabulky {\tt czech\_polygon}. Stejně jsou
řešeny i relace tvořené cestami (tabulka~{\tt czech\_line}), respektive body
({\tt czech\_point}). Ještě je dle schematu vytvořena tabulka
{\tt czech\_roads}, která obsahuje všechny silnice v~ČR.

Každý prvek má přiřazen, v~rámci tabulky, jedinečné číslo {\tt gid}.
Dále je vytvořena pro každý prvek geometrie a pro~každý
atribut (dle schématu) stejnojmenný sloupec s~hodnotou. Například
bod s~atributem {\tt landuse=forest} má ve~sloupci {\tt landuse}
hodnotu {\tt forest}.

Takto vytvořená databáze bude použita pro~analýzu dat IPR a~jejich přípravu pro importu do~OSM.

\section{Data IPR}
\label{IPR data}
%%% ML: zde by se hodila reference na kap., kde o tom pisete
IPR na svém webu od~jara roku 2015 zveřejňuje svá data v režimu otevřených dat. Vektorová data zveřejňuje ve formátech GeoJSON, DXF, GML a ESRI Shapefile. Dále je na~výběr mezi dvěma
%%% ML: jsou to srs ane kartograficka zobrazeni (jste studenten geodezie!)
souřadnicovými systémy S-JTSK (EPSG:5514) nebo WGS 84 (EPSG:4326).

Celkem IPR zveřejňuje (toho času) 96 datových setů, které jsou rozděleny
do~kategorií.\footnote{viz webové stránky IPR, \url{www.geoportalpraha.cz/cs/opendata}}

Pokud není řečeno jinak, jedná se o data ve vektorové formě.
Těmito kategoriemi jsou:

\begin{itemize}
    \item   3D model - Obsahuje 3D modely budov a mostů v Praze a
            digitální model povrchu (DMP), dále rastrově digitální
            model povrhu a terénu, absolutní a relativní výšky budov.

            Z těchto dat by se pro OSM dala využít poloha budov
            ze~souboru Budovy 3D. Budovy byly do OSM již importovány
            %%% ML: mate nekde zkratku vysvetlenu
            ze~zdroje RÚIAN, není třeba je tedy importovat znovu.

    \item   Digitální technická mapa Prahy - zde jsou k dispozici
            ve~vektorové podobě všechny inženýrské sítě v~Praze,
            technické budovy a parcelní hranice.

            Do OSM se, možná zatím, nepřidávají inženýrské sítě.
            V~současnosti lze ale do OSM přidávat zdroje veřejného osvětlení.
            %%% ML: nasledujici veta je uplne zbytecna, jsou takovato data v IPR?
            %Takže pokud by nějaký soubor dat obsahoval
            %značky veřejného osvětlení, mohla by se tato
            %data importovat do OSM.
            Hranice města Prahy a jeho městských
            částí již v~OSM jsou dostupná. Datovou sada Technické budovy by bylo
            možné použít jako zdroj pro budoucí aktualizaci OSM.
            %%% ML: no prubehy hranic parcel obsahuje RUIAN, neni treba nic obkreslovat, to se mozna delo drive, dnes jiz rozhodne ne...
            Průběhy hranic parcel lze již nyní bezproblémově
            obkreslovat z rastrových podkladů, které veřejně
            poskytuje ČÚZK. 

    \item   Doprava - obsahuje cyklistické trasy a značky, pěší trasy,
            parkovací zóny, automaty, P+R parkoviště a mapy zón PID.
            Cyklistické trasy v Praze lze do OSM importovat, jsou ale
            %%% ML: uzivateli (a ne uzivately)! vedouci prace neni jazykovy korektor ;-)
            již velmi dobře zmapované samotnými uživateli. Datová sada
            obsahuje bodové značky, z~těch by se některé mohly
            importovat. V~případě datové sady 
            Cyklistická doprava~-~značky, byla data prohledána. Dle IPR
            uváděné hodnoty {\tt 103} ve sloupci {\tt DRUH}
            pro~Bikesharing, bohužel žádné takové hodnoty
            neobsahovaly. Import dat z této datové sady nebyl proto
            dál brán v potaz. Bylo by možné do OSM přidat parkovací
            zóny (jako nový atribut u~stávajících komunikací).
            Vhodná data jsou parkovací automaty a také P+R parkoviště.

    \item   Geologie - obsahuje vektorové mapy radonového nebezpečí.
            Tato data se do~OSM nepřidávají.

    \item   Hluk - obsahuje hlukové mapy, a to ve~dne a v~noci. Oboje
            v~rastrové formě. Tato data se do OSM nepřidávají. Dále
            obsahuje i protihlukové bariéry, které by bylo možné
            do~OSM přidat.

    \item   Kvalita životního prostředí - obsahuje datovou sadu
            Oblasti svozu komunálního odpadu. Tato data není vhodné
            do~OSM začlenit. Dále ale obsahuje datovou sadu
            Odpadní Zařízení pro občany, která obsahuje informace
            o~všech sběrnách odpadu v~Praze. Tato datová sada se jeví
            jako další vhodná pro~import.

    \item   Mapové podklady - obsahuje klady mapových listů různých
            měřítek (1:500 až 1:10~000). Klady mapových listů nejsou
            vhodná data pro~OSM. Dále kategorie obsahuje i datové sady
            vrstevnic, a to po~5~m, 2~m a 1~m. Vrstevnice přímo OSM
            neobsahuje, avšak byly by velkým zpřesněním stávajícího
            zdroje vrstevnic. Nebo jako zdroj pro~projekty, které
            kombinují OSM s~jinými daty.\footnote{Projekt mtbmap.cz nebo \url{http://mapa.prahounakole.cz/}}

    \item   Občanská vybavenost obsahuje pouze datovou sadu Veřejné
            toalety. Jistě jsou již toalety v Praze částečně
            zmapovány, ale i tak by bylo vhodné je doplnit. Jsou tedy
            vybrána jako další vhodná data pro~import.

    \item   Ochrana přírody a krajiny, obsahuje data o ochranných
            pásmech památných stromů. Bohužel ne samotné stromy.
            Vhodná data pro import jsou z~datové sady Přírodní parky a
            Významný krajinný prvek - registrovaný. Z těchto datových
            sad by se mohly doplnit, nebo aktualizovat data v OSM.

    \item   Ortofoto obsahují letecké snímky (rastr) Prahy 
            s~rozlišením až 5~cm na~pixel jak ve~viditelném spektru
            světla, tak i v~infračerveném.

            Ortofoto je vhodné pouze jako podkladový zdroj,
            například pro mapovací aplikaci JOSM.

    \item   Ovzduší obsahuje vektorové mapy znečištění ovzduší a také
            zdroje znečištění. Tato data se v OSM nemapují, jediné co
            by se mohlo přidat do OSM jsou bodové zdroje, a to
            například značkou pro~komín. Dále jsou v~této kategorii i
            bonity, a to z~různých hledisek (ovzduší, půdy,
            osvit, atd.). Tato data nejsou vhodná pro import do~OSM.

    \item   Platný územní plán. V této kategorii jsou datové sady
            Veřejně prospěšných staveb (plošné, liniové a bodové).
            Datová sada VPS obsahuje také informace o~P+R u~stanic
            metra. Tyto informace pocházejí z~územního plánu, a tedy nemusí být
            realizovány. Po prozkoumání a porovnání s~datovou sadou
            Záchytná parkoviště P+R, lze konstatovat, že jsou všechny záznamy obsaženy již v této datové sadě.

    \item   Socioekonomická data obsahuje pouze vektorovou mapu ceny
            pozemků. Data o ceně pozemků se do~OSM nepřidávají. 

    \item   Technická infrastruktura - vodní hospodářství, obsahuje
            záplavová území Q20, Q50 a Q100 a území zaplavené
            povodněmi v roce 2013. Informace o záplavových územích se do~OSM
            nepřidávají. Dále jsou v této kategorii datové sady
            Protipovodňové ochrany, které obsahují údaje o všech dočasných
            protipovodňových zdech. Jelikož se jedná o~dočasné překážky, a to~ještě jen v~době povodně, nemá smysl je do OSM přidávat.

    \item   Urbanismus - Z~této kategorie by se mohly pro OSM využít
            informace z~datové sady
            Stavební uzávěry - dopravní. Bylo by ale nutné udržovat
            aktuálnost infomací o plánovaných dopravních uzavírkách.
            Dále by bylo ještě možné přidat do OSM počet pater budov
            z~datové sady Podlaživost.
\end{itemize}


\subsection{Licenční problém}
\label{Licenční problém}
Jak již bylo řečeno výše, IPR svá data zveřejnil a stále zveřejňuje
(v době tvorby této práce) pod~licencí CC~BY-SA~4.0, viz kapitola \ref{licence CC BY-SA 4.0}.

%%% ML: tak co, vete chybi druha polovina (duvod: tyto dve licence nejsou kompatibbilni)
Jelikož jsou ale data OSM distribuována pod licencí ODbL (v1.0), která
neumožňuje začletnit data z licencí CC BY-SA. 

Na některá data (informace), jako například místní názvy, ulice atd.
se nemohou vztahovat autorská práva. Neboť nevyžadují žádnou 
kreativitu pro jejich \uv{vytvoření}. Ovšem na soubor (databázi) těchto dat
vytvořený podle daných kritérií již může být právní
ochrana uplatněna. A právě proto vznikla licence ODbL, která se
zaměřuje na ochranu databáze jako celku.

U dat chráněných autorským právem a danou licencí se vyžaduje jejich
vzdání. V~některých zemích, kde se nelze úplně vzdát autorských
práv, se vyžaduje omezení autorských práv na~nejnižší možnou
míru. Autor souhlasí, že nebude vymáhat svoje autorská práva, po dobu,
kdy jsou data uložena v databázi. \cite{ODbl}

IPR byl kontaktován, zdali by mohla být jeho data použita do OSM.
Dle vyjádření vyplynulo, že IPR by neměl problém s použitím svých dat,
jelikož jsou zveřejněna pod myšlenkou Opendata. Byla navrhnuta
možnost udělit od~IPR pro~OSM licenční výjimku.

Vznikla by tu však situace, kdy by byla data IPR k dispozici 
pod~licencí CC BY-SA a v~OSM (sice modifikovaná) pod~licencí
ODbL. Tohoto dualismu by mohl využít někdo, komu by licence CC BY-SA
neumožňovala jeho záměr. Mohl by si totiž data obstarat z databáze OSM,
sice modifikovaná, ale již pod novou licencí (ODbL), která by mu jeho
záměr umožnila.

Hlavní rozdíl je totiž v tom, že pod licencí CC~BY-SA sice může data
různě měnit a upravovat, ale poté když je zveřejňuje, musí zachovat
licenci CC~BY-SA. Nemusí ale uvolnit (zveřejnit) data, která k nim 
přidal. V~případě licence ODbl se tato situace diametrálně liší.
Uživatel opět smí data upravovat nebo k~nim přidávat jiná data, ale
poté může svoje dílo distribuovat pod~jinou licencí, za~podmínek, že
uvolní veškerá data, která k tomu použil.

Vhodnější by tedy bylo, změnit licenci dat zveřejňovaných IPR
na~ODbL.

%%% ML: nepredbihal bych, je to spise spekulace
%Dle vyjádření se tato situace bude ze strany IPR řešit tak,
%že se změní licence u distribuovaných dat na ODbL. Tato změna
%licencování se očekává v~několika měsících.


\section{IprDownloader}
\label{IprDownloader}
Jak již bylo řečeno, tak IPR distribuje geografická data ve formě datových souborů stažitelných z webových stránek organizace.
Kromě samotných datových souborů je k~dispozici kanál ve formátu ATOM ({\tt http://opendata.iprpraha.cz/feed.xml}).
Pomocí něho se lze také dostat ke všem distribuovaným datům. Jelikož
by bylo zdlouhavé stahovat všechny tyto soubory ručně, byl
v rámci práce vytvořen program, který umožňuje stahování a také import dat primárně do geodatabáze PostGIS. Skript byl
%%% ML: zde by se hodil odkaz do teoreticke casti
napsán v~programovacím jazyce Python a využívá knihovny GDAL.

Zdrojový kód programu je rozdělen do tří souborů: 
{\tt IprDownloader.py}, \newline {\tt IprBase.py} a {\tt IprPg.py}.


\subsection{IprDownloader}
Skript {\tt IprDowloader.py} parsuje vstupní údaje a
zpracovává je k~dalšímu použití v programu, viz \ref{argparse}. Základními parametry programy jsou
{\tt ---alike, ---crs, ---format, ---outdir, ---d}.
Dalšími vstupními parametry mohou údaje o databázi, kam se mají data
importovat. Jde o 
{\tt ---dbname, ---dbschema, ---dbhost, ---dbport, ---dbuser, ---dbpassword}.

%%% ML: kartografickych souradnic ? (student geodezie by mel znat
%%% rozdil mezi kartografickym zobrazenim a souradnicovym systeme)
%%% Cely odstavec mi neda smysl, zatim jsem jej zakomentoval
%% Skript hlídá vstupní formát kartografických souřadnic, a pokud se
%% neshodují s~možnými, oznámí chybu o špatném vstupu. Pokud uživatel
%% vloží kartografické zobrazení ve formě EPGS kódu, program jej převede
%% na~jeho název.


\subsection{IprBase}
Skript {\tt IprBase.py} definuje třídu {\tt IprDownloader} se všemi
potřebnými funkcemi. Většina funkcí slouží pro otevírání, čtení a
parsování XML souborů, viz kapitola \ref{xmltodict}. Obsahuje funkci
pro~hledání, která prochází všechny záznamy v ATOM souboru a porovnává, zda název
definovaný parametrem {\tt alike} je obsažen v názvu souboru dat. Poté najde dle nastavení ({\tt
  crc, format}) daný soubor a stáhne jej do~definovaného adresáře.


\subsection{IprPg}
Skript {\tt IprPg.py } definuje třídu {\tt IprDownloaderPg}, která
dědí všechny metody rodičovské třídy {\tt IprDownloader } a definuje další
funkce potřebné pro~správný import stažených dat do databáze
PostgreSQL. Přesněji stažený soubor zkontroluje, jestli není 
komprimován v~archivu {\tt *.zip}, pokud ano, tak vytvoří nový adresář,
který má stejný název jako archiv. Do nově vytvořeného adresáře poté 
extrahuje všechna data z archivu.


\section{Ovládání}
Při spuštění vypíše program všechna nalezená data,
která IPR dává k~dispozici. Pro hledání a označení, která data se mají
stáhnout, slouží parametr {\tt ---alike}. Pokud hledaný řetězec obsahuje mezeru,
tak je nutné jej umístit do uvozovek. Program rozlišuje velká~a~malá
písmena, háčky a čárky.

Příklad: {\tt \$ iprdownloader.py ---alike 'Technická mapa'}

Dále jsou k dispozici nastavení.
\begin{itemize}
    \item souřadnicový systém {\tt ---crs }. Podporován je S-JTSK a WGS-84. Kromě názvu lze použít i EPGS kódy (5514 a 4326). Přednastavený souřadnicový systém je S-JTSK.

    \item datový formát souborů {\tt ---format}.
    Mezi podporované vektorové formáty patří:
        \subitem {\tt json }  JavaScript Object Notation
        \subitem {\tt dxf  }  AutoCAD DXF
        \subitem {\tt gml  }  Geography Markup Language
        \subitem {\tt shp  }  ESRI Shapefile
        
        Výchozím formátem je {\tt shp} .
         
    Pokud uživatel chce stahovat rastrová data, je potřeba změnit
    předdefinovaný formát souboru na adekvátní rastrový formát např.
    {\tt tif} atd.
    
    \item adresář na lokálním počítači {\tt ---outdir}. Je předdefinován
    adresář {\tt data} ve~složce, kde je uložen samotný program.

\end{itemize}

Pro stažení dat stačí přidat parametr {\tt ---download} .

Příklad:

{\tt \$ iprdownloader.py ---alike 'Technická mapa' ---crs 4326 $\backslash$ \newline ---format gmp ---outdir IPR}

V případě importu stažených souborů do databáze PostGIS je nutné zadat minimálně název databáze {\tt ---dbname}.
To v~případě, pokud je databáze PostGIS umístěna na lokálním počítači. Pokud
je databáze umístěna na vzdáleném počítači, je nutné zadat jeho adresu
{\tt ---dbhost} a port {\tt ---dbport}. Pokud je ještě vyžadován
autorizovaný přístup, použije se přístupové jméno {\tt ---dbuser} a
heslo {\tt ---dbpassword}. Dále je možné zvolit si schéma v databázi
{\tt ---dbschema}. Pokud je definován název databáze, není již potřeba
používat parametr {\tt ---download} pro stahování dat.

Příklad:

{\tt \$ iprdownloader.py ---alike 'Technická mapa' $\backslash$ \newline ---dbname pgis\_osm\_jakl ---dbschema IPR ---dbhost geo102.fsv.cvut.cz  $\backslash$ \newline ---dbport 5432 } 
 

\section{Navržená data pro import}
\label{Navržená data pro import}

Po analýze všech dat byla vyhodnocena vhodná data pro začlenění
do~OSM.

\begin{itemize}
    \item   Výstupy PID, obsahuje vstupy/výstupy z metra.
    \item   Odpadní zařízení pro občany, dle IPR obsahuje sběrny
                odpadu. 
    \item   Cyklistická doprava - značky, dle technické dokumentace by
                mělo 
            obsahovat bodové značky stojanů Bikesharing.
    \item   Veřejné toalety
    \item   Parkovací automaty
    \item   Záchytná parkoviště P+R
\end{itemize}

Vybraná data byla pomocí IprDownloader.py stažena a naimportována
do~škol\-ní databáze PostGIS. Tato školní databáze obsahovala i aktuální data OSM. V~programu QGIS byla data pomocí SQL SELECTu
filtrována, na jejich základě byly vytvořeny nové tabulky. Data byla filtrována za použití doporučených atributů na české
OSM wiki. \cite{OSMfeatures}

Vybraná data IPR byla prohledána a porovnána s aktuálním stavem v~OSM.
Byly navrženy vhodné atributy, které by se ze~zdrojových dat daly
%%% ML: jako poznamka pod carou link na diskuzi
naplnit. Vybraná data s navrženými atributy byla uveřejněna na Talk-cz.


\subsection{Výstupy PID}
\label{Výstupy PID}
V datech od IPR bylo obsaženo celkem 353 záznamů. Jednalo se
o~výstupy z~metra (dveře, schodiště a výtahy).

V OSM databázi byly v tabulce {\tt osm.czech\_points}
nalezeny záznamy s hodnotou
\begin{verbatim}
    railway = subway_entrance
\end{verbatim}
Vytvořená tabulka z dat OSM obsahovala 236 záznamů. Po~porovnání dat
bylo zřejmé, že nejvíce jsou v OSM zmapovány výstupy v centru, kde
jsou posuny v~řádech decimetrů až metr. V okrajových částech města
nejsou zmapovány všechny výstupy, nebo chybí výstupy pro celé stanice.
Protože body vstupů do metra jsou většinou součástí linií, není vhodné
body mazat a vytvářet nové. Bylo potřeba najít mezi stávajícími body
v~OSM a IPR adekvátní dvojice. To znamená, že se jedná o body, které
reprezentují stejný objekt v~realitě.

%%% ML: obcas mluvite o vstupech a jindy o vystupech, coz je to same
Ani editace stávajících elementů není příliš přínosná. Vstupy
do~metra jsou minimálně 3 metry široké a případné změny by byly v řádů
cm až dm. Bylo by ale vhodné přidat výstupy z metra tam, kde nejsou
v~databázi OSM. Jedná se o stanice metra Zličín, Luka, Kačerov, Rajská
zahrada a Černý most. Celkem se tedy jedná pouze o 26 bodů, které by
bylo snadnější přidat ručně, protože výstupy jsou dostatečně zmapované,
pouze občas chybí v uzlech atribut {\tt railway~=~subway\_entrance}.


\subsection{Odpadní zařízení pro občany}
\label{Odpadní zařízení pro občany}
Data od IPR obsahují seznam 36 sběrných dvorů s velmi detailním
popsání (příjem, otevírací doba, zřizovatel). 

Z~OSM tabulky {\tt osm.czech\_point} byly záznamy filtrovány podle
hodnot
\begin{verbatim}
    amenity = recycling
    recycling_type = centre
\end{verbatim}    
Bylo nalezeno celkem 11 záznamů (bodů), z~toho bylo 6 bodů (z~IPR)
ve~vzdálenosti 100 metrů od bodů z OSM. Těchto 6 bodů bude vhodné
pouze přesunout. Zbylé body je nutné v databázi OSM nově vytvořit.

Z dostupných dat z tabulky byly navrženy tyto atributy
\begin{verbatim}
    amenity = recycling
    recycling_type = centre
    opening_hours =
    fee =
    operator =
    source =
    source:loc = IPR
\end{verbatim}
Navržené atributy by se naplnily z tabulky dat od IPR. Otevírací doba
by byla vložena do~{\tt opening\_hours}. Hodnoty v~tomto atributu
by byly formátovány dle doporučeného zápisu.\footnote{\url{http://wiki.openstreetmap.org/wiki/Key:opening\_hours}}
U~atributu {\tt fee} bylo navrženo {\tt free of charge for the Prague citizens}, tedy zdarma pro~občany Prahy. Hodnota
{\tt operator} by byla ze~stejnojmenného sloupce z~dat od~IPR. 
Atribut {\tt source} by byl odvozen ze~sloupce {\tt POSKYT}.

Dále byly navrženy atributy
\begin{verbatim}
    recycling:paper = yes
    recycling:plastic = yes
    recycling:metal = yes
    recycling:wood = yes
    recycling: ... = yes
\end{verbatim}
Podle toho co zde lze recyklovat, dle hodnoty ve~sloupci
%%% ML: odpadprijem
{\tt odpadprije}.


\subsection{Veřejné toalety}
\label{Veřejné toalety}
Datový soubor Veřejné toalety obsahuje jednu tabulku dat. V~této
tabulce je celkem 247 záznamů. U každého záznamu je uvedena lokalita a
provozovatel. Dále je uvedena otevírací doba, zdali je
nutné zaplatit poplatek, jestli je bezbariérový a zdroj těchto dat.

Záznamy odpovídající toaletám byly v~OSM vyhledány podle atributu
\begin{verbatim}
    amenity = toilets
\end{verbatim}

%%% ML: nasledujici odstavec prosim cely prepiste
a bylo zjištěno, toho času, 185 záznamů. Překrytí dat IPR a OSM bylo.
Buď k bodu z IPR nebyl v okruhu 5 metrů žádný bod z OSM, nebo
byl, ale měl jiné souřadnice. Také nastaly situace, kdy v OSM
datech byl jen jeden bod, ale z dat od IPR tam bylo více (záznamů) 
bodů. Bylo by tedy vhodné stávající data doplnit daty z IPR.

Na základě rešerše dat z IPR a doporučených atributů byly vybrány tyto
\begin{verbatim}
    acces = yes/permissive
    fee = no/yes
    wheelchair = yes/no
    opening_hours =
    operator =
    description =
    source = IPR
    source:location = IPR
\end{verbatim}
Pokud by hodnota byla {\tt fee~=~yes} připojil by se atribut
\begin{verbatim}
    fee:price =
\end{verbatim}
a jeho hodnota by byla cena v Kč.

Hodnota u {\tt description} by byla převzata ze sloupce
{\tt lokalita}. U~atributu {\tt operator} byla navržena hodnota
{\tt Hlavní Město Praha}, ale jen u~bodů, které mají v tabulce
(data od IPR) ve sloupci {\tt Operator} hodnotu {\tt HMP}.
U zbývajících bodů (záznamů) by tento atribut neměl žádnou hodnotu.


\subsection{Parkovací automaty}
\label{Parkovací automaty}
Datová sada Parkovací automaty obsahovala celkem 452 záznamů (bodů)
parkovacích automatů. Je zde uveden pouze {\tt TYP} a Poskytovatel dat.
Nelze tedy určit zdali automat přijímá mince, bankovky (českou nebo
cizí měnu) a nebo platební karty.
\\*
Pro porovnání byly vyhledány body z OSM s atributy
\begin{verbatim}
    amenity = vendings_machine
    vending = parking_tickets
\end{verbatim}
Podle těchto parametrů bylo nalezeno pouze 8 bodových prvků v databázi
OSM. Pouze dva záznamy z~OSM jsou v~blízkosti k bodům z~IPR. Jeden je
vzdálený do 5 metrů a druhý do 20 metrů. Nelze tedy určit, zda oba (IPR
a OSM bod) odkazují na stejný objekt, nebo na dva různé. Proto se
původní body nebudou přesouvat, a ani rušit. Nové prvky půjde zpětně
rozlišit podle atributu {\tt source:loc}.
\\*
U nově přidaných uzlů byly navrženy tyto atributy
\begin{verbatim}
    amenity = vendings_machine
    vending = parking_tickets
    source = IPR
    source:loc = IPR
\end{verbatim}
Do~hodnoty {\tt source} by byla použita hodnota ze~sloupce
{\tt POSKYT}. Hodnota u~atributu {\tt source:loc} by byla použita
{\tt IPR}.

\subsection{Záchytná parkoviště P+R}
\label{Záchytná parkoviště P+R}
Datová sada Záchytná parkoviště P+R obsahoval 16 záznamů. V~tabulce
bohužel nejsou žádné jiné údaje, kromě polohy uložené v~geometrii a
poskytovatel dat ve~sloupci {\tt poskyt}.

Z dat OSM byly vyhledány body pomocí atributu
\begin{verbatim}
    amenity = park_ride
\end{verbatim}
Tento atribut je stále ve fázi schvalování, tudíž nebyly nalezeny
žádné záznamy. 
Proto bylo vyzkoušeno hledání pomocí obecnějšího atributu
\begin{verbatim}
    amenity = parking
\end{verbatim}
Na území Prahy bylo nalezeno, toho času, 182 záznamů (bodů). Bylo
vyzkoušeno hledání s přidaným atributem {\tt name}, který by
obsahoval P+R. Byl nalezen pouze jeden záznam.

Dále bylo vyzkoušeno hledání bodů s atributy
\begin{verbatim}
    amenity = parking
    park_ride = yes
\end{verbatim}
Na území Prahy byl nalezen pouze jeden prvek.

Je tedy navrženo importovat všechny prvky z databáze IPR.
Navržené atributy jsou
\begin{verbatim}
    amenity = parking
    park_ride = yes
    source = HMP-URM
    source:loc = IPR
\end{verbatim}

%%% ML: odkaz na diskuzi, proc je to zmineno pouze u P+R, ta diskuze
%%% tykala vsech datovych sad predpokladam
Výše zmíněné návrhy byly takto zveřejněny v diskuzi na stránkách
Talk-cz.

\subsection{Budovy 3D - výšky budov}
\label{Budovy 3D - výšky budov}
V~komentářích k návrhům v~Talk-cz bylo dále doporučeno přidání
výšek budov ze~souboru dat Budovy 3D.
\begin{verbatim}
    building:height
\end{verbatim}
V návrhu bylo doporučeno označení pro zdroj tohoto importu, pro případné pozdější úpravy a opravy.
\begin{verbatim}
    source:building:level =
\end{verbatim}

Při rešerši dat uložených ve školní geodatabázi bylo zjištěno,
že data stahovaná a ukládaná z OSM neobsahují potřebné údaje.
Přesněji ukládaná tabulka neobsahovala údaje o výšce budov
(sloupec {\tt height} a sloupec {\tt building\::height}).
Proto bylo nutné základní schéma pro {\tt osm2pgsql} změnit tak, aby se tyto údaje také 
ukládaly. Upravené schéma bylo uloženo a bylo nastaveno jako výchozí.

Také se ukázalo, že bude vhodné, aby se ukládal údaj, jestli
není polygon vytvořen z relace budovy, protože v rámci této relace
(budovy) mají definovanou výšku ({\tt height}, nebo
{\tt building\::height}) pouze její části, ze~kterých je složena.
K~tomuto rozeznání byl použit atribut {\tt builing\::part}.

V sekci Budovy 3D je dále uloženo 128 datových sad. Tyto datové sady obsahují budovy jen z malé oblasti, protože
by datová sada budov pro~celé území Prahy byla moc velká (až v~řádů
GB). Každá datová sada je přibližně obdélníková oblast z území Prahy a
má svůj originální název.

%%% ML: pridat odkaz na schema v priloze
V rámci této práce bude pracováno s~datovou sadou
{\tt BD\_Prah73}. Po~vybrání této jedné datové sady (oblasti) byla
vytvořena minimální ohraničující oblast okolo ní a přidán přesah o
100~metrů na každou stranu. Tedy byl vytvořen obdélník okolo všech dat
(polygonů) v~jedné datové sadě. Byl zvolen tento postup, aby se z~dat
OSM nevyřadily polygony, které by nebyly celým objemem uvnitř.

Pomocí tohoto obdélníku byly vybrány budovy z OSM uvnitř tohoto
obdélníku a poté byla vytvořena nárazníková zóna (buffer) s hodnotou
3~m pro~každou budovu. Byl zvolen tento postup, protože budovy z IPR a
OSM nemusí mít přesně stejnou polohu a tvar (polygonů). Tento dočasný
datasest byl nazván {\tt bud73\_OSM}.

Pro velikost nárazníkové zóny kolem budov z OSM se provedla analýza.
Modelováno bylo s hodnotami po~0.5~m od~0.5~m až do vzdálenosti 5~m.
Při malé velikost nárazníkové zóny nastávala situace, že v~této zóně
nebyla nalezena žádná budova (polygon budovy) z datasetu IPR. Mohlo by
se zdát, že čím větší nárazníková zóna kolem budov z OSM, tím lépe.
Avšak, čím byla velikost nárazníkové zóny zvolena větší, tím více se
stalo, že v této nárazníkové zóně "uvízlo" více budov (polygonů)
z~datasetu IPR. Bylo tedy nutné zvolit "zlatou střední cestu" tj. 3 m.

Další možností jak se vyvarovat, aby v~nárazníkové zóně kolem budov
z~OSM "uvízlo více polygonů budov (z dataset IPR) je spouštět skript
vícekrát a postupně volit větší a větší nárazníkovou zónu. Skript je
napsán tak, aby vybral a vytvořil nárazníkovou zónu, jen těm budovám
(polygonům), které nemají atribut {\tt building:part} a ani
{\tt building:height}. Ovšem tento postup spouštět skript vícekrát je
samozřejmě také více časově náročné.

Dále byl proveden průzkum dat z~{\tt BD\_Prah73} datové sady
ze~souboru Budovy 3D (z IPR). Bylo zjištěno, že vždy polygony, které
vytvářejí 3D model jedné budovy, mají stejnou hodnotu ve~sloupci
{\tt id\_bud}. V~3D geometrii těchto polygonů budov jsou uloženy
všechny (x,y,z) souřadnice vrcholů, jež určují polygon. Polygony se
stejnou hodnotou {\tt id\_bud} lze dále dělit dle hodnot
ve~sloupci {\tt typ}.
\\*
Těmi to hodnotami jsou
\begin{verbatim}
    dilci plocha stresni kruhove plochy
    komin
    sikma stresni plocha
    svisla obvodova stena
    vikyr_stresni nadstavba
    vytah_vetrani_klimatizace
    vodorovna stresni plocha
    vyznacena vez na strese
    zakladna vikyre, stresni nadstavby
    zakladova deska
 .. a další
\end{verbatim}

Poté byly z~datové sady {\tt BD\_Prah73} vybrány všechny polygony,
které měly ve~sloupci {\tt typ} obsahují frázi {\tt stresni}.
Geometrie těchto vybraných 3D polygonů byly opraveny funkcí
{\tt ST\_MakeValid()} a následně "přetvořeny" na 2D polygony funkcí
{\tt ST\_Force\_2D()}. Vynechala se jen Z-ová souřadnice z~geometrie.
Kvůli některým situacím, kdy k~sobě polygony přesně nedoléhaly, bylo
nutné vytvořit z~nich nárazníkovou zónu (buffer) a jako dostačující
byla zvolena hodnota 0.1 m. Následně byly tyto polygony takzvaně
"rozpuštěny". Rozpuštění znamená, že polygony, které se dotýkají
minimálně jednou společnou stranou, se sloučí do~jednoho polygonu.
Rozpuštěny byly polygony navzájem jen ty, které měly stejnou hodnotu
ve~sloupci {\tt id\_bud}. Tím se docílilo, že každá budova bude
reprezentována jenom jedním 2D polygonem. Tato dočasná datová sada
byla nazvána {\tt bud73\_IPR}.

Následně byly pro každý polygon z~{\tt bud73\_OSM} hledány
všechny polygony z~{\tt bud73\_IPR}, které by byly uvnitř.
Na~tuto operaci byla použita funkce
\begin{verbatim}
    ST_Within( bud73_OSM.geom , bud73_OSM.geom )
\end{verbatim}
Tímto byly nalezeny prvky z dat IPR, které jsou "přibližně na stejném
místě" jako budovy z OSM. Do tabulky o dvou sloupcích byly uloženy
pouze údaje {\tt gid} pro budovy z OSM a {\tt id\_bud} polygonu budovy
(z dat IPR). Tato dočasná tabulka byla nazvána {\tt gid\_id}.

Dále tedy bylo možné k těmto prvkům ({\tt gid} OSM budov) přiřadit
výšku podle hodnoty {\tt id\_bud}, kterou mají všechny polygony
tvořící jednu budovu stejnou. Nejprve musely být vybrány pouze ty
záznamy {\tt gid}, které jsou v tabulce {\tt git\_id} pouze jednou.
To jest polygon budovy z~OSM, ve které byl nalezen pouze jeden polygon
z datasetu IPR. Následně k nim byly připojeny výšky podle hodnoty
{\tt id\_bud}. Pro nalezení výšky budovy \footnote{dle \url{http://wiki.openstreetmap.org/wiki/Cs:Simple_3D_Buildings}}
byly použity všechny polygony tvořící budovu. V~rámci jedné budovy
(podle stejné hodnoty v {\tt id\_bud}) byla výška budovy vypočtena
jako rozdíl nejvyššího bodu a nejnižšího bodu budovy.
Přesněji byla vypočtena takto
\begin{verbatim}
    select	id_bud,
            max(maxZ) - min(minZ) as Height
    from(   select	id_bud,
                    ST_ZMax(geom) as maxZ,
                    ST_ZMin(geom) as minZ
            from    ipr.bd3_prah73
        ) as max_min
    group by id_bud
\end{verbatim}
Ve skriptu je tato dočasná tabulka označena {\tt osm\_height}.

Nakonec se získané hodnoty výšek budov přidají do původního datasetu
(tabulky) {\tt praha\_building\_osm}. 
Dočasné tabulky {\tt bud73\_osm, bud73\_ipr, gid\_id} a
{\tt osm\_height} jsou nakonec vymazány.






