\chapter{Praktická část}
\label{3-Praktická část}

\section{server PostGIS}
\label{server PostGIS}
V rámci této bakalářské práce s daty byla zřízena databáze PostGIS na školním
serveru. Data z OSM jsou do školní databáze nahrávána pomocí programu {\tt osm2pgsql.py}.\footnote{dostupné z \url{http://wiki.openstreetmap.org/wiki/Osm2pgsql}}
Ten nejprve stáhne aktuální data pro celou evropu a následně ořeže
jen na území ČR. Tyto data jsou každý den pravidelně aktualizována.

Tabulky v databázi byly vytvořeny podle základního schematu, který vyfiltruje některé údaje.
Pro každou ze tří tříd prvků (bod, liniem a polygon) z OSM je vytvořena samostatná 
tabulka. Pro uzly je vytvořena tabulka {\tt czech\_point}.
Pro~cesty z~OSM je vztvořena tabulka {\tt czech\_line} a pro plošné 
prvky (uzavřené cesty s atributy pro plochy) je vytvořena tabulka
{\tt czech\_polygon}. Ještě je vytvořena tabulka {\tt czech\_roads},
která obsahuje všechny silnice v ČR.

Každý prvek má přiřazen, v rámci tabulky, jedinečné číslo {\tt PID}, 
dále je vytvořena geometrie a pak pro každý atribut je stejnojmený 
sloupce s~hodnotou. 
Například bod s atributem {\tt landuse = forest} má vse sloupci 
{\tt landuse} hodnotu {\tt forest}.

Dále tato databáze bude používána pro analýzu a úpravu dat z~IPR a
pro~import do~OSM.


\section{IPR data}
\label{IPR data}
IPR na svém webu od jara roku 2015 zveřejǔne data.
U všech zveřejněných dat, které jsou ve~vektorovém formátu jsou
možnosti souborvé formáty GeoJSON, DXF, GML nebo ESRI Shapefile.
Dále je na výběr mezi dvěma kartografickými zobrazeními S-JTSK nebo WGS 84.

Celkem IPR zveřejňuje 96 druhů datových souborů a ty jsou rozděleny
do~kategorii.\footnote{viz stránka IPR \url{www.geoportalpraha.cz/cs/opendata}}

Pokud není řečeno jinak, jedná se o data ve vektorové formě.
Těmito kategoriemi jsou:

\begin{itemize}
    \item   3D model - Obsahuje 3D modely budov a mostů v Praze a digitální
            model povrchu (DMP) a dále rastrově digitální model povrhu a
            terénu, absolutní a relativní výšky budov.

            Z těchto dat by se pro OSM dala využít poloha budov ze~soubru
            Budovy 3D. Budovy jsou ale již importovány ze zdroje RUIAN, takže
            již nejsou potřeba přidávat.

    \item   Digitální technická mapa Prahy - jsou zde k dispozici v vektorové
            podobě inženýrské sítě v Praze, technické budovy a hranice parcel.

            Do OSM se, možná zatím, nepřidávají inženýrské sítě. V tuto nyní
            lze do OSM přidávat zdroje veřejného osvětlení. Takže pokud by
            nějaky sbour obsahoval značky veřejného osvětlení mohlo by se tato
            data importovat do OSM. Hranice města Prahy a městských čístí již
            v~OSM je. Technické budovy by bylo možné použít jako zdroj
            pro budovucí aktualizaci OSM.

    \item   Doprava - obsahuje cyklistické trasy a značky, pěší trasy
            parkovací zóny, automaty, P+R parkoviště a mapy zon PID (vektor)

            Cyklysticé trasy v Praze by se mohli importovat ale jsou již velmi
            dobře zmapované samotnými uživateli. Datová sada obsahuje bodové značky
            a z těch by se některé mohli importovat.
            V případě dat Cyklistická doprava - značky, byla data prohledána, dle IPR
            uváděné hodnoty {\tt 103} ve sloupci {\tt DRUH} pro Bikeshating, bohužel žádné
            takové hodnoty neobsahovaly. Import dat z této tabulky nebyl dál brán v potaz.
            Bylo by možné do OSM přidat parkovací zony (jako nový atribut u stávajících komunikací).
            Vhodná data jsou parkovací automaty a také P+R parkoviště.

    \item   Geologie - obsahuje vektorové mapy radonového nebezpečí. Tato data
            se do OSM nepřidávají.

    \item   Hluk - obsauhjí hlukové mapy a to ve dne a v noci v rasterové formě.
            Tato data se do OSM nepřidávají. Dále obsahuje i protihlukové bariéry,
            tyto data by bylo možné do OSM přidat.

    \item   Kvalita životního prostředí - Obsahuje datovou sadu Oblasti svozu komunílního odbadu.
            Tyto data nejsou vhodná do OSM začlenit. Dále ale obsahuje datovou sadu
            Odpadní Zařízení pro občany, a toto obsahuje informace o všech sběrnách v Praze,
            tedy tato datová sada je vhodná pro import.

    \item   Mapové podklady - obsahuje klady mapových listů různých měřítek (1:500 až 1:10 000)
            Tyto data nejsou vhodná do OSM. Dále obsahuje i vrstevnice a to po
            5~m, 2~m a 1~m. Vrstevnice přímo OSM neobsahuje a ani sám nevykresluje.
            Je ale možné tyto data využít pro kompozice s OSM.\footnote{projekt mtbmap.cz nebo http://mapa.prahounakole.cz/}

    \item   Občanská vybavenost obsahuje pouze datovou sadu Veřejné toalety.
            Tyto data by bylo vhodné do OSM přidat.

    \item   Ochrana přírody a krajiny, obsahuje data ochraná pásma památných stromů.
            Bohužel ne samotné stromy. Vhodná data jsou z datové sady Přírodní parky a
            Významný krajinný prvek - registrovaný. Z těchto datových sad by se mohly
            doplnit nebo aktualizovat data v OSM.

    \item   Ortofoto obsahují letecké snímky (rastr) Prahy až s rozlišením 5cm
            na~pixel jak ve~viditelném spektru světla tak i v infračerveném.

            Ortofoto jsou vhodná pouze jako podkladový zdroj,
            například pro mapovací aplikaci JOSM.

    \item   Ovzduší, obsauje vektorové mapy znečištění ovzduší a také
            zdoje znečištění. Tyto data se v OSM nemapuje, jediné co by se
            mohlo přidat do OSM jsou bodové zdroje, a to značkou pro komín.
            Dále v této kategorii jsou i Bonity a to z různých hledisek (ovzduší, osvit...).
            Tyto data nejsou vhodná pro OSM.

    \item   Platný územní plán. V této kategorii jsou datové sady Veřejně prospěšných staveb
            (plošné, liniové a bodové). Datová sada VPS obsahuje také infomace o P+R u stanic metra.
            Data jsou z územního plánu, a tedy nemusí být realizovány.
            Po prozkoumání aktuálních a porovnání s datovou sadou Záchytná parkoviště P+R,
            jsou již tyto záznamy obsaženy v datové sadě Záchytná parkoviště P+R.

    \item   Socioekonomická data, obsahuje pouze vektorovou mapu ceny pozemků.
            Tyto data se do OSM neřidávají.

    \item   Technická infrastruktura - vodní hospodářství, obsahuje záplavová území
            Q20, Q50 a Q100 a uzemí zaplavené v roce 2013. Tyto data se do OSM
            nepřidávají. Dále v této kategorii jsou datové sady Protipovodňové ochrany,
            jsou zde protipovodnové, dočasné, zdi. Je možné tyto zdi přidat,
            ale jelikož se jedná o dočasné překážky a to ještě jen v době akutní povodně,
            nemjí smysl je do OSM přidávat.

    \item   Urbanismus z této kategorie by se dalo pro OSM využít informace
            z~datové sady Stavební uzávěry - dopravní. Ale by bylo nutné udržovat
            aktuálnost infomací o plánovaných dopravních uzavírkách.
            Dále by se mohla do OSM přidat počet pater budov z datové sady
            Podlaživost.
\end{itemize}


\subsection{licenční problém}
\label{licenční problém}
Jak již bylo řečeno výše, IPR svá data zveřejnil a stále zveřejňuje pod licencí CC-BY SA 4.0 viz \ref{licence CC BY-SA 4.0}.

Jelikož jsou ale data OSM distribuována pod licencí ODbL,
době tvorby této práce, neumožňuje jejich začlenění do databíze OSM.

Licence ODbl je licence zaměřená na databáze. Zaručuje licencční 
ochranu databáze jako celku. 

Na některá data (informace), jako například místní názvy (ulice, ..)
se nemohou vzdathovat autorská práva. Nebo na infomace, která 
nevyžadovala žádnou kretivitu pro "vytvoření" . 

Vyžaduje ale vzdání se licenčních nároků 
na data, která tam uživatel přidal. V některých zemích, kde se nelze 
úplně vzdát autoriskcýh práv, vyžaduje jejich vzdání se na nejnižší 
možnou míru a také souhlasí, že nebude vymáhat svoje autorská práva.
\cite{ODbl}

IPR byl kontaktován, zdali by mohla být jeho data použita do import.
Dle vyjádření vyplinulo, že IPR by neměl problém s použitím svých dat, jelikož
jsou šířena pod myšlenkou opendata. Byla navrhnuta možnost udělit od IPR vyjímku
licenčního použití pro OSM, avšak toto by neřešilo problém, že by tato data byla
k dispožici z OSM pro koncové uživatele pod licencí ODbL. Tedy co by někteý
uživatel nemohl s daty, distribuovaných přímo od IPR, provádět mohl by si je
obstarat z databáze OSM a pod licencí, která by jeho aktivitu neomezovala.

Vhodnější by bylo, změnit licencování dat zveřejněných IPR na ODbl.
Dle vyjádření se tato situace bude ze strany IPR řešit tak, že se změní 
licence u distribuovaných dat na ODbl. Tato změna licencování se očekává 
v~několika měsících.


\section{IprDownloader}
\label{Iprdownloader}
Jak již bylo řečeno tak IPR distibuje svá data a datové soubory přímo na své 
stránce. Data tam jsou distribuována možností odkazů ke stažení nebo je
k~dispozici kanál ve formátu ATOM ({\tt http://opendata.iprpraha.cz/feed.xml}).
Pomocí něho se lze také dostat ke všem distribuovaným datům.

Jelikož by bylo zdlouhavé stahovat všechny tyto soubory ručně, proto byl
vytvořen program, který umožňuje stahování a import dat. Skript byl napsán
v~programovacím jazyce Python a využívá knihovny GDAL.

Je rozdělen na tři soubory 
{\tt , IprDownloader.py , IprBase.py} a {\tt IprPg.py}.


\subsection{{\tt IprDownloader.py}}
Na primární skript {\tt IprDowloader.py}, který parsuje vstupní údaje a
zpracovává je k~dalšímu použití ve sriptu, viz \ref{argparse} .
Těmito základními údaji jsou
({\tt ---alike, ---crs, ---format, ---outdir ---d}). Dalšími vstupními údaji mohou
údaje o databázi, kam se mají data importovat. Těmi to vstupy jsou
({\tt ---dbname ---dbschema ---dbhost ---dbport ---dbuser ---dbpassword}).

Skript hlídá vstupní formát kartografických souřadnic a pokud se neshodují
s~možnými, oznámí chybu o špatném vstupu. Pokud uživatel vloží kartografické
zobrazení ve formě EPGS kodu, program jej převede na jeho název.


\subsection{{\tt IprBase.py}}
Skript {\tt IprBase.py} definuje třídu {\tt IprDownloader} se všemy 
potřebnými funkcemi. Většina funkcí slouží pro otevírání, čtení a parsování
XML souborů, viz \ref{xmltodict} .
Obsahuje funkci pro hledání, která prochází všechny záznamy a 
porovnává, zda vstupní přibližný název je obsažen v názvu souboru dat. Poté 
najde odkaz na xml souboru a otevře jej. Následuje vždy čtení a parsování. 
Poté v naparsovaných udajích najde dle nastavení {\tt crc, format} správný soubor 
a stáhne jej do definovaného adresáře.


\subsection{{\tt IprPg.py}}
Skript {\tt IprPg.py } definuje třídou {\tt IprDownloaderPg }, která dědí 
všechny funkce od třídy {\tt IprDownloader } a definuje další funkce potřebné pro 
správný import stažených dat do databáze PostgreSQL. 
Přesněji stažený soubor zkontroluje, jestli není v kompresován v archivu 
{\tt *.zip} , jeslit ano, extrahuje ho do nové složky se stejným jménem jako
archýv ve stejném adresáři jako archiv. Následně složí string pro PostgreSQL 
ze~vstupních údajů. Pro soubory ve formátu ESRI Shapefile opravý definici 
kartografického zobrazení uloženou v souboru {\tt *.prj }. 


\section{Ovládání}
Při pouhém spuštění skriptu vypíše všechna nalezená data, která IPR dává 
k~dispozici. Pro hledání a označení, která data se mají stánout slouží 
{\tt ---alike }. Za toto uživatel připojí slovo nebo frázi, jež chce v~datech
hledat. Pro~sousloví, které obsahuje mezeru, je vhodné jej umístit do uvozovek. 
Program rozlišuje velká~x~malá písmena a háčky a čárky.

příklad:

{\tt \$ iprdownloader.py ---alike 'Technická mapa'}
  

Dále jsou k dispozici nastavení.
\begin{itemize}
    \item kartografické zobrazení {\tt ---crs } . K dispozici jsou dvě
     (S-JTSK nebo WGS-84), respektive lze vkládat čtyři různé vstupy, protože 
     lze tyto kartografická zobrazení vložit i EPGS kody (5514 a 4326). 
     Přednastavené zobrazení je S-JTSK.
    \item datová formát souborů {\tt ---format} .
    
    Pro vektorové soubory je možné zvolit pro vektorová data:
        \subitem {\tt json }  JavaScript Object Notation
        \subitem {\tt dxf  }  AutoCAD DXF
        \subitem {\tt gml  }  Geography Markup Language
        \subitem {\tt shp  }  ESRI shapefile
        
    Je předdefinován formát {\tt shp} .
         
    Pokud uživatel chce stahovat rasterová data, je potřeba změnit předdefinovaný
    formát souboru na adekvártní rastrový formát např. ({\tt tif}) atd.
    
    \item adresu adresáře na místním počítači. Je předdefinován adresář 
    {\tt /data/} ve~složce, kde je uložen samotný program.     
\end{itemize}

Pro stažení dat stačí přidat parametr {\tt ---download} .

příklad:

{\tt \$ iprdownloader.py ---alike 'Technická mapa' ---crs 4326 ---format gmp ---outdir IPR}

Pro možnost stahované datové souboury importovat rovnou do databáze PostGIS, je 
minimálně nutné zadat název databáze {\tt ---dbname }. To v případě pokud je
databáze PostGIS umístěna v lokální síti. Pokud je databáze umístěna na jiném
serveru, je nutné zadat adresu {\tt ---dbhost} a port {\tt ---dbport} . Pokud je
ještě vyžadován autorizovaný přístup, použije se přístupové jméno {\tt ---dbuser} a
heslo {\tt ---dbpassword} . Je možné zvolit si schema v databázi {\tt ---dbschema} . 
Pokud je vložen název databáze, není již potřeba používat parametr {\tt ---download} .

Příklad:

{\tt \$ iprdownloader.py ---alike 'Technická mapa' ---dbname pgis\_osm\_jakl ---dbschema IPR ---dbhost geo102.fsv.cvut.cz  ---dbport 5432 } 
 

\section{Navržená data pro import}
\label{Navržená data pro import}

Po projití všech dat bylo vyhodnocena vhodna data pro začlenění do OSM. 
\begin{itemize}
    \item   Výstupy PID, obsahuje vstupy/výstupy z metra.
    \item   Odpadní zařízení pro občany, dle IPR obsahuje sběrny odpadu. 
    \item   Cyklistická doprava - značky, dle technické dokumentace by mělo 
            obsahovat bodové značky stojanů Bikesharing.
    \item   Veřejné toalety
    \item   Parkovací automaty
    \item   Záchytná parkoviště P+R
\end{itemize}

Všechna vybraná data byla pomocí IprDownloader.py stažena a naiportována 
do~školní databáze PostGIS. Poté byl program QGIS připojen na školní databázi.
Poté byla v programu QGIS pomocí SQL SELECTu filtrována data z aktuální
verze OSM na školním serveru a z každého SELECTu byla vytvořena nová tabulka 
v PostGIS databázi na školním severu. Data byla filtrována za použití doporučených
atributů na české osm wiki. \cite{OSMfeatures}

Tyto vhodná data byla projit a porovnána s aktuálním stavem v OSM.
Byly navrženy vhodné atributy který by se ze zdrojových dat daly naplnit.
Takto navržené data byla navržena na hromadná korespondenční konverzace talk-cz.


\subsection{Výstupy PID}
\label{Výstupy PID}
V datech od IPR bylo obsaženo celkem 353 záznamů. Jednalo se o vstupy/výstupy
z~metra (dveře, schodiště a výtahy).

V OSM databázi bylo hledání v tabulce {\tt osm.czech\_points} bylo použit
atribut
\begin{verbatim}
    railway = subway_entrance
\end{verbatim}
Vytvořená tabulka z dat OSM obsahovala 236 prvků.
Po porovnání dat bylo zřejmé, že nejvíce jsou v OSM zmapovány
výstupy v centru, kde jsou posuny v řádech decimetrů až metr. V okrajových
částech města nejsou zmapovány všechny výstupy nebo chybí výstupy pro celé
stanice. Protože body vstupů do metra jsou většinou součástí linií,
není vhodné body mazat a vytvářet nové. Bylo potřeba najít mezi stávajícími body
v~OSM a IPR adekvátní dvojce. To znamená, že se jedná o body, které reprezentují 
stejnou věc v~realitě.

Ani editaci stávajících není příliš přínosné, když všechny vstupy do metra jsou
minimálně 3 metry široké a všechny změny by byly v řádů cm až dm. 
Bylo by ale vhodné přidat vsupy do metra tam, kde nejsou v databázi OSM.
Jedná se o stanice metra Zličín, Luka, Kačerov, Rajská zahrada a Černý most. 
Celkem se tedy jedná pouze o 26 bodů, které by bylo snadnější přidat ručně, 
protože vstupy jsou dostatečně zmapované jen zde není na některém uzlu atribut {\tt railway~=~subway\_entrance}


\subsection{Odpadní zařízení pro občany}
\label{Odpadní zařízení pro občany}
Data od IPR obsahují seznam 36 směrných dvorů s velmi detailním popsání co 
jakou látku přijímá, otevírací dobu, zřizovatel. 

Z OSM z tabulky {\tt osm.czech\_point} byly prvky vyfiltrovány podle hodnot 
\begin{verbatim}
    amenity = recycling
    recycling_type = centre
\end{verbatim}    
Bylo nalezeno celkem 11 prvků, a z toho bylo 6 prvků z IPR ve vzdálenosti 
100 metrů od prvků z OSM. Těchto 6 prvků bude vhodné pouze přesunout. Zbylé 
prvky je nutné v databázi OSM nově vytvořit.

Z dostupných dat z tabulky byly navrženy tyto atributy: 
\begin{verbatim}
    amenity = recycling
    recycling_type = centre
    opening_hours =
    fee =
    operator = 
    source = 
    source:loc = IPR
\end{verbatim}
Tyto navržené atributy by se naplnily daty z tabulky dat z IPR.
Otevírací doba by byla vložena do {\tt opening\_hours~= }.
Hodnoty v tomto atributu by byly formátovány dle doporučeného formátování.\footnote{http://wiki.openstreetmap.org/wiki/Key:opening\_hours}
 U atributu {\tt fee~= }
bylo navrženo {\tt free of charge for the Prague citizens} , tedy zdarma pro občany Prahy.
Hodnota {\tt operator~= } by byla ze stejnojmeného sloupce z dat IPR.
Atribut {\tt source~= } by byl ze sloupce {\tt POSKYT }.

Dále byly navrženy atributy:
\begin{verbatim}
    recycling:paper = yes
    recycling:plastic = yes
    recycling:metal = yes
    recycling:wood = yes
    recycling: ... = yes
\end{verbatim}
Podle toho co zde lze recyklovat, dle hodnoty ve sloupci {\tt odpadprije }.


\subsection{Veřejné toalety}
\label{Veřejné toalety}
Datový soubor Veřejné toalety obsahuje jednu tabulku dat. V této tabulce je 
celkem 247 záznamů. U každého záznamu je uvedena lokalita, provozovatel. 
Dále je uvedena doba, po kterou je přístupný, zdali je nutné zaplatit poplatek,
jestli je bezbariérový a zdroj těchto dat.

Veřejné toalty byl v datech OSM byl vyhledán podle atribu
\begin{verbatim}
    amenity = toilets
\end{verbatim}

bylo zjištěno, toho času, 185 záznamů. Překrytí dat IPR a OSM bylo většinou
jeden bod z OSM a z IPR na tomto místě bylo, buď jeden ale posunutý, nebo
více bodů v okruhu několika metru. Na některých místech, převážně mimo centrum
Prahy a mimo linky metra, chyběli body veřejných toaletet úplně.
Bylo tedy vhodné stávající data doplnit data z IPR.

Na IPR dat z tabulky a základě doporučených atributů byly vybrány tyto atributy:
\begin{verbatim}
    acces = yes/permissive
    fee = no/yes
    wheelchair = yes/no
    opening_hours =
    operator =
    description =
    source = IPR
    source:location = IPR
\end{verbatim}
Pokud by hodnota byla {\tt fee~=~yes} připojila by se značka
\begin{verbatim}
    fee:price =
\end{verbatim}
hodnota by byla uvedena v Kč.

Hodnota u {\tt description~= } by byla hodnota ze sloupce {\tt lokalita} .
U atributu {\tt operator~= } byla navržena hodnota {\tt Hlavní Město Praha} ,
ale jen u~prvků, u~kterých je uvedeno HMP ve sloupci {\tt Operator}.
Tam kde není žádná hodnota, by se tento atribut nepřidal.


\subsection{Parkovací automaty}
\label{Parkovací automaty}
Soubor dat Parkovací automaty od IPR obsahovale celkem 452 záznamů parkovacích
automatů. Je zde uveden pouze TYP a Poskytovatel dat. Nelze tedy určit zdali
automat přijímá mimo, asi klasické, mince i bankovky, cizí měnu nebo i umožňuje
platbu platebními kartami.
Pro porovnání, byly vyhledány body z OSM s atributy:
\begin{verbatim}
    amenity = vendings_machine
    vending = parking_tickets
\end{verbatim}
Podle těchto parametrů bylo nalezeno pouze 8 bodových prvků v databázi OSM.
Pouze dva záznamy z OSM jsou v blízkosti k bodum z IPR.
Jeden je ve vzdalenosti okolo 5 metrů a druhý okolo 20 metrů. Nelze tedy určit
zda oba odkazují na stejný objekt nebo na dva různé.
Proto se původní prvky nebudou rušit. Nové prvky půjde zpětně rozlišit podle
atributu {\tt source:loc~= } .

U nově přidaných uzlů byly navrženy tyto atributy:
\begin{verbatim}
    amenity = vendings_machine
    vending = parking_tickets
    source = IPR
    source:loc = IPR
\end{verbatim}
Do hodnoty {\tt source~= } by byla použita hodnota ze sloupce {\tt POSKYT}) .
Hodnota u atributu {\tt source:loc~= } by byla použita {\tt IPR} .

\subsection{Záchytná parkoviště P+R}
\label{Záchytná parkoviště P+R}
Soubor dat Záchytná parkoviště P+R obsahoval 16 prvků.
V tabulce bohužel nejsou žádné jiné údaje, kromě polohy uložené v geometrii
a poksytovatel dat ve sloupci {\tt poskt}.

Z dat OSM byly vyhledány body pomocí atributu
\begin{verbatim}
    amenity = park_ride
\end{verbatim}
Tento atribut je stále ve fázi schvalování, tudíž nebyly nalezeny žádné záznamy.
Proto bylo vyzkoušeno hledání pomocí obecnějšího atributu
\begin{verbatim}
    amenity = parking
\end{verbatim}
Na území Prahy bylo nalezeno,toho času, 182 prvků.
Bylo vyzkoušeno hledání s přidaným atributem {\tt name~= }, který by obsahoval P+R.
Byl nalezen pouze jeden záznam.

Dále bylo vyzkoušeno hledání bodů s atributy 
\begin{verbatim}
    amenity = parking
    park_ride = yes
\end{verbatim}
Na území Prahy byl nalezen pouze jeden prvek.

Je tedy navrženo importovat všechny prvky z databáze IPR.
Navržené atributy jsou
\begin{verbatim}
    amenity = parking
    park_ride = yes
    source = HMP-URM
    source:loc = IPR
\end{verbatim}


Výše zmíněné návrhy byly takto zveřejněny v diskuzi na stránkách talk-cz.

\subsection{Budovy 3D - výšky budov}
\label{Budovy 3D - výšky budov}
V komentářích k bylo dále navrženo přidání výšek budov ze souboru dat Budovy 3D.
\begin{verbatim}
    building:height
\end{verbatim}
V návrhu bylo doporučeno označení pro zdroj tohoto importu, pro případné pozdější úpravy a opravy.
\begin{verbatim}
    source:building:level =
\end{verbatim}

Při rešerši dat uložencýh na školní geodatabázi bylo zjištěno,
že data stahovaná a ukládaná z OSM neobsahují potřebné údaje.
Přesněji dataset (tabulka) neobsahovala údaje výšky budov
(sloupc {\tt height} a sloupec {\tt building\:height}.
Proto bylo nutné základní schéma změnit tak, aby se tyto údaje také ukládaly.
Upravené schema bylo uloženo a bylo nastaveno jako výchozí.

Bylo také vyhodnocena že bude vhodné aby se ukládal údaj, zdali
je polygon samostatná budova nebo pouze část relace budovy.
K tomuto rozeznání je užíván atribut {\tt builing\:part},
a jeho hodnoty byly zaznamenány do stejnjmeného sloupce.


V souboru Budovy 3D je dále uložno 128 datových sed pro menší oblasti.
Tyto datové sady obsahují budovy jen z malé oblasti, protože by datová sada budov
z~celého území Prahy byl moc velký (až v řádů GB).
Každá datová sada je přibližně obdélníková oblast z území prahy
a má svůj originální název.

V tomto příkladě bude pracováno s datovou sadou {\tt BD\_Prah73} .
Po vybrání této jedné datové sady (oblasti) byla vytvořena minimální
ohraničující oblast okolo ní a přidána přesah o 100 metrů na každou stranu.
Tedy byl vytvořen obdélník okolo jedné datové sady.

Pomocí tohoto obdélníku byly vybrány budovy z OSM uvnitř tohoto obdélníku
a přitom byly vytvořeny nárazníkové zony (buffer) s hodnotou 3 m pro každou budovu.
Byl zvolen tento postup, protože budovy z IPR a OSM nemusí mít přesně stejnou
polohu a tvar (polygonů).Tento dočasný datasest byl nazván {\tt bud73\_OSM}.

Pro velikost nárazníkové zóny kolem budov z OSM se provedla analyza.
Modelováno bylo s hodnotami po~0.5~m od~0.5~m až do vzdálenost 5~m.
Při malé velikosti nárazníkové zóny nastalávala situace,
že v této zóně nebyla nalezena žádná budova (polygon budovy) z datasetu IPR.
Mohlo by se zdát, že čím větší nárazníková zóna kolem budov z OSM tím lépe.
Avšak, čím byla velikost nárazníkové zóny zvolena větší, tím více se stalo,
že v této nárazníkové zóně "uvízlo" více budov (polygonů) z datasetu IPR.
Bylo tedy nutné zvolit "zlatou střední cestu" tj. 3 m.

Další možností jak se vyvarovat aby v nárazníkové zoně kolem budov z OSM "uvízlo více
polygonů budov (z dataset IPR) je spouštět skript vícekrát a postupně volit větší
a větší nárazníkovou zónu. Skript je napsán tak aby vybral a vytvořil nárazníkovou zónu
, jen těm budovám (polygonům), které nemají atribut {\tt building:part} a ani
{\tt building:height}. Ovšem tento postup spouštět skript vícekrát je samozřejmě také
více časově náročný.

Dále byl proveden průzkum dat z {\tt BD\_Prah73} datové sady ze souboru Budovy 3D (z IPR).
Bylo zjištěno, že vždy polygony, které vytvářejí 3D model jedné budovy,
mají stejnou bodnotu ve sloupci {\tt id\_bud}.
V 3D geometrii těchto polygonů budov jsou uloženy všechny (x,y,z) souřadnice
vrcholů, jenž určují polygon.
Polygony se stejnou hodnotou {\tt id\_bud} lze dále dělit dle hodnoty
ve~sloupci {\tt typ} na
\begin{verbatim}
    dilci plocha stresni kruhove plochy
    komin
    sikma stresni plocha
    svisla obvodova stena
    vikyr_stresni nadstavba
    vytah_vetrani_klimatizace
    vodorovna stresni plocha
    vyznacena vez na strese
    zakladna vikyre, stresni nadstavby
    zakladova deska
 .. a další
\end{verbatim}

Poté byly z datové sady {\tt BD\_Prah73} vybrány všechny polygony, které
měly ve sloupci {\tt typ} obsahují frázi {\tt stresni}.
Geometrie těchto vybraných 3D polygonů byly opraveny funkcí {\tt ST\_MakeValid()}
a následně "přetvořeny" na 2D polygony funkcí {\tt ST\_Force\_2D()}.
Vynechala se jen Z-ová souřadnice z geometrie.
Kvůli některým situacím, kdy k sobě polygony přesně nedoléhaly, bylo nutné
vytvořit nich nárazníkovou zónu (buffer) a jako dostačující byla zvolena hodnota 0.1 m.
Následně byly tyto polygony takzvaně "rozpuštěny". Rozpuštění znamená,
polygony, které se dotýkají minimálně jednou společnou stranou sloučí
do~jednoho polygonu.
Rozpuštěny byly polygony navzajem jen ty, které měly stejnou hodnotu ve sloupci 
{\tt id\_bud}.
Tím se docílilo, každá budova bude reprezentována jenom jedním 2D polygonem.
Tenta dočasná datová sada byla nazvána {\tt bud73\_IPR}


Následně mohlo být pro každý polygon z~{\tt bud73\_OSM} hledány
všechny polygony z~{\tt bud73\_IPR}, které by byly uvnitř.
Na tuto operaci byla použita funkce
\begin{verbatim}
    ST_Within( bud73_OSM.geom , bud73_OSM.geom )
\end{verbatim}
Tímto byly nalezeny prvky z dat IPR, které jsou "přibližně na stejném místě"
jako budovy z OSM. Do tabulky o dvou sloupcích byly uloženy pouze údaje
{\tt gid} pro budovy z OSM a {\tt id\_bud} polygonu budovy (z dat IPR).
Tato do4asn8 tabulka byla nazvána {\tt gid\_id}.


Dále tedy bylo možné k těmto prvkům ({\tt gid} OSM budov) přiřadit výšku podle hodnoty {\tt id\_bud},
kterou mají všechny polygony tvořící jednu budovu stejnou.
Njeprve musely být vybrány pouze ty záznamy {\tt gid},
které jsou v tabulce {\tt git\_id} pouze jednou.
To jest polygon budovy z OSM, ve které byl nalezen pouze jeden polygon z datasetu IPR.
Následně byly k připojeny výšky
Pro nalezení výšky budovy\footnote{dle \url{http://wiki.openstreetmap.org/wiki/Cs:Simple_3D_Buildings}}
byly požity všechny všechny polygony tvořící budovu.
V~rámci jedné budovy (podle stejné hodnoty v {\tt id\_bud}, byla výška budovy vypočtena
jako rozdíl nejvyššího bodu a nejnižšího bodu budovy.
Přesněji byla vypočtena takto
\begin{verbatim}
    select	id_bud,
            max(maxZ) - min(minZ) as Height
    from(   select	id_bud,
                    ST_ZMax(geom) as maxZ,
                    ST_ZMin(geom) as minZ
            from    ipr.bd3_prah73
        ) as max_min
    group by id_bud
\end{verbatim}
Ve skriptu je tato dočasné tabulky označena {\tt osm\_height}.

Nakonec se získané hodnota výšky budovy přidá do puvodního datasetu (tabulky)
{\tt praha\_building\_osm}. Dočasné tabulky {\tt bud73\_osm, bud73\_ipr, gid\_id}
a {\tt osm\_height}.






