\chapter{Použité technologie}
\label{3-technologie}

\section{PostgreSQL}
\label{PostgreSQL}

PostgreSQL je open source objektově-relační databázový systém. Za dlouhou dobu vývoje se stal robustním a hojně využívaným řešením hlavně díky své spolehlivosti. Lze s ním pracovat ve všech známých operačních systémech. 
\cite{PostgreSQL}

\subsection{PostGIS}
\label{PostGIS}
PostGIS je nadstavba databázového systému PostgreSQL.
Rozšiřuje jej o funkce a~datové typy, které umožňují uložení, manipulaci a správu geografických objektů. Mezi podporované geometrické typy patří bod (point), linie (linestring),
nebo polygon (polygon). Dále je možno prvky sdružit 
do tzv. multiprvků (multiPoint, multiLineString nebo multiPolygon). Geometrie prvku určuje
jeho polohu v daném souřadnicovém systému. Data je tedy možno transformovat do jiných PostGISem podporovaných souřadnicových systémů. Systém kromě základních operací jako výpočet délky, obvodu, plochy, podporuje i sofistikovanější funkce jako je např. výpočet obalové zóny a další.

Sotware PostGIS je šířen pod GNU General Public License (GPLv2).

\section{Python}
\label{Python}
Programovací jazyk Python vznikl v roce 1991. Navrhl ho Guido van
Rossum. Je vyvíjen jako open source projekt. Inspiroval se hlavně
programovacím jazykem ABS, který byl přímo vytvořen pro~výuku 
začátečníků v programování. Python je proto jeden z nejvýhodnějších
programovacích jazyků pro začínající programátory, ale i přesto ho lze
použít pro~praktické programování.

Jedná se o jednoduchý programovací jazyk podporující různá
programovací paradigmata, hlavně objektové, imperativní, procedurální
a v~omezené míře i funkcionální. Je multiplatformní, lze jej tedy
použít na~různých operačních systémech. Klade velký důraz na~syntaxi 
psaného kódu, využívá hlavně prázdné znaky v~psaném kódu.

Programy nebo skripty lze psát libovolném textovém editoru. Vhodné je
použít standardu PEP~8, který definuje správnou strukturu odsazení kódu.

V současné době je vydána stabilní verze 3.5.7, která oproti předchozí
verzi 2.x mění některé již \uv{zažité} vlastnosti jazyka. Například 
nejviditelnější je změna funkce {\tt print()}. Ve verzi 2.x stačilo napsat
 {\tt print 'Ahoj'}. Ve verzi 3.x je již nutné řetězec vložit
do~závorek  {\tt print('Ahoj')}. Tyto dvě dnes používané verze 2.x
a 3.x jsou navzájem nekompatibilní.
\cite{python} 
\cite{wikiPython} 
  
\subsection{Knihovna argparse}
\label{argparse} 
Knihovna argparse je knihovna 
programovacího jazyka Python, která řeší konzolový vstup do aplikace a 
zpracovává ho pro~další použití v programu.\cite{argparse}

\subsection{Knihovna xmltodict}
\label{xmltodict} 
Tato Python knihovna umí číst soubory ve formátu XML a zpracovat jejich obsah do datové struktury, která je jednodušší pro manipulaci v samotném
programu. Data z~XML zpracuje do~slovníku (dictionary), kde klíčové slovo (key) reprezentuje hodnota(value) ve slovníku.
Hodnotami mohou být dále i řady dalších klíčových slov (key).\cite{xmltodict}

Příklad zpracování dat ve formátu XML.

{\scriptsize
\begin{lstlisting}
<head>
  <many>ele1</many>
  <many>ele2</many>
</head>
\end{lstlisting}
}

Zpracování dat provedeme v Pythonu příkazem:

{\scriptsize
\lstset{language=Python}
\begin{lstlisting}
with open('example.xml') as fd: 
  doc = xmltodict.parse(fd.read()) 
\end{lstlisting}
}

Objekt {\tt doc} vypadá následovně:

{\scriptsize
\lstset{language=Python}
\begin{lstlisting}
OrderedDict([(u'head', OrderedDict([(u'many', [u'ele1', u'ele2'])]))]) }
\end{lstlisting}
}

Data v této struktuře lze již jednoduše procházet.

{\scriptsize
\lstset{language=Python}
\begin{lstlisting}
doc['head']['many'][0]
ele1
\end{lstlisting}
}

\subsection{Knihovna GDAL}
\label{GDAL}
Jedná se o knihovnu pro práci s geografickými daty umožňující jejich
čtení a~zápis. Je napsána v programovacím jazyce C++. Je považována
za~jeden z hlavních open~source projektů hojně využívaných ve~sféře
GIS. Podporuje velkou škálu rastrových a vektorových formátů.
Díky knihovně PROJ.4 podporuje celou škálu souřadnicových systémů a umožňuje transformaci dat mezi nimi. \cite{GDAL}

\section{QGIS}
\label{QGIS}
QGIS je program vytvořený za~účelem prohlížení a editace geografických
dat. Jedná se o open source software. Je tedy možné, aby na~jeho vývoji
spolupracoval kdokoli. Vlastní uživatelské nástroje se píší ve formě tzv. zásuvných modulů, které je možné implementovat v jazyce C++ nebo Python.
Tyto moduly si ostatní uživatelé mohou poté stáhnout a začlenit do své instalace QGISu. QGIS také umožňuje využívat jazyk Python v rámci vestavěné konzole. Lze tak vytvářet vlastní jednoduché skripty v jazyku Python a spouštět
je přímo v~QGISu.

Od svého vzniku byla vydána již spousta verzí tohoto programu. Dříve byly
verze pojmenovávány podle měsíců planet Jupitera a Saturnu, později se začaly používat jména měst. V současnosti (duben 2016) je
k~dispozici verze 2.14 Essen.
