\chapter{Použité technologie}
\label{3-technologie}

\section{PostgreSQL}
\label{PostgreSQL}
PostgreSQL je Opensource databázový systém, který je vyvíjen déle než patnáct
let. Za dlouhou dobu vývoje se stal robustní a také hojně využívaný.
Postupem času získal svou spolehlivostní silnou reputaci.
Lze s ním pracovat ve všech známých operačních systémech. 
\cite{PostgreSQL}

\subsection{PostGIS}
\label{PostGIS}
PostGIS je objektově-relační nadstavba databázové struktury PostgreSQL.
Rozšiřuje ji o funkce, které umožňují geometrické operace s objekty
v~ní uložené. Třídy prvků mohou být buď bod (point), linie (line),
nebo polygon (polygon). Dále je možné více prvků sdružit
do~multiPoint, multiLine, nebo multiPolygon. Geometrie prvku určuje
jeho polohu v určité souřadnicové soustavě. Dále určuje i
kartografickou soustavu, ve~které je jeho geometrie určena. Je tedy
dále možné při známých transformacích mezi soustavami provádět
transformace souřadnic. Základní operace jako délka, obvod, plocha,
ale jsou zde i sofistikovanější funkce, které lze s objekty nebo
i mezi objekty provádět. 

Sotware PostGIS je šířen pod GNU General Public License (GPLv2).

\section{Python}
\label{Python}
Programovací jazyk Python vznikl v roce 1991. Navrhl ho Guido van
Rossum. Je vyvíjen jako open source projekt. Inspiroval se hlavně
programovacím jazykem ABS, který byl přímo vytvořen pro~výuku 
začátečníků v programování. Python je proto jeden z nejvýhodnějších
programovacích jazyků pro začínající programátory, ale i přesto ho lze
použít pro~praktické programování.

Jedná se tedy o jednoduchý programovací jazyk podporující různá
programovací paradigmata, hlavně objektové, imperativní, procedurální
a v~omezené míře i funkcionální. Je multiplatformní, lze jej tedy
použít na~různých operačních systémech. Klade velký důraz na~syntaxi 
psaného kódu, využívá hlavně prázdné znaky v~psaném kódu.

Programy nebo skripty lze psát v textovém editoru, ale je vhodné
použít standartu PEP~8, který kontroluje prázdné znaky a správnou
strukturu skriptu. 

V současné době je vydána stabilní verze 3.5.7, která vůči předchozí
verzi 2.x mění některá, již "zažitá" syntaxe. Například 
nejviditelnější je změna funkce {\tt print } pro verzi 2.x stačilo
 {\tt print 'Ahoj'}  a pro verzi 3.x je již nutné string vložit
do~závorek  {\tt print('Ahoj')}  . Tyto dvě dnes používané verze 2.x
a 3.x jsou navzájem nekompatibilní.
\cite{python} 
\cite{wikiPython} 
  
\subsection{knihovna argparse}
\label{argparse} 
Knihovna argparse, jejíž autor je Tshepang Lekhonkhobe, je knihovna 
do~programovacího jazyku Python. Řeší konzolový vstup do aplikace a 
zpracovává ho pro~další použití v programu.\cite{argparse}

\subsection{knihovna xmltodict}
\label{xmltodict} 
Tato knihovna umí číst soubory ve formátu XML a zpracovává je
do~snadnější formy dat, která je i jednodušší pro manipulaci samotného
programu. Data z~XML zpracuje do~formy datové řady (array), kde každý
člen této řady může být opět řada dat.\cite{xmltodict}

Příklad parsování dat ve XML formátu.

{\scriptsize
\begin{lstlisting}
<head>
  <many>ele1</many>
  <many>ele2</many>
</head>
\end{lstlisting}
}

Po otevření a parsování se v Python(u) provede příkazem.

{\scriptsize
\lstset{language=Python}
\begin{lstlisting}
with open('example.xml') as fd: 
  doc = xmltodict.parse(fd.read()) 
\end{lstlisting}
}

Objekt {\tt doc} má poté podobu.

{\scriptsize
\lstset{language=Python}
\begin{lstlisting}
OrderedDict([(u'head', OrderedDict([(u'many', [u'ele1', u'ele2'])]))]) }
\end{lstlisting}
}

V této datové řadě lze již snadno procházet. 

{\scriptsize
\lstset{language=Python}
\begin{lstlisting}
doc['head']['many'][0]
ele1
\end{lstlisting}
}

\subsection{knihovna GDAL/OGR}
\label{GDAL/OGR}
Jedná se o knihovnu pro práci s geografickými daty a umožňuje jejich
čtení a zápis. Je psána v programovacím jazyce C++. Je považován
za~jeden z hlavních open~source projektů a je hojně používán ve~sféře
GIS. Pracuje s vektorovými a rastrovými datovými formáty. Podporuje
velkou škálu souborových formátů, a to pro~vektorová data, ale i
pro~rastrová. Obsahuje databázi definic kartografických projekcí
používaných na~celém světě a databázi jejich známých transformací.
Umožňuje tedy komplexní operace s geografickými daty. \cite{GDAL}

\section{QGIS}
\label{QGIS}
QGIS je program vytvořen za~účelem prohlížení a editace geografických
dat. Jedná se o Opensource software. Je tedy možné, aby na~jeho vývoji
spolupracoval kdokoli. Je možné si vytvořit novou funkci, nebo přímo
zásuvný modul do programu. Je možné je psát v jazyce C++ nebo Python.
Tyto moduly si uživatelé mohou poté stáhnout a přidat do~QGISu a 
využívat. QGIS také umožňuje využívat konzolově jazyk Python. Je možné
si tak vytvářet vlastní jednoduché skripty v jazyce Python a spouštět
je v~QGISu.

Od svého vzniku byla vydána již spousta verzí programu.Dříve byly
verze pojmenovávány podle měsíců planet Jupitera a Saturnu, později
a dosud se dávají jména měst. V současnosti (duben 2016) je
k~dispozici verze 2.14 Essen.
