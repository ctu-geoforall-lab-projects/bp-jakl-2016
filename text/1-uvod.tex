\chapter{Úvod}
\label{1-uvod}

K projektu OpenStreeMap (OSM) jsem se dostal již před několika lety.
Zaujala mě možnost sám tvořit a upravovat mapu.
Přidávat nové infomace ze svého okolí vlastním sběrem dat
a poté využívat vytvořenou mapu.

Postupem času jsem zjistil, co je, a co již není vhodné vkládat do mapy.
Když mi bylo na Fakultě stavební ČVUT v Praze nabídnuto dělat
bakalářskou práci na téma, které se dotýká problematiky datových importů OSM,
neváhal jsem. Nedávno totiž Institut plánování a rozvoje hlavního města Prahy
(IPR) uveřejnil všechna svá geografická data.
Tím vznikla možnost začlenit některá vhodná data do projektu OSM.

Obsahem této práce je čtenáře nejprve seznámit s projektem
OpenStreetMap. Přiblížit mu vznik a vývoj projektu, užití a licenci.
Jendá se o cíl, kam se budou importovat zpracovaná data z IPR.
Dále přiblížit problematiku datových importů do OSM a vysvětlit
pojem Opendata.

V Praktické čísti navrhnout program v programovacím jazyce Python,
který by tato data stahoval z~IPR a následně importoval do pracovní
geodatabáze PostGIS a popsat ovládání vytvořeného programu.

Následně provést rešerši zveřejněných dat IPR a porovnat je s daty,
která již jsou v OSM. Navrhnout, jaká data by byla vhodná importovat.
V průběhu práce konzultovat tento záměr s~komunitou OSM.
Nakonec vybraná vhodná data připravit k importu z~geodatabáze PostGIS
do OSM.
