\chapter{Úvod}
\label{1-uvod}

%%% ML: obecne receno by chtelo vylepsit cesky jazyk v celem textu,
%%% jeho kvalita pokulhava

K projektu OpenStreeMap (OSM) jsem se dostal již před několika lety.
Zaujala mě možnost sám tvořit a upravovat mapu.
Přidávat nové infomace ze svého okolí vlastním sběrem dat
a poté využívat vytvořenou mapu.

Postupem času jsem zjistil, co je, a co již není vhodné vkládat do mapy.
Když mi bylo na Fakultě stavební ČVUT v Praze nabídnuto dělat
bakalářskou práci na téma, které se dotýká problematiky datových importů OSM,
neváhal jsem. Nedávno totiž Institut plánování a Rozvoje hlavního města Prahy
(IPR) uveřejnil svá všechna geografická data.
Tím se vznikla možnost začlent některá vhodná data do projektu OSM.

Obsahem této práce je čtenáře nejprve seznámit s projektem OpenStreetMap.
Přiblížit mu jeho vývoj a účel jeho vzniku, užití a licenci.
Protože se jedná o cíl, kam se budou importovat zpracovaná data z IPR.

Dále přiblížit problematiku datových importů do tohoto projektu. Uvést některé
problémy, které se v minulosti u již proběhlých importech objevily, a jak se
řešily. Rozebrat možné další problémy, které mohou vzniknout při importaci dat
z~IPR a navrhnout jejich možná řešení.

%%% ML: zde se lisi forma, vz "Dale priblizit" vs. "Navrh skriptu"
V Praktické čísti navrhnout program v programovacím jazyce Python,
který by tato data stahoval z~IPR a následně importoval do pracovní geodatabáze
PostGIS. Popisat ovládání vytvořeného programu.

Analizovat data zveřejněná IPR a porovnat je s daty, která již jsou v OSM.
Navrhnout, jaká data by byla vhodná importovat.
Konzultovat tento záměr s~komunitou OSM.
Nakonec vybraná vhodná data připravit k importu z~geodatabáze PostGIS do OSM.
