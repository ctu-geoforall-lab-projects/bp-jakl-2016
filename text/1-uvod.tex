\chapter{Úvod}
\label{1-uvod}

K projektu OpenStreetMap (OSM) jsem se dostal již před několika lety.
Zaujala mě možnost sám tvořit a upravovat \uv{mapu}.
Přidávat nové infomace ze svého okolí vlastním sběrem dat
a poté využívat takto vytvořenou mapu.

Postupem času jsem zjistil, co je, a co již není vhodné vkládat do mapy.
Když mi bylo na Fakultě stavební ČVUT v Praze nabídnuto vypracovat
bakalářskou práci na téma, které se dotýká problematiky datových importů OSM,
neváhal jsem. Nedávno totiž Institut plánování a rozvoje hlavního města Prahy
(IPR) uveřejnil všechna svá geografická data.
Tím vznikla možnost začlenit další vhodná data do projektu OSM.

Obsahem této práce je čtenáře nejprve seznámit s projektem
OpenStreetMap. Přiblížit mu vznik a vývoj projektu, užití a jeho licenci.
%%% ML: tato veta nedava smysl, co jste chtel presne rici
Jendá se o cíl, kam se budou importovat zpracovaná data z IPR.
Dále přiblížit problematiku datových importů do OSM a vysvětlit
pojem otevřená data (Open Data).

V praktické části navrhnout aplikaci v programovacím jazyce Python,
která by data poskytovaná IPR umožnila dávkově stáhnout a následně
importovat do pracovní geodatabáze PostGIS a popsat ovládání
vytvořené aplikace.

Následně provést rešerši zveřejněných dat IPR a porovnat je s daty,
která již jsou v OSM. Navrhnout, jaká data by byla vhodná importovat.
V průběhu práce konzultovat tento záměr s~komunitou OSM.
Nakonec vybraná data připravit k~importu z~geodatabáze PostGIS
do OSM.
