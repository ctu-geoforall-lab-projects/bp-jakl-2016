\chapter{Úvod}
\label{1-uvod}

%%% ML: prvni dve vety preformulovat (2x moznost, "a zaujala me..." +
%%% "videt vysledek" nezni hezky cesky..)

%%% ML: obecne receno by chtelo vylepsit cesky jazyk v celem textu,
%%% jeho kvalita pokulhava

K projektu OpenStreetMap (OSM) jsem se dostal již před lety a zaujala mě možnost pomáhat tvořit 
a upravovat mapu. Možnost přidávat a opravovat objekty ze svého okolí a následně 
vidět výsledek a možnost ho využívat. Postupem času jsem zjistil, co je vhodné a 
co již nevhodné vkládat do mapy. Když mi bylo na Fakultě stavební ČVUT v Praze nabídnuto dělat 
bakalářskou práci na téma, které se dotýká problematiky datových importů OSM,
%%% ML: pri prvni pouziti zkratky by mela byt rezepsana, viz OSM vyse, zde IPR
%%% ML: "data z IPR" zni divne
neváhal jsem. Nedávno totiž byla zveřejněna data z IPR a vznikla možnost ujmout 
se začlenění některých těchto dat do projektu OSM. 

Obsahem této práce je čtenáře nejprve seznámit s projektem OpenStreetMap. 
Přiblížit mu jeho vývoj a účel jeho vzniku. Jeho možnosti užití a licenci, 
%%% ML: hlavni misto? mluvit o uziti a licenci...
jakožto hlavní místo, kam se bude soustředit konečný výsledek této práce.

Dále přiblížit problematiku datových importů do tohoto projektu. Uvést některé
%%% ML: objevily a resili...
problémy, které se v minulosti u již proběhlých importech objevily a jak se 
řešili. Rozebrat možné další problémy, které mohou vzniknout při importaci dat 
z~IPR a navrhnout jejich možná řešení. 

%%% ML: zde se lisi forma, vz "Dale priblizit" vs. "Navrh skriptu"
Návrh skriptu v programovacím jazyku Python, který by tato data stahoval z~IPR 
a následně importoval do pracovní geodatabáze PostGIS. Popis použitých knihoven a 
tříd funkcí v~knihovně GDAL.

Analizovat data zveřejněná IPR a porovnat je s daty, která již jsou v OSM. 
Navrhnout, jaká data by byla vhodná importovat. Konzultovat tento záměr
%%% ML: komunita ... poradit se s nimi (klidne bych druhou cast vety vynechal)
s~komunitou OSM a poradit se s~nimi v~případě některých otázkek. Nakonec 
vybraná data připravit k importu z~geodatabáze PostGIS do OSM metodou „SQL dump“.
