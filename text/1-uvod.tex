\chapter{Úvod}
\label{1-uvod}

K projektu OSM jsem se dostal již před lety a zaujala mě možnost pomáhat tvořit a upravovat mapu. Možost přidávat a opravovat objekty ze svého okolí a následně vidět výsledek a možnost ho využívat. Postupem času jsem zjistil co je vhodné a co již nevhodné vkládat do mapy. Když mi bylo na škole nabídnuto dělat bakalářkou práci na téma jež se dotýká problematiky dávkových importů OSM, neváhal jsem.
Nedávno totiž byla zveřejněna data z IPR a vznikla možnost ujmout se začlenění některých těchto dat do projektu OSM. 


Obsahem této práce je čtenáře nejprve seznámit s projektem OpenStreetMap. Přiblížit mu jeho vývoj a účel jeho vzniku. Jeho možnosti užití a licenci. Jakožto hlavní místo, kam se bude soustředit konečný výsledek této práce.

Dále přiblížit problematiku datových importů do tohoto projektu. Uvést některé problémy, které se v minulosti u již proběhlých importech objevily a jak se řešili. Rozebrat možné další problémy, které mohou vzniknout při importaci dat z IPR a navrhnout jejich možná řešení. 

Návrh skriptu v programovacím jazyku Python, který by tato data stahoval z IPR a následně importoval do pracovní databáze postGIS. Popis použitých knihoven a tříd funkcí v knihovně GDAL.

Analizovat data zveřejněná IPR a porovnat je s daty která již jsou v OSM. Navrhnout, která data by byla vhodná importovat. Konzultovat tento záměr s komunitou OSM a poradit se s nimi v případě některých otázkách. Nakonec vybraná data připravit k importu z databáze PostGIS do OSM metodou „SQL dump“.
