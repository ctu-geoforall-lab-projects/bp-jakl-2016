\chapter*{Závěr}
\label{5-zaver}

V~bakalářské práci bylo popsáno základní dělení prvků v~databázi
OpenStreetMap, licencování a hlavní zdroje dat. Vysvětlen termín
Opendata (otevřená data) a jeho rozdělení.
Při~řešení licenční otázky byl zjištěn problém, protože licence dat
OSM a IPR jsou rozdílné, neumožňují prozatím v~současné době začlenění
dat IPR do~databáze OSM a popsány hlavní rozdíly mezi těmito licencemi.
Problém byl řešen tak, že se vyčká na~změnu licence u~dat IPR, a
v~tomto mezičase se import připraví, aby se poté mohl už jen provést.


Pro~snadnější práci s~daty byl vytvořen program, který umožňuje hledání
všech distribuovaných dat od~IPR a dále umožňuje jejich stažení a
následný import do~geodatabáze PostgreSQL. Program byl napsán v~jazyce
Python.


Všechna data zveřejněná IPR byla prozkoumána (v~programu QGIS)
a bylo posouzeno, zdali jsou či nejsou vhodná začlenit je do~databáze
OpenStreeMap. Po~vybrání vhodných začlenění do~databáze OSM byla
provedena rešerše aktuálního stavu v~databázi OSM.
Po~porovnání vzešly čtyři datové sady vhodné pro~import.
Jmenovitě Parkovací automaty, Parkoviště P+R, Odpadní zařízení
pro~občany a Veřejné toalety. Následně z~nich byla vytvořena a upravena
data s~ohledem doporučených atributů OSM.
Na~diskuzním Talk-cz byl tento záměr zveřejněn a v~diskuzi dále bylo
navrženo, zabývat se ještě importem výšek budov.


Po~změně licence u~dat zveřejňovaných IPR se samotný import může provést
v~programu {\tt osmosis}.
Jako zajímavé by se do~budoucna jevilo zpracovat 3D modely budov
z~datasetu Budovy 3D. Navrhnout program (skript), který by vytvořil
modely budov v~OSM.
