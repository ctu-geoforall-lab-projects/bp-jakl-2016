\chapter{Závěr}
\label{5-zaver}

V~bakalářské práci bylo popsáno základní dělení prvků v~databázi
OpenStreetMap, licencování a hlavní zdroje dat. Dále byl vysvětlen termín
Opendata (otevřená data).
Při~řešení licenční otázky byl zjištěn problém. Licence dat
OSM a IPR nejsou kompatibilní. Nelze tedy v~současné době začlenit
data IPR do~databáze OSM. V rámci práci byly popsány hlavní rozdíly mezi těmito licencemi.
Problém byl řešen tak, že se vyčká na~změnu licence u~dat IPR, a
v~tomto mezičase se import připraví, aby se poté mohl už jen provést.


Pro~snadnější práci s~daty byl vytvořen program, který umožňuje hledání
všech distribuovaných dat IPR a dále jejich stažení a
následný import do~geodatabáze PostGIS. Program byl napsán v~jazyce
Python.


Všechna data zveřejněná IPR byla prozkoumána (v~programu QGIS)
a bylo posouzeno, zdali jsou či nejsou vhodná začlenit je do~databáze
%%% ML: zacleneni ?
OpenStreetMap. Po~vybrání vhodných začlenění do~databáze OSM byla
provedena rešerše aktuálního stavu v~databázi OSM.
Na základě toho vzešly čtyři datové sady vhodné pro~import.
Jmenovitě Parkovací automaty, Parkoviště P+R, Odpadní zařízení
pro~občany a Veřejné toalety. Následně z~nich byla vytvořena a upravena
data s~ohledem doporučených atributů OSM.
Na~diskuzním fóru Talk-cz byl tento záměr zveřejněn a v~diskuzi dále bylo
navrženo, zabývat se ještě importem výšek budov.


Po~změně licence u~dat zveřejňovaných IPR se samotný import může provést
v~programu {\tt osmosis}.
Jako zajímavé by se do~budoucna jevilo zpracovat 3D modely budov
z~datasetu Budovy 3D. Což by znamenalo navrhnout program (skript), který by vytvořil
modely budov v~OSM.
