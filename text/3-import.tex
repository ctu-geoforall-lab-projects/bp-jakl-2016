\chapter{Importy}% je videt
\label{2-importy}

Hromadné datové importy do OSM jsou cenným zdrojem dat. V rámci České 
republiky proběhlo již několik dávkových importů. Většina z nich byla 
začlenění dat ze státních databází. Hlavní výhoda je jejich celistvost 
v rámci státu nebo oblasti, nevýhodou může být ne vždy aktuálnost 
informací. Některá data mohou být sbírána a zveřejněna i s delším 
časovým odstupem.   

Samotný proces importu vykonají stoje respektive výpočetní technika a 
není to tedy moc náročné na čas a lidské zdroje jako samotný sběr. 
Avšak větší časovou náročnost zabere příprava, a to jak samotných dat 
nebo napsání programu (skriptu) tak i probrání problematiky s~komunitou OSM. 

Větší čas zabere vybrat, jaká data jsou vhodná a přínosná. Prozkoumat 
zdali již nejsou v databázi částečně zanesena od jednotlivých 
uživatelů vlastním sběrem dat. Není příliš vhodné odstraňovat z mapy 
něco, co tam některý uživatel s~velkým úsilím zanesl. Také se může 
stát, že uživatel se znalostní místních poměrů bude vědět více, jak se 
věci mají v jeho okolí, něž-­li uživatel dělajíc import od stolu 
(například malé vodní toky, jejichž koryto se hýbe). Proto se volí 
řešení s~vytvořením duplicit. Je totiž snažší odstranit duplicitně 
vloženou správnou informaci, než-li po vymazání původní a nahrání nové 
řešit, že ta původní jedna byla správně. Toto lze řešit, kdyby tato 
jedna změna byla v jednom changelist­-u. Tuto změnu by bylo možné 
vrátit, což v řešení hromadných importů, kdy je mnoho změn, není 
možné.

Je tedy nutné zvážit velikost takto vytvořených duplicit. A rozhodnout 
jak postupovat při jejich pozdějším řešením. Zdali autor původních dat 
nečerpal ze stejného zdroje nebo zdali nejsou staršího data a bude 
tedy snadné vyřešit, která data ponechat. 

Je taky potřeba řešit otázku „tagování“. Problém pravděpodobně nebude 
přiřadit hlavní atribut dané „věci“ (například 
„řeka“~~waterway=river“) ale atributy, jež nejsou v mapě 
vykreslovány, ale odliší od sebe prvky nebo je zařadí do společné 
oblasti (atribut is\-in, city.. atd). Toto je vhodné nejprve vyřešitpo 
poradě s komunitou OSM na diskuzi task­osm.cz

\section{Licenční otázka}
\label{2-Importy}

Po vyřešení otázky, která data začlenit do databáze OSM a s jakým 
klíčem, je také nutné vyřešit licenční otázky. Není určeno, v jakém 
pořadí by se mělo toto řešit. Je spíše vhodné řešit nejprve licenční podmínky. 
Zdali autor nebo vlastník je nakloněn myšlence opendata.  

Pokud by například vlastník dat, která chce zveřejnit nebo již 
zveřejnil, nebyla pod licencí kompatibilní s OSM. Tedy nebyla by 
pod~licencí ODbL. Tento problém leze řešit tak, že buď by se počkalo
na~změnu licence nebo by se musela udělit speciální výjimka pro OSM.
Bohužel by tato výjimka  problém neřešila, jelikož byla data z OSM dále
distribuována pod licencí ODbL. Lepším řešení je zveřejňovat otevřená
data pod vícero licencemi. Popřípadě pod licencí, která je „kompatibilní“ 
s~ostatními nebo s~předchozími. Například CC­~BY~SA~4.0 je zpětně kompatibilní 
s~licencemi ...   ale zpětně to nemusí platit.  

