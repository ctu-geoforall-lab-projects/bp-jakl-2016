\chapter{Praktická část}
\label{3-Praktická část}

Jak již bylo řečeno tak IPR distibuje svá data a datové soubory přímo na své 
stránce. Data tam jsou distribuována možností odkazů kde stažení nebo je
k~dispozici kanál ve formátu ATOM ({\tt http://opendata.iprpraha.cz/feed.xml}). 
Pomocí něho se lze také dostat všem distribuovaným datům 


\section{IprDownloader}
\label{Iprdownloader}
Jelikož by bylo zdlouhavé stahovat všechny tyto soubory ručně, byl vytvořen 
program, který umožňuje stahování a import dat. Skript byl napsán 
v~programovacím jazyce Python a využívá knihovny GDAL. 

Je rozdělen na tři soubory 
{\tt , IprDownloader.py , IprBase.py} a {\tt IprPg.py}.


\subsection{{\tt IprDownloader.py}}
Na primární skript {\tt IprDowloader.py}, který parseruje vstupní údaje a 
zpracovává je k~dalšímu použití ve sriptu. Těmito základními údaji jsou 
({\tt --alike, --crs, --format, --outdir --d}). Dalšími vstupními údaji mohou 
údaje o databázi, kam se mají data importovat. Těmi to vstupy jsou 
({\tt --dbname --dbhost --dbport --dbuser --dbpassword}).

Skript hlídá vstupní formát kartografických souřadnic a pokud se neshodují 
s~možnými oznámí chybu o špatném vstupu. Pokud uživatel vloží kartografické 
zobreazení ve formě EPGS kodu program jej převede na jeho název. 


\subsection{{\tt IprBase.py}}
Skript {\tt IprBase.py} definuje třídu {\tt IprDownloader} se všemy 
potřebnými funkcemi. Většina funkcí slouží pro otevírání, čtení a parserování
XML souborů. Obsahuje funkci pro hledání, která prochází všechny záznamy a 
porovnává, zda vstupní přibližný název je obsažen v názvu souboru dat. Poté 
najde odkaz na xml souboru a otevře jej. Následuje vždy čtení a parserování. 
Poté v naparserovaných udajích najde dle nastavení {\tt crc, format} správný soubor 
a stáhne jej do definovaného adresáře.


\subsection{{\tt IprPg.py}}
Skript {\tt IprPg.py } definuje třídou {\tt IprDownloaderPg }, která dědí 
všechny funkce od třídy {\tt IprDownloader } a definuje další funkce potřebné pro 
správný import stažených dat do databáze PostgreSQL. 
Přesněji stežený sboubor zkontroluje jeslit není v kompresován v archivu 
{\tt *.zip} , jeslit ano, extrahuje ho do nové složky se stejným jménem jako
archýv ve stejném adresáři jako archiv. Následně složí string pro PostgreSQL 
ze~vstupních údajů. Pro soubory ve formátu ESRI Shapefile opravý definici 
kartografického zobrazení uloženou v souboru {\tt *.prj }. 


\section{Ovládání}
Při pouhém spuštění skriptu vypíše všechna nalezená data, která IPR dává 
k~dispozici. Pro hledání a označení, která data se mají stánout slouží 
{\tt ---alike }. Za toto uživatel připojí slovo nebo frázi, jež chce v~datech
hledat. Pro~sousloví, které obsahuje mezeru je vhodné jej umístit do uvozovek. 
Program rozlišuje velká~x~malá písmena a háčky a čárky.

Příkad:

{\tt \$ iprdownloader.py ---alike 'Technická mapa'}
  

Dále jsou k dispozici nastavení.
\begin{itemize}
    \item kartografické zobrazení {\tt ---crs } . K dispozici jsou dvě
     (S-JTSK nebo WGS-84), respektive lze vkládat čtyři různé vstupy, protože 
     lze tyto kartografická zobrazení vložit i EPGS kody (5514 a 4326). 
     Přednastavené zobrazení je S-JTSK.
    \item datová formát souborů {\tt ---format} .
    
    Pro vektorvé soubory je možné zvolit pro vektorová data:
        \subitem {\tt json }  ..   
        \subitem {\tt dxf  }  ..   
        \subitem {\tt gml  }  ..                    
        \subitem {\tt shp  }  ESRI shapefile
        
    Je předdefinován formát {\tt shp} .
         
    Pokud uživatel chce stahovat rasterová data je potřeba změnit předdefinovaný
    formát souboru na adekvártní rastrový formát ({\tt tif,...})
    
    \item adresu adresáře na místním počítači. Je předdefinován adresář 
    {\tt /data/} ve~složce, kde je uložen samotný program.     
\end{itemize}

Pro stažení dat stačí přidat parametr {\tt --download} .

Příklad:

{\tt \$ iprdownloader.py ---alike 'Technická mapa' ---crs 4326 ---format gmp ---outdir IPR}

Pro možnost stážené datov souboury rovnou importovat do databáze PostGIS je 
minimálně nutné zadat název databáze {\tt ---dbname }. To v případě pokud je
databáze PostGIS umístěna v lokální líti. Pokud je databáze umístěna na jiném
serveru je nutné zadat adresu {\tt ---dbhost} a port {\tt ---dbport} . Pokud je
ještě vyžadován autorizovaný přístup použije sepřístupové jméno {\tt ---dbuser} a
heslo {\tt ---dbpassword} . 
Pokud je vložen název databáze není již potřeba používat parametr {\tt ---download} .

Příklad:

{\tt \$ iprdownloader.py ---alike 'Technická mapa' ---dbname pgis\_osm\_jakl ---dbhost geo102.fsv.cvut.cz  ---dbport 5432 } 
 

