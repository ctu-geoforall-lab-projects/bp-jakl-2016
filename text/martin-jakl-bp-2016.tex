%%%%%%%%%%%%%%%%%%%%%%%%%%%%%%%%%%%%%%%%%%%%%%%%%%%%%%%%%%%%%%%%%%%%%%%%%%%%%%%%%%%%%%%%%%%%%%%%%%%%%%%%%%
%%													%%
%% 	BAKALÁŘSKÁ PRÁCE - Začleňování geografických datových sad do OpenStreetMap			%%
%% 				 Martin Jákl								%%
%%													%%
%% pro formátování využita šablona: http://geo3.fsv.cvut.cz/kurzy/mod/resource/view.php?id=775 		%%
%%													%%
%%%%%%%%%%%%%%%%%%%%%%%%%%%%%%%%%%%%%%%%%%%%%%%%%%%%%%%%%%%%%%%%%%%%%%%%%%%%%%%%%%%%%%%%%%%%%%%%%%%%%%%%%% 

\documentclass[%
  12pt,         			% Velikost základního písma je 12 bodů
  a4paper,      			% Formát papíru je A4
  oneside,       			% jednostranný tisk
  pdftex    % překlad bude proveden programem 'pdftex' do PDF
]{report}       			% Dokument třídy 'zpráva'

\linespread{1.3}


\usepackage[utf8x]{inputenc}    % Kódování zdrojových souborů je UTF8x
%\PrerenderUnicode{č,ř,ž,š,ě}    % přidání vyjímky pro znaky s diakritikou
\usepackage[czech]{babel}	% použití češtiny, angličtiny


\usepackage[square,sort,comma,numbers]{natbib}

\usepackage{caption}
\usepackage{subcaption}
\captionsetup{font=small}
\usepackage{enumitem} 
\setlist{leftmargin=*} % bez odsazení

\makeatletter
\setlength{\@fptop}{0pt}
\setlength{\@fpbot}{0pt plus 1fil}
\makeatletter

\usepackage{graphicx} 
\usepackage{color}
\definecolor{purple}{rgb}{0.4,0.0,0.4}

\usepackage{transparent}
\usepackage{wrapfig}
\usepackage{float} 

\usepackage{cmap}           
\usepackage[T1]{fontenc}    

\usepackage{textcomp}
\usepackage[compact]{titlesec}
\usepackage{amsmath}
\addtolength{\jot}{1em} 

\usepackage{chngcntr}
\counterwithout{footnote}{chapter}

\usepackage{acronym}
\usepackage{listings} % nastavení barev pro XML
\lstset{language=XML,
  breaklines=true,
  literate={ý}{{\'y}}1,
  stringstyle=\color{blue},
  identifierstyle=\color{black},
  keywordstyle=\color{purple}\bf,
  morekeywords={osm,way,nd,tag,head,many}}

\usepackage[
    unicode,                
    breaklinks=true,        
    hypertexnames=false,
    colorlinks=true, % true for print version
    citecolor=black,
    filecolor=black,
    linkcolor=black,
    urlcolor=black
]{hyperref}         

\usepackage{url}
\usepackage{fancyhdr}

\usepackage[
  cvutstyle,          
  bachelor           
]{thesiscvut}


\newif\ifweb
\ifx\ifHtml\undefined % Mimo HTML.
    \webfalse
\else % V HTML.
    \webtrue
\fi 

\renewcommand{\figurename}{Obrázek}
\def\figurename{Obrázek}

\newcommand{\obrazek}[1]{(viz obr. \ref{#1}/s\pageref{#1})}

\pagestyle{empty} % vypne číslování stránek

%%%%%%%%%%%%%%%%%%%%%%%%%%%%%%%%%%%%%%%%%%%%%%%%%%%%%%%%%%%%%%%%%
%%%%%%%%%%% Definice informací o dokumentu  %%%%%%%%%%%%%%%%%%%%%
%%%%%%%%%%%%%%%%%%%%%%%%%%%%%%%%%%%%%%%%%%%%%%%%%%%%%%%%%%%%%%%%%

%% Název práce
\nazev{Začleňování geografických datových sad do~OpenStreetMap}{ Integration of geographic datasets into OpenStreetMap}

%% Jméno a příjmení autora
\autor{Martin}{Jákl}

%% Jméno a příjmení vedoucího práce včetně titulů
\garant{Ing. Martin Landa, Ph.D.}

%% Označení oboru studia
\oborstudia{Geodézie, kartografie a geoinformatika}{}

%% Označení ústavu
\ustav{Katedra geomatiky }{}

%% Rok obhajoby
\rok{2016}

%Mesic obhajoby
\mesic{červen}

%% Místo obhajoby
\misto{Praha}

%% Abstrakt
\abstrakt{
Tato bakalářská práce se zabývá problematikou datových importů
do~projektu OpenStreetMap. Nejprve vysvětluje problematiku
licencování dat OSM a nastiňuje myšlenku Opendata. Dále se zabývá
možností začlenění datových sad z~IPR Praha. Rozebírá nekompatibilitu
licencí ODbL a CC BY-SA. V~praktické části se zabývá vytvořením
programu pro~stahování dat z~IPR Praha a importem do~PostGIS databáze.
Následně se zabývá rešerší a vlastním návrhem vhodných dat
pro~začlenění do projektu OpenStreetMap.
}%
{This bachelor thesis deals with problems of data imports
OpenStreetMap. First explains issues of licensing OSM data and
outlines the idea of OpenData. It also deals with the~possibility to
integrate data sets from IPR Prague. It analyzes the~incompatibility
of licenses ODbL and CC BY-SA. The~practical part deals with
the~creation of a program for downloading data from IPR Prague and
importing them into PostGIS database. Subsequently deals with the
researches and own draft of appropriate data for inclusion in
OpenStreetMap.
}

%% Klíčová slova
\klicovaslova
{OpenStreetMap, import, IPR, Python, Opendata}%
{OpenStreetMap, import, IPR, Python, Opendata}

%%%%%%%%%%%%%%%%%%%%%%%%%%%%%%%%%%%%%%%%%%%%%%%%%%%%%%%%%%%%%%%%%%%%%%%%

%%%%%%%%%%%%%%%%%%%%%%%%%%%%%%%%%%%%%%%%%%%%%%%%%%%%%%%%%%%%%%%%%%%%%%%%
%% Nastavení polí ve Vlastnostech dokumentu PDF
%%%%%%%%%%%%%%%%%%%%%%%%%%%%%%%%%%%%%%%%%%%%%%%%%%%%%%%%%%%%%%%%%%%%%%%%
\nastavenipdf
%%%%%%%%%%%%%%%%%%%%%%%%%%%%%%%%%%%%%%%%%%%%%%%%%%%%%%%%%%%%%%%%%%%%%%%

%%% Začátek dokumentu
\begin{document}

\catcode`\-=12  % pro vypnuti aktivniho znaku '-' pouzivaneho napr. v \cline 

% aktivace záhlaví
\zahlavi

% předefinování vzhledu záhlaví
\renewcommand{\chaptermark}[1]{%
	\markboth{\MakeUppercase
	{%
	\thechapter.%
	\ #1}}{}}

% Vysázení přebalu práce
%\vytvorobalku

% Vysázení titulní stránky práce
\vytvortitulku

% Vysázení listu zadani
\stranka{}%
	{\sffamily\Huge\centering\ ZDE VLOŽIT LIST ZADÁNÍ}%
%	{\sffamily\centering Z~důvodu správného číslování stránek}

% Vysázení stránky s abstraktem
\vytvorabstrakt

% Vysázení prohlaseni o samostatnosti
\vytvorprohlaseni

% Vysázení poděkování
\stranka{%nahore
       }{%uprostred
       }{%dole
       \sffamily
	\begin{flushleft}
		\large
		\MakeUppercase{Poděkování}
	\end{flushleft}
	\vspace{1em}
		%\noindent
	\par\hspace{2ex}
	{Chtěl bych poděkovat vedoucímu mé bakalářské práce,
 \\ Ing. Martinu Landovi,~Ph.D., za~odborné rady a~pomoc při zpracování této práce.
    Dále bych chtěl poděkovat své rodině za projevenou podporu a~trpělivost.}
}

% Vysázení obsahu
\obsah

% Vysázení seznamu obrázků
\seznamobrazku

% Vysázení seznamu tabulek
%\seznamtabulek

% jednotlivé kapitoly
% \setcounter{page}{1}  % nastaví čítač stránek znovu od jedné
\chapter{Úvod}
\label{1-uvod}

K projektu OSM jsem se dostal již před lety a zaujala mě možnost pomáhat tvořit 
a upravovat mapu. Možnost přidávat a opravovat objekty ze svého okolí a následně 
vidět výsledek a možnost ho využívat. Postupem času jsem zjistil, co je vhodné a 
co již nevhodné vkládat do mapy. Když mi bylo na škole nabídnuto dělat 
bakalářkou práci na téma, které se dotýká problematiky datových importů OSM, 
neváhal jsem. Nedávno totiž byla zveřejněna data z IPR a vznikla možnost ujmout 
se začlenění některých těchto dat do projektu OSM. 

Obsahem této práce je čtenáře nejprve seznámit s projektem OpenStreetMap. 
Přiblížit mu jeho vývoj a účel jeho vzniku. Jeho možnosti užití a licenci, 
jakožto hlavní místo, kam se bude soustředit konečný výsledek této práce.

Dále přiblížit problematiku datových importů do tohoto projektu. Uvést některé 
problémy, které se v minulosti u již proběhlých importech objevily a jak se 
řešili. Rozebrat možné další problémy, které mohou vzniknout při importaci dat 
z~IPR a navrhnout jejich možná řešení. 

Návrh skriptu v programovacím jazyku Python, který by tato data stahoval z~IPR 
a následně importoval do pracovní databáze postGIS. Popis použitých knihoven a 
tříd funkcí v~knihovně GDAL.

Analizovat data zveřejněná IPR a porovnat je s daty, která již jsou v OSM. 
Navrhnout, která data by byla vhodná importovat. Konzultovat tento záměr 
s~komunitou OSM a poradit se s~nimi v~případě některých otázkek. Nakonec 
vybraná data připravit k importu z~databáze PostGIS do OSM metodou „SQL dump“.

\chapter{Teorie}
\label{2-Teorie}

\section{OpenStreetMap}
\label{OpenStreetMap}

\subsection{Vznik}
\label{vznik}
OpenStreetMap (OSM) je projekt, jež vznikl s cílem vytvoření a sběru 
volně dostupných geografických dat a následně jejich možné vizualizace
do topografických map. Projekt založil Steve Coast v červenci roku 
2004 v Anglii. Jako inspirace mu posloužil projekt Wikipedia.

Zprvu projekt využívalo jen pár nadšenců, postupem času ale získal
projekt popularitu. S nárůstem počtu uživatelů se zvyšoval i objem dat.
Bylo tedy nutné navýšit kapacitu a zabezpečení serverů.
Také bylo potřeba zlepšit síťové řešení a infrastrukturu.
Tím je myšleno, že data byla rozdělena na více serverů.

V dubnu roku 2006 byla založena nadace OpenStreetMap Foundantion pro financováni 
projektu jako takového (zaměstnanců, běhu a zajištění bezpečnosti serverů atd.). \cite{wikiOSM}

\subsection{Struktura dat}
\label{struktura dat}
OSM data jsou nyní uložena v databázi PostgreSQL. \cite{OSMserver}

Pro samotnou výměnu dat ale slouží souborový formát {.osm}, který využívá datový
formát  XML. 
Jeho výhoda je jasná struktura, snadná orientace v kódu pro člověka. 
Nevýhodou je ovšem větší objem dat, který lze ale snížit kompresí. 

Prvky (elementy) jsou v OSM rozděleny na:
\begin{itemize}
    \item uzel (node) 
    \item cesta (way)
        \subitem neuzavřená
        \subitem uzavřená
        \subitem plocha (area)
    \item relace (relation) 
\end{itemize}

\subsubsection{uzel (node) }

Základním bodovým elementem je uzel (node), který je definován jedinečným
identifikátorem ({\tt id=}). 
Je ukládána verze uzlu, časový údaj, kdy byl do databáze přidán
a je také uvedeno v jaké změně to bylo provedeno ({\tt changeset=~}).
Každý uzel má svoje souřadnice uloženy v souřadnicovém systému WGS~84 (EPGS 4326). 
Dále je možné připojit různé atributy (tag) tj. klíče a hodnotou ({\tt tag~k=~v=~}).
\\*
Příklad XML souboru pro uzel (strom):

{\scriptsize \begin{lstlisting}
<node id="2905214905" visible="true" version="1" changeset="22804106" 
      timestamp="2014-06-08T06:57:20Z" user="Salamandr" 
      uid="1708065" lat="50.1036981" lon="14.3897278">
    <tag k="natural" v="tree"/>
    <tag k="source" v="bing:ortofoto"/>
</node>
\end{lstlisting} }

\subsubsection{cesta (way) }

Další liniový element je cesta (way), ta je vždy určena dvěma nebo více uzly.
Každá cesta má také svůj jedinečný identifikátor ({\tt way~id=~}) v rámci všech cest.
Má také základní atributy ({\tt id=*, visible=*, version=*, .. }), 
stejně jako u elementu uzel.
Dále však musí být seznam bodů, ze kterých se skládá.
Je zde uvedena pouze reference na id bodů, nikoliv jejich souřadnice.
Cestám lze taktéž přiřadit různé atributy.

Cesty se dále rozdělují na neuzavřené a uzavřené.
Jen uzavřené cesty mohou tvořit plošné objekty (area),
a jen těm se mohou přiřadit atributy určené pro plochy.
Například les ({\tt landuse=forest}).

Příklad XML souboru pro cestu:

{\scriptsize
\begin{lstlisting}
<way id="87249754" visible="true" version="2" changeset="34489106"
     timestamp="2015-10-07T11:52:41Z" user="Petr Dlouhý" uid="17615">
    <nd ref="1014526199"/>
    <nd ref="1014525941"/>
    <nd ref="1014526337"/>
    <nd ref="1014526022"/>
    <nd ref="1014526277"/>
    <nd ref="1014525984"/>
    <tag k="highway" v="path"/>
    <tag k="source" v="bing:ortofoto"/>
</way>
\end{lstlisting}
}
      
\subsubsection{relace (relation) }

Speciálním elementem je relace (relation), do které lze zahrnout
jeden a více elementů. Mohou se do nich spojit elementy stejného druhu (uzel a uzel),
nebo různého (uzly, cesty nebo i relace).
Například dálnice je tvořena několika cestami (way), a ty
jsou zahrnuty do společné relace Dálnice D1. Nebo například turistická trasa
(od~KČT) může být relace sdružující jak liniové elementy (cesty, pěšiny), tak i bodové
(rozcestníky, vyhlídky, apod. ).

U atributů popsaných na OpenStreetMap Wikipedii \cite{OSMfeatures} je vždy uvedeno,
k jakému druhu elementu je
vhodné či nevhodné je použít (dle komunity uživatelů OSM). Pokud se stane, že je použit
na~jiný, než doporučený druh elementu, může být následně problém v některých vykreslovačích. 
Viz \ref{Vykreslovače}

\subsection{Změna licence}
\label{změna licence}

Původně byla data OSM a generované grafické dlaždice distribuována pod licencí
Creative~Commons~AllributionShare~Alike 2.0 (CC~BY~SA~2.0).

Tato licence umožňovala užití (distribuci ale i editaci) díla (dat) pod podmínkou,
že bude uveden zdroj OpenStreetMap.org ve viditelné části
vytvořených mapových dlaždic \cite{OSMlicence}.

  \begin{figure}[hbt]
    \centering
      \includegraphics[scale=0.75]{./pictures/attribution_example.png}
      \caption{Příklad umístění licence,
                zdroj \url{http://www.openstreetmap.org/copyright}}
      \label{fig:attribution_example}
  \end{figure} 

V roce 2012 byla licence publikovaných dat změněna na Open Data Commons
Open Database Licence (ODbL).
Důvodem byla lepší licenční ochrana dat v databázi. 
Starší licence (CC BY-SA) chránila data pouze pomocí autorského zákona. 
Sice obě licence jsou "Uveďte autora" a "Zachovejte licenci", avšak rozdíl mezi
"starou" a "novou" je ten, že pokud někdo pod "starou" licencí (CC BY-SA),
vytvoří mapu (nebo jiné dílo) musí zachovat stejnou licenci.
Nemusí však uvolnit přímo data, která při tom použil.

Data OSM pod novou licencí, je možné jakkoli upravit,
a zveřejnit pod jakoukoli novou licencí,
která je podobná s myšlenkou Opendata, 
za podmínky, že také uvolní všechny své doplňky a vylepšení.\cite{OSMlicenceChange}.

Tato změna licence přinesla problém
s~daty, které byly poskytnuty projektu za předchozí licence
(CC~BY~SA~2.0). Bylo nutné se dotázat každého z dřívějších
přispěvatelů dat, ať už právnických osob, tak i fyzických osob,
zdali souhlasí, že jejich data budou distribuována pod novou licencí.
U přispěvatelů, kteří k tomuto nesvolili, nebo ti co se nevyjádřili,
bylo nutné jejich příspěvky z~databáze OSM vymazat.
Tato situace nastala pouze ve zlomku případů (méně než 1 \%).
Nejvíce tato změna licencí ohrozila data v zemích jako je Polsko a Nový Zéland. \cite {OSMlicenceIssue}

V současnosti jsou tedy data OSM distibuována pod licencí ODbL a
generované grafické dlaždice pod licencí CC-BY SA 4.0. \cite{OSMlicence}

\subsection{Licence ODbL}
Licence ODbl \footnote{Plné znění (anglicky) viz na \url{http://opendatacommons.org/licenses/odbl/1.0/}}
se zabývá právní ochranou databází, včetně
veškerých autorských a příbuzných práv k databázi.

Licence ODbl umožňuje
\begin{itemize}
    \item    kopírovat, distribuovat a užívat data
    \item    vytvářet nová data z původních
    \item    měnit původní data
\end{itemize}

Při použití databáze s licencí ODbl, je k ni
nutné připojit kopii přesného zněnít licence ODbl.
Při nakládnání s databází ponechat nedotčena veškerá autorská práva a
uvést její zdroj (nebo autora).
Veškeré odvozené databáze musí splňovat podmínky ODbl,
tj. buď přímo licence ODbl, nebo licence s ní kompatibilní.
\cite{Nesetril2013thesis}

\subsection{Zdroje dat}
\label{Zdroje dat}
Jak bylo zmíněno, byla snaha, aby mapová data tvořili jedinci prvotním
sběrem dat, tj. měřením vlastními GNSS (GPS, Glonass) přijímači a
znalostí místních poměrů (uzavřené silnice, stezky atd.).  A takto
vzniklé dílo volně užívat k vlastním potřebám. Komunita přispěvatelů se
zprvu pomalu, později poměrně rychle rozrostla a dnes čítá 3,7 milionů
registrovaných uživatelů s alespoň jednou vytvořenou změnou v OSM a
2,7 milionů účtů aktivních přispěvatelů.\cite{OSMstats}

Takto vzniklé mapové podklady byly velice vhodné i pro další projekt, dnes již
velice rozšířený a známý jako Geocashing (GC). Projekt GC začal mapové
podklady od OSM užívat a zároveň jeho uživatelé začali sami tvořit a
přispívat do OSM. 

Přispěvatelé dat do OSM musí respektovat licenci projektu OdbL.
Tudíž i jejich zdroj dat musí splňovat tuto licenci. Proto by měli
všechny svoje změny, které v OSM vytvoří, řádně ozdrojovat atributem
s~klíčem 
{\tt source}.
V případě vlastního sběru dat se vyplňuje hodnotou
{\tt source=survey} ,
popřípadě uvést zdroj, odkud čerpali. Pokud tuto povinnost poruší a
použijí zdroj, jež není kompatibilní s licenční politikou OSM, tak ostatní 
přispěvatelé do~OSM tuto změnu, aby předešeli sporům, odstraní. 
V tomto případě dojde k odstranění celé sady změn, byť by v~ní byl pouze jeden prvek, jenž to poruší.

Druhým významným zdrojem dat jsou soukromé subjekty (společnosti).
Většinou jde o podkladové zdroje dat, například ortofoto. Pro obkreslování
silničních síti z~leteckých nebo satelitních snímků. V~jejich případě to
bylo řešeno písemným svolením, nebo smlouvou. Jako významným zdrojem
podkladových map pro obkreslování byly mapy vyhledávače Bing od společnosti
Microsoft. Ta svolila použít jejich letecké snímky většiny
obydlené pevniny. \footnote{\url{http://wiki.openstreetmap.org/wiki/Bing}}

Třetím zdrojem dat a zároveň postupně dominujícím, co do jeho obsahu, jsou
datové sady ze státního sektoru. Tyto datové sady jsou nejvhodnějším zdrojem
dat.

\subsection{Importy}
\label{Importy}
Výraz import v tomto případě znamená začlenění většího množství dat z datové sady nebo datových sad
z~jednoho datového úložiště (databáze serveru) na jiný. Při velkých objemech dat
se využívá výkonu výpočetní techniky z důvodu její bezchybovosti, a také
z~důvodu časové náročnosti. Na člověku poté zbývá zvolit postup importu
a naprogramovat skript nebo program, podle něhož technika samotný proces
provede. 

Datové importy z veřejných databází do databáze OSM jsou velmi cenné. 
Otevřené geografické, ale i jiné, databáze státu a jeho veřejných institucí 
financovaných státem jsou komplexní. Komplexní v tom smyslu, že obsahují celistvý
soubor dat, protože je daná instituce vyžaduje ke svém chodu. Jistá nevýhoda tu 
ale může být, data nemusí být vždy úplně aktuální. Některá data mohou 
být sbírána a zveřejněna i s větším časovým odstupem.

V rámci České Republiky proběhlo již několik hromadných datových importů. Jak 
již bylo řečeno, více času zabere samotná příprava na import.
V~případě importu dat do OSM, nejen napsání skriptu, ale i nutná diskuze tohoto záměru
na~diskuzní konferenci Talk-cz. 

\subsection{Talk-cz}
\label{Talk-cz}
Tato diskuze probíhá přes posílání emailových zpráv do společné konference. 
Uživatelům chodí emaily z probíhající diskuze a pokud na nějaký chtějí reagovat,
tak pošlou email na adresu serveru, na kterém diskuze běží, a musí do Předmětu 
napsat Re: a předmět zprávy, na kterou reagují. Server tyto zprávy pomocí 
předmětu a času řadí. Diskuze je poté k dispozici na webových stránkách.\footnote{\url{http://lists.openstreetmap.org/listinfo/talk-cz}}

\subsection{Vykreslovače}
\label{Vykreslovače}
Na hlavní stránce OSM (\url{http://www.openstreetmap.org}) je k dispozici mapová aplikace. Ta nabízí pět
„základních“ přednastavených vrstev vykreslených z~dat z OSM.
Pro snadné vykreslení dat do grafických dlaždic se používá takzvaný pseudo Mercatorovo
zobrazení nebo Web Mercator (EPSG 3857), a to proto, že snadno vykreslí celý povrch Země (až na velké zkreslení u polů). 
První, kdo zobrazení Web Mercartor použil byla spolčnost Google pro svoje mapy Google Maps.\cite{WebMercator}
%%% ML: mozna dodat screenshoty do prilohy?

\begin{itemize}
  \item Standardní vrstva - vykresluje všechny prvky přiměřeně.
  \item Cyklomapa - vykresluje cyklostezky, výškopis. 
  \item Dopravní mapa - vykresluje silniční a železniční sítě.
  \item MapQuest Open - podkladová mapa právě pro potřeby Geocaching.
  \item Humanitární mapa, která vykresluje důležité věřejné služby a potlačuje ostatní prvky. 
\end{itemize}

Existují další stránky, jež kombinují data z OSM a jiných zdrojů a vytváří z nich tematické mapy.
Například mapu turistických a cyklistických tras vykresluje
pro celou Evropu mtbmap.\footnote{mtbmap.cz}

%%% priklad? mozna dodat hezky screenshot?
Zajímavými projekty jsou například ty, které k 2D mapě přidávájí „třetí“ rozměr a
vytvářejí tzv. 2.5D mapu. Většinou jde o 3D zobrazení budov, mostů (dle
atributů), popřípadě i stromů.

%%% http://demo.f4map.com/#lat=50.1036063&lon=14.3898880&zoom=19


\section{Opendata}
\label{opendata}

Základní myšlenka otevřených dat vznikla v USA.
Ta říká, pokud vzniknou jakákoli data z veřejných peněz, měla by tedy být
veřejně přístupná. Některé studie uvádí, že tento jev měl kladný efekt na ekonomiku.
\footnote{\url{http://www.worldbank.org/content/dam/Worldbank/document/Open-Data-for-Economic-Growth.pdf}}
\footnote{\url{http://www.otevrenadata.cz/res/data/001/003498.pdf}}

Proto jako první hlavní zdroj byly satelitní snímky povrchu Země (Landsat) %\footnote{\url{http://landsat.usgs.gov/}}
a digitální model terénu (projekt SRTM) %\footnote{\url{http://www2.jpl.nasa.gov/srtm/}}
s~rozlišením 30x30 m od NASA (pro pozdější vykreslení vrstevnic).

Tento trend se začal rozšiřovat zprvu do zemí západní Evropy
(Velké Británie, Francie či Německa), ale i země mimo Evropu.\cite{OpendataTrends}

%%% Jednim z projektu fondu jsou prave
%%% Otevrena data
V České Republice se tomuto věnuje fond Otakara Motejla, založen na jeho počest,
a jeho hlavním projektem je Otevřená data. Spolupracuje s nadací Society Fund
Praha a Rekonstrukce Státu.
V rámci těchto uskupení je vyvíjen tlak na transparentnost veřejné správy,
zveřejňování smluv a dat státních institucí, jelikož jejich získání a údržba byla
placena z~veřejných zdrojů.
%%% rozsirit
\\
\\*
Otevřená data lze dělit na 5 stupňů,
dle otevřenosti.

\begin{itemize}

    \item   Data jsou dostupná v WWW síti s jasnými licenčními podmínkami.
            Není zde žádný požadavek na datová formát a proto není brán jako
            dostatečný způsob zveřejňování dat. Těmito daty ku příkladu jsou
            Webové Mapové Služby (WMS). Jsou veřejně k dispozici avšak
            nezveřejňují přímo data ale jen obrázky generované z nich.

    \item   Zveřejněná data jsou již ve strojově čitelném formátu, který je
            veřejně dobře znám. Nejčastěji se jedná o charakter tabulky.
            Musí umožňovat přístup k jednotlivým řádku a obsahu buněk.

    \item   Třetí stupěn navíd od druhého vyžaduje aby k jeho čtení nebyl
            vyžadován speciální software ale aby byl volně dostupný ze sítě WWW.
            Spadají sem tedy dokumenty formátů OpenOffice (Office Open XML,
            OpenDocument). Pro prostorová data formáty GML, KML nebo GeoPackage.

    \item   U stupně otevřenosti 4 je v distribuované datové sadě
            povidnost zavést identifikaci entity ve tvaru Internationalized
            Resource Identifier (IRI)

    \item   U nejvyššího možného stupně otevřenosti je vyžadováno aby distribuce
            splňovala standardy popojených dat (Linked Data).

\end{itemize}
%% http://opendata.gov.cz/standardy:stupne-otevrenosti

\begin{figure}[hbt]%[p]
    \centering
    \includegraphics[width=0.8\textwidth]{./pictures/5star-steps.png}
    \caption{Stupně otevřenosti dat, převzdato z \url{http://5stardata.info}}
    \label{fig:Stupně otevřenosti dat}
\end{figure}

Díky této iniciativě došlo v rámci České Republiky ke zveřejnění
dat státních úřadů s různým úspěchem a stupněm otevřenosti.
Avšak tento trend zpřístupnění a zveřejňování pozvolna pokračuje.
Příkladem buď Ministerstvo vnitra, Ministerstvo financí nebo Český úřad zeměměřický a katastrální.


\section{IPR Praha}
\label{IPR Praha}
IPR Praha je zkratka názvu Institut plánování a rozvoje a hlavního města Prahy. 
Tento institut se věnuje urbanismu, architektury a rozvoje města Prahy. Hlavním
úkolem IPR je tvorba územního plánu Prahy a významným úkolem IPR je zajišťovat
zpracování geografických informací. Spravuje data a mapy města Prahy. Od roku 
2002 poskytuje na svých stránkách zdarma webové aplikace bez limitu využití. 
Po~rozvoji Pražského geoportálu došlo k jejich většímu využívání. Na základě 
platných Pravidel pro poskytování dat a výstupů z datových souborů a datového 
skladu Geografického byla ode dne 1.~4.~2015 zveřejněny datové soubory a další 
webové služby. Tato data byla uveřejněna pod licencí CC-BY SA 4.0 \cite{IPR}
\subsection{licence CC BY-SA 4.0}
\label{licence CC BY-SA 4.0}
Licence CC BY-SA 4.0 je zde uvedena ve zjednodušeném znění.

"Uživatel s smí
\begin{itemize}
    \item   Sdílet - rozmnožovat a distribuovat materiál prostřednictvím jakéhokoli média v jakémkoli formátu
    \item   Upravovat - remixovat, změnit a vyjít z původního díla
\end{itemize}
pro jakýkoliv účel, a to i komerční.

Poskytovatel licence nemůže odvolat tato oprávnění do té doby, dokud dodržujete licenční podmínky.

Uveďte původ — Je Vaší povinností uvést autorství, poskytnout s dílem odkaz
na~licenci a vyznačit Vámi provedené změny. Toho můžete docílit jakýmkoli
rozumným způsobem, nicméně nikdy ne způsobem naznačujícím, že by poskytovatel
licence schvaloval nebo podporoval Vás nebo Váš způsob užití díla.

Zachovejte licenci — Pokud budete toto dílo upravovat, pozměňovat nebo
na~něj navazovat, musíte svoje odvozená díla vystavovat pod stejnou
licencí jako původní dílo."

Podrobněji viz přesné znění na stránkách CraetiveCommmons.
\footnote{https://creativecommons.org/licenses/by-sa/4.0/legalcode/}

\chapter{Použité technologie}
\label{3-technologie}

\section{PostgreSQL}
\label{PostgreSQL}

PostgreSQL je open source objektově-relační databázový systém. Za dlouhou dobu vývoje se stal robustním a hojně využívaným řešením hlavně díky své spolehlivosti. Lze s ním pracovat ve všech známých operačních systémech. 
\cite{PostgreSQL}

\subsection{PostGIS}
\label{PostGIS}
PostGIS je nadstavba databázového systému PostgreSQL.
Rozšiřuje jej o funkce a~datové typy, které umožňují uložení, manipulaci a správu geografických objektů. Mezi podporované geometrické typy patří bod (point), linie (linestring),
nebo polygon (polygon). Dále je možno prvky sdružit 
do tzv. multiprvků (multiPoint, multiLineString nebo multiPolygon). Geometrie prvku určuje
jeho polohu v daném souřadnicovém systému. Data je tedy možno transformovat do jiných PostGISem podporovaných souřadnicových systémů. Systém kromě základních operací jako výpočet délky, obvodu, plochy, podporuje i sofistikovanější funkce jako je např. výpočet obalové zóny a další.

Sotware PostGIS je šířen pod GNU General Public License (GPLv2).

\section{Python}
\label{Python}
Programovací jazyk Python vznikl v roce 1991. Navrhl ho Guido van
Rossum. Je vyvíjen jako open source projekt. Inspiroval se hlavně
programovacím jazykem ABS, který byl přímo vytvořen pro~výuku 
začátečníků v programování. Python je proto jeden z nejvýhodnějších
programovacích jazyků pro začínající programátory, ale i přesto ho lze
použít pro~praktické programování.

Jedná se o jednoduchý programovací jazyk podporující různá
programovací paradigmata, hlavně objektové, imperativní, procedurální
a v~omezené míře i funkcionální. Je multiplatformní, lze jej tedy
použít na~různých operačních systémech. Klade velký důraz na~syntaxi 
psaného kódu, využívá hlavně prázdné znaky v~psaném kódu.

Programy nebo skripty lze psát libovolném textovém editoru. Vhodné je
použít standardu PEP~8, který definuje správnou strukturu odsazení kódu.

V současné době je vydána stabilní verze 3.5.7, která oproti předchozí
verzi 2.x mění některé již \uv{zažité} vlastnosti jazyka. Například 
nejviditelnější je změna funkce {\tt print()}. Ve verzi 2.x stačilo napsat
 {\tt print 'Ahoj'}. Ve verzi 3.x je již nutné řetězec vložit
do~závorek  {\tt print('Ahoj')}. Tyto dvě dnes používané verze 2.x
a 3.x jsou navzájem nekompatibilní.
\cite{python} 
\cite{wikiPython} 
  
\subsection{Knihovna argparse}
\label{argparse} 
Knihovna argparse je knihovna 
programovacího jazyka Python, která řeší konzolový vstup do aplikace a 
zpracovává ho pro~další použití v programu.\cite{argparse}

\subsection{Knihovna xmltodict}
\label{xmltodict} 
Tato Python knihovna umí číst soubory ve formátu XML a zpracovat jejich obsah do datové struktury, která je jednodušší pro manipulaci v samotném
programu. Data z~XML zpracuje do~slovníku (dictionary), kde klíčové slovo (key) reprezentuje hodnota(value) ve slovníku.
Hodnotami mohou být dále i řady dalších klíčových slov (key).\cite{xmltodict}

Příklad zpracování dat ve formátu XML.

{\scriptsize
\begin{lstlisting}
<head>
  <many>ele1</many>
  <many>ele2</many>
</head>
\end{lstlisting}
}

Zpracování dat provedeme v Pythonu příkazem:

{\scriptsize
\lstset{language=Python}
\begin{lstlisting}
with open('example.xml') as fd: 
  doc = xmltodict.parse(fd.read()) 
\end{lstlisting}
}

Objekt {\tt doc} vypadá následovně:

{\scriptsize
\lstset{language=Python}
\begin{lstlisting}
OrderedDict([(u'head', OrderedDict([(u'many', [u'ele1', u'ele2'])]))]) }
\end{lstlisting}
}

Data v této struktuře lze již jednoduše procházet.

{\scriptsize
\lstset{language=Python}
\begin{lstlisting}
doc['head']['many'][0]
ele1
\end{lstlisting}
}

\subsection{Knihovna GDAL}
\label{GDAL}
Jedná se o knihovnu pro práci s geografickými daty umožňující jejich
čtení a~zápis. Je napsána v programovacím jazyce C++. Je považována
za~jeden z hlavních open~source projektů hojně využívaných ve~sféře
GIS. Podporuje velkou škálu rastrových a vektorových formátů.
Díky knihovně PROJ.4 podporuje celou škálu souřadnicových systémů a umožňuje transformaci dat mezi nimi. \cite{GDAL}

\section{QGIS}
\label{QGIS}
QGIS je program vytvořený za~účelem prohlížení a editace geografických
dat. Jedná se o open source software. Je tedy možné, aby na~jeho vývoji
spolupracoval kdokoli. Vlastní uživatelské nástroje se píší ve formě tzv. zásuvných modulů, které je možné implementovat v jazyce C++ nebo Python.
Tyto moduly si ostatní uživatelé mohou poté stáhnout a začlenit do své instalace QGISu. QGIS také umožňuje využívat jazyk Python v rámci vestavěné konzole. Lze tak vytvářet vlastní jednoduché skripty v jazyku Python a spouštět
je přímo v~QGISu.

Od svého vzniku byla vydána již spousta verzí tohoto programu. Dříve byly
verze pojmenovávány podle měsíců planet Jupitera a Saturnu, později se začaly používat jména měst. V současnosti (duben 2016) je
k~dispozici verze 2.14 Essen.

\chapter{Praktická část}
\label{4-Praktická část}

\section{Geodatabáze PostGIS}
\label{Server PostGIS} 
V rámci této bakalářské práce byla zřízena geodatabáze PostGIS na školním
serveru {\em geo102}. Data z OSM jsou do školní databáze nahrávána pomocí programu {\tt osm2pgsql}.\footnote{dostupné z \url{http://wiki.openstreetmap.org/wiki/Osm2pgsql}}
%%% ML: to neni pravda, shellovy skript (viz git) pracuje primo s daty CR
%Ten nejprve stáhne aktuální data pro celou Evropu a následně je ořeže
%jen na území ČR.
%%% ML: neskodilo by zminit, ze tyto aktualnize zajistuje shellovy skript, ktyery je soucasti prilohy na cd
Obsah databáze je každý den pravidelně aktualizován podle aktuálního stavu OSM.

%%% ML: my jsme to zakladni schema ale upravovali (height)
Tabulky v databázi byly vytvořeny podle základního schématu. Pro~každou třídu prvků (uzel, cesta
a relace) je vytvořena samostatná tabulka. Pro~uzly je vytvořena
tabulka {\tt czech\_point}, pro~cesty tabulka {\tt czech\_line} a pro~plošné prvky (uzavřené cesty s atributy
pro~plochy) tabulka {\tt czech\_polygon}. Relace jsou
%%% ML: plochami a ne plochamy! (to uz me prekvapilo, rikal jste, ze text projde jazykovou korekci)
řešeny tak, že pokud jsou tvořeny plochami, vytvoří z~nich program
{\tt osm2pgsql} multipolygon a ten vloží do~tabulky {\tt czech\_polygon}. Stejně jsou
řešeny i relace tvořené cestami (tabulka~{\tt czech\_line}), respektive body
({\tt czech\_point}). Ještě je dle schematu vytvořena tabulka
{\tt czech\_roads}, která obsahuje všechny silnice v~ČR.

Každý prvek má přiřazen, v~rámci tabulky, jedinečné číslo {\tt gid}.
Dále je vytvořena pro každý prvek geometrie a pro~každý
atribut (dle schématu) stejnojmenný sloupec s~hodnotou. Například
bod s~atributem {\tt landuse=forest} má ve~sloupci {\tt landuse}
hodnotu {\tt forest}.

Takto vytvořená databáze bude použita pro~analýzu dat IPR a~jejich přípravu pro importu do~OSM.

\section{Data IPR}
\label{IPR data}
%%% ML: zde by se hodila reference na kap., kde o tom pisete
IPR na svém webu od~jara roku 2015 zveřejňuje svá data v režimu otevřených dat. Vektorová data zveřejňuje ve formátech GeoJSON, DXF, GML a ESRI Shapefile. Dále je na~výběr mezi dvěma
%%% ML: jsou to srs ane kartograficka zobrazeni (jste studenten geodezie!)
souřadnicovými systémy S-JTSK (EPSG:5514) nebo WGS 84 (EPSG:4326).

Celkem IPR zveřejňuje (toho času) 96 datových setů, které jsou rozděleny
do~kategorií.\footnote{viz webové stránky IPR, \url{www.geoportalpraha.cz/cs/opendata}}

Pokud není řečeno jinak, jedná se o data ve vektorové formě.
Těmito kategoriemi jsou:

\begin{itemize}
    \item   3D model - Obsahuje 3D modely budov a mostů v Praze a
            digitální model povrchu (DMP), dále rastrově digitální
            model povrhu a terénu, absolutní a relativní výšky budov.

            Z těchto dat by se pro OSM dala využít poloha budov
            ze~souboru Budovy 3D. Budovy byly do OSM již importovány
            %%% ML: mate nekde zkratku vysvetlenu
            ze~zdroje RÚIAN, není třeba je tedy importovat znovu.

    \item   Digitální technická mapa Prahy - zde jsou k dispozici
            ve~vektorové podobě všechny inženýrské sítě v~Praze,
            technické budovy a parcelní hranice.

            Do OSM se, možná zatím, nepřidávají inženýrské sítě.
            V~současnosti lze ale do OSM přidávat zdroje veřejného osvětlení.
            %%% ML: nasledujici veta je uplne zbytecna, jsou takovato data v IPR?
            %Takže pokud by nějaký soubor dat obsahoval
            %značky veřejného osvětlení, mohla by se tato
            %data importovat do OSM.
            Hranice města Prahy a jeho městských
            částí již v~OSM jsou dostupná. Datovou sada Technické budovy by bylo
            možné použít jako zdroj pro budoucí aktualizaci OSM.
            %%% ML: no prubehy hranic parcel obsahuje RUIAN, neni treba nic obkreslovat, to se mozna delo drive, dnes jiz rozhodne ne...
            Průběhy hranic parcel lze již nyní bezproblémově
            obkreslovat z rastrových podkladů, které veřejně
            poskytuje ČÚZK. 

    \item   Doprava - obsahuje cyklistické trasy a značky, pěší trasy,
            parkovací zóny, automaty, P+R parkoviště a mapy zón PID.
            Cyklistické trasy v Praze lze do OSM importovat, jsou ale
            %%% ML: uzivateli (a ne uzivately)! vedouci prace neni jazykovy korektor ;-)
            již velmi dobře zmapované samotnými uživateli. Datová sada
            obsahuje bodové značky, z~těch by se některé mohly
            importovat. V~případě datové sady 
            Cyklistická doprava~-~značky, byla data prohledána. Dle IPR
            uváděné hodnoty {\tt 103} ve sloupci {\tt DRUH}
            pro~Bikesharing, bohužel žádné takové hodnoty
            neobsahovaly. Import dat z této datové sady nebyl proto
            dál brán v potaz. Bylo by možné do OSM přidat parkovací
            zóny (jako nový atribut u~stávajících komunikací).
            Vhodná data jsou parkovací automaty a také P+R parkoviště.

    \item   Geologie - obsahuje vektorové mapy radonového nebezpečí.
            Tato data se do~OSM nepřidávají.

    \item   Hluk - obsahuje hlukové mapy, a to ve~dne a v~noci. Oboje
            v~rastrové formě. Tato data se do OSM nepřidávají. Dále
            obsahuje i protihlukové bariéry, které by bylo možné
            do~OSM přidat.

    \item   Kvalita životního prostředí - obsahuje datovou sadu
            Oblasti svozu komunálního odpadu. Tato data není vhodné
            do~OSM začlenit. Dále ale obsahuje datovou sadu
            Odpadní Zařízení pro občany, která obsahuje informace
            o~všech sběrnách odpadu v~Praze. Tato datová sada se jeví
            jako další vhodná pro~import.

    \item   Mapové podklady - obsahuje klady mapových listů různých
            měřítek (1:500 až 1:10~000). Klady mapových listů nejsou
            vhodná data pro~OSM. Dále kategorie obsahuje i datové sady
            vrstevnic, a to po~5~m, 2~m a 1~m. Vrstevnice přímo OSM
            neobsahuje, avšak byly by velkým zpřesněním stávajícího
            zdroje vrstevnic. Nebo jako zdroj pro~projekty, které
            kombinují OSM s~jinými daty.\footnote{Projekt mtbmap.cz nebo \url{http://mapa.prahounakole.cz/}}

    \item   Občanská vybavenost obsahuje pouze datovou sadu Veřejné
            toalety. Jistě jsou již toalety v Praze částečně
            zmapovány, ale i tak by bylo vhodné je doplnit. Jsou tedy
            vybrána jako další vhodná data pro~import.

    \item   Ochrana přírody a krajiny, obsahuje data o ochranných
            pásmech památných stromů. Bohužel ne samotné stromy.
            Vhodná data pro import jsou z~datové sady Přírodní parky a
            Významný krajinný prvek - registrovaný. Z těchto datových
            sad by se mohly doplnit, nebo aktualizovat data v OSM.

    \item   Ortofoto obsahují letecké snímky (rastr) Prahy 
            s~rozlišením až 5~cm na~pixel jak ve~viditelném spektru
            světla, tak i v~infračerveném.

            Ortofoto je vhodné pouze jako podkladový zdroj,
            například pro mapovací aplikaci JOSM.

    \item   Ovzduší obsahuje vektorové mapy znečištění ovzduší a také
            zdroje znečištění. Tato data se v OSM nemapují, jediné co
            by se mohlo přidat do OSM jsou bodové zdroje, a to
            například značkou pro~komín. Dále jsou v~této kategorii i
            bonity, a to z~různých hledisek (ovzduší, půdy,
            osvit, atd.). Tato data nejsou vhodná pro import do~OSM.

    \item   Platný územní plán. V této kategorii jsou datové sady
            Veřejně prospěšných staveb (plošné, liniové a bodové).
            Datová sada VPS obsahuje také informace o~P+R u~stanic
            metra. Tyto informace pocházejí z~územního plánu, a tedy nemusí být
            realizovány. Po prozkoumání a porovnání s~datovou sadou
            Záchytná parkoviště P+R, lze konstatovat, že jsou všechny záznamy obsaženy již v této datové sadě.

    \item   Socioekonomická data obsahuje pouze vektorovou mapu ceny
            pozemků. Data o ceně pozemků se do~OSM nepřidávají. 

    \item   Technická infrastruktura - vodní hospodářství, obsahuje
            záplavová území Q20, Q50 a Q100 a území zaplavené
            povodněmi v roce 2013. Informace o záplavových územích se do~OSM
            nepřidávají. Dále jsou v této kategorii datové sady
            Protipovodňové ochrany, které obsahují údaje o všech dočasných
            protipovodňových zdech. Jelikož se jedná o~dočasné překážky, a to~ještě jen v~době povodně, nemá smysl je do OSM přidávat.

    \item   Urbanismus - Z~této kategorie by se mohly pro OSM využít
            informace z~datové sady
            Stavební uzávěry - dopravní. Bylo by ale nutné udržovat
            aktuálnost infomací o plánovaných dopravních uzavírkách.
            Dále by bylo ještě možné přidat do OSM počet pater budov
            z~datové sady Podlaživost.
\end{itemize}


\subsection{Licenční problém}
\label{Licenční problém}
Jak již bylo řečeno výše, IPR svá data zveřejnil a stále zveřejňuje
(v době tvorby této práce) pod~licencí CC~BY-SA~4.0, viz kapitola \ref{licence CC BY-SA 4.0}.

%%% ML: tak co, vete chybi druha polovina (duvod: tyto dve licence nejsou kompatibbilni)
Jelikož jsou ale data OSM distribuována pod licencí ODbL (v1.0), která
neumožňuje začletnit data z licencí CC BY-SA. 

Na některá data (informace), jako například místní názvy, ulice atd.
se nemohou vztahovat autorská práva. Neboť nevyžadují žádnou 
kreativitu pro jejich \uv{vytvoření}. Ovšem na soubor (databázi) těchto dat
vytvořený podle daných kritérií již může být právní
ochrana uplatněna. A právě proto vznikla licence ODbL, která se
zaměřuje na ochranu databáze jako celku.

U dat chráněných autorským právem a danou licencí se vyžaduje jejich
vzdání. V~některých zemích, kde se nelze úplně vzdát autorských
práv, se vyžaduje omezení autorských práv na~nejnižší možnou
míru. Autor souhlasí, že nebude vymáhat svoje autorská práva, po dobu,
kdy jsou data uložena v databázi. \cite{ODbl}

IPR byl kontaktován, zdali by mohla být jeho data použita do OSM.
Dle vyjádření vyplynulo, že IPR by neměl problém s použitím svých dat,
jelikož jsou zveřejněna pod myšlenkou Opendata. Byla navrhnuta
možnost udělit od~IPR pro~OSM licenční výjimku.

Vznikla by tu však situace, kdy by byla data IPR k dispozici 
pod~licencí CC BY-SA a v~OSM (sice modifikovaná) pod~licencí
ODbL. Tohoto dualismu by mohl využít někdo, komu by licence CC BY-SA
neumožňovala jeho záměr. Mohl by si totiž data obstarat z databáze OSM,
sice modifikovaná, ale již pod novou licencí (ODbL), která by mu jeho
záměr umožnila.

Hlavní rozdíl je totiž v tom, že pod licencí CC~BY-SA sice může data
různě měnit a upravovat, ale poté když je zveřejňuje, musí zachovat
licenci CC~BY-SA. Nemusí ale uvolnit (zveřejnit) data, která k nim 
přidal. V~případě licence ODbl se tato situace diametrálně liší.
Uživatel opět smí data upravovat nebo k~nim přidávat jiná data, ale
poté může svoje dílo distribuovat pod~jinou licencí, za~podmínek, že
uvolní veškerá data, která k tomu použil.

Vhodnější by tedy bylo, změnit licenci dat zveřejňovaných IPR
na~ODbL.

%%% ML: nepredbihal bych, je to spise spekulace
%Dle vyjádření se tato situace bude ze strany IPR řešit tak,
%že se změní licence u distribuovaných dat na ODbL. Tato změna
%licencování se očekává v~několika měsících.


\section{IprDownloader}
\label{IprDownloader}
Jak již bylo řečeno tak IPR distribuuje svá data a datové soubory přímo
na~své stránce. Data tam jsou distribuována možností odkazů ke stažení,
nebo je k~dispozici kanál ve formátu ATOM ({\tt http://opendata.iprpraha.cz/feed.xml}).
Pomocí něho se lze také dostat ke všem distribuovaným datům. Jelikož
by bylo zdlouhavé stahovat všechny tyto soubory ručně, proto byl
vytvořen program, který umožňuje stahování a import dat. Skript byl
napsán v~programovacím jazyce Python a využívá knihovny GDAL.

Je rozdělen na tři soubory 
{\tt , IprDownloader.py , IprBase.py} a {\tt IprPg.py}.


\subsection{{\tt IprDownloader.py}}
Na primární skript {\tt IprDowloader.py}, který parsuje vstupní údaje a
zpracovává je k~dalšímu použití ve skriptu, viz \ref{argparse}. Těmito
základními údaji jsou
({\tt ---alike, ---crs, ---format, ---outdir ---d}).
Dalšími vstupními údaji mohou údaje o databázi, kam se mají data
importovat. Těmito vstupy jsou
({\tt ---dbname ---dbschema ---dbhost ---dbport ---dbuser ---dbpassword}).

Skript hlídá vstupní formát kartografických souřadnic, a pokud se
neshodují s~možnými, oznámí chybu o špatném vstupu. Pokud uživatel
vloží kartografické zobrazení ve formě EPGS kódu, program jej převede
na~jeho název.


\subsection{{\tt IprBase.py}}
Skript {\tt IprBase.py} definuje třídu {\tt IprDownloader} se všemi
potřebnými funkcemi. Většina funkcí slouží pro otevírání, čtení a
parsování XML souborů, viz \ref{xmltodict}. Obsahuje funkci
pro~hledání, která prochází všechny záznamy a porovnává, zda vstupní
přibližný název je obsažen v názvu souboru dat. Poté najde odkaz
na~XML souboru dat a otevře jej. Následuje vždy čtení a parsování.
Poté v naparsovaných údajích najde dle nastavení ({\tt crc, format})
správný soubor a stáhne jej do~definovaného adresáře.


\subsection{{\tt IprPg.py}}
Skript {\tt IprPg.py } definuje třídou {\tt IprDownloaderPg }, která
dědí všechny funkce od třídy {\tt IprDownloader } a definuje další
funkce potřebné pro~správný import stažených dat do databáze
PostgreSQL. Přesněji stažený soubor zkontroluje, jestli není 
kompresován v~archivu {\tt *.zip}, jestli ano, vytvoří nový soubor,
který ma stejný název jako archiv. Do nově vytvořeného souboru poté 
extrahuje všechna data z archivu.


\section{Ovládání}
Při pouhém spuštění programu vypíše program všechna nalezená data,
která IPR dává k~dispozici. Pro hledání a označení, která data se mají
stáhnout slouží {\tt ---alike }. Za toto uživatel připojí slovo, nebo
frázi, jež chce v~datech hledat. Pro~sousloví, které obsahuje mezeru,
je vhodné jej umístit do uvozovek. Program rozlišuje velká~x~malá
písmena, háčky a čárky.

Příklad:
{\tt \$ iprdownloader.py ---alike 'Technická mapa'}

Dále jsou k dispozici nastavení.
\begin{itemize}
    \item kartografické zobrazení {\tt ---crs }. K dispozici jsou dvě
    (S-JTSK nebo WGS-84), respektive lze vkládat čtyři různé
    vstupy, protože lze tyto kartografická zobrazení vložit i EPGS
    kódy (5514 a 4326). Přednastavené zobrazení je S-JTSK.

    \item datový formát souborů {\tt ---format}.
    Pro vektorové soubory je možné zvolit pro vektorová data:
        \subitem {\tt json }  JavaScript Object Notation
        \subitem {\tt dxf  }  AutoCAD DXF
        \subitem {\tt gml  }  Geography Markup Language
        \subitem {\tt shp  }  ESRI shapefile
        
    Je předdefinován formát {\tt shp} .
         
    Pokud uživatel chce stahovat rastrová data, je potřeba změnit
    předdefinovaný formát souboru na adekvátní rastrový formát např.
    ({\tt tif}) atd.
    
    \item adresář na místním počítači {\tt ---outdir}. Je předdefinován
    adresář {\tt /data/} ve~složce, kde je uložen samotný program.

\end{itemize}

Pro stažení dat stačí přidat parametr {\tt ---download} .

příklad:

{\tt \$ iprdownloader.py ---alike 'Technická mapa' ---crs 4326 ---format gmp ---outdir IPR}

Pro možnost stahované soubory importovat rovnou do databáze
PostGIS, je minimálně nutné zadat název databáze {\tt ---dbname }.
To v~případě pokud je databáze PostGIS umístěna v lokální síti. Pokud
je databáze umístěna na~jiném serveru, je nutné zadat adresu
{\tt ---dbhost} a port {\tt ---dbport}. Pokud je ještě vyžadován
autorizovaný přístup, použije se přístupové jméno {\tt ---dbuser} a
heslo {\tt ---dbpassword}. Je možné zvolit si schéma v databázi
{\tt ---dbschema}. Pokud je vložen název databáze, není již potřeba
používat parametr {\tt ---download} pro stahování dat.

Příklad:

{\tt \$ iprdownloader.py ---alike 'Technická mapa' ---dbname pgis\_osm\_jakl ---dbschema IPR ---dbhost geo102.fsv.cvut.cz  ---dbport 5432 } 
 

\section{Navržená data pro import}
\label{Navržená data pro import}

Po projití všech dat byla vyhodnocena vhodná data pro začlenění
do~OSM.

\begin{itemize}
    \item   Výstupy PID, obsahuje vstupy/výstupy z metra.
    \item   Odpadní zařízení pro občany, dle IPR obsahuje sběrny
                odpadu. 
    \item   Cyklistická doprava - značky, dle technické dokumentace by
                mělo 
            obsahovat bodové značky stojanů Bikesharing.
    \item   Veřejné toalety
    \item   Parkovací automaty
    \item   Záchytná parkoviště P+R
\end{itemize}

Všechna vybraná data byla pomocí IprDownloader.py stažena a
naimportována do~školní databáze PostGIS. Poté byl program QGIS
připojen na školní databázi. Poté byla v programu QGIS pomocí SQL
SELECTu filtrována data z aktuální verze OSM na školním serveru a
z~každého SELECTu byla vytvořena nová tabulka v~PostGIS databázi
na~školním severu. Data byla filtrována za použití doporučených
atributů na české osm wiki. \cite{OSMfeatures}

Tato vhodná data byla prohledána a porovnána s aktuálním stavem v~OSM.
Byly navrženy vhodné atributy, které by se ze~zdrojových dat daly
naplnit. Vybraná data s navrženými atributy byla uveřejněna na Talk-cz.


\subsection{Výstupy PID}
\label{Výstupy PID}
V datech od IPR bylo obsaženo celkem 353 záznamů. Jednalo se
o~výstupy z~metra (dveře, schodiště a výtahy).

V OSM databázi bylo hledáno v tabulce {\tt osm.czech\_points}
a záznamy s hodnotou
\begin{verbatim}
    railway = subway_entrance
\end{verbatim}
Vytvořená tabulka z dat OSM obsahovala 236 záznamů. Po~porovnání dat
bylo zřejmé, že nejvíce jsou v OSM zmapovány výstupy v centru, kde
jsou posuny v řádech decimetrů až metr. V okrajových částech města
nejsou zmapovány všechny výstupy, nebo chybí výstupy pro celé stanice.
Protože body vstupů do metra jsou většinou součástí linií, není vhodné
body mazat a vytvářet nové. Bylo potřeba najít mezi stávajícími body
v~OSM a IPR adekvátní dvojice. To znamená, že se jedná o body, které
reprezentují stejný objekt v~realitě.

Ani editace stávajících není příliš přínosná, když všechny vstupy
do~metra jsou minimálně 3 metry široké a všechny změny by byly v řádů
cm až dm. Bylo by ale vhodné přidat výstupy do metra tam, kde nejsou
v~databázi OSM. Jedná se o stanice metra Zličín, Luka, Kačerov, Rajská
zahrada a Černý most. Celkem se tedy jedná pouze o 26 bodů, které by
bylo snadnější přidat ručně, protože výstupy jsou dostatečně zmapované,
jen zde není na některém uzlu atribut {\tt railway~=~subway\_entrance}.


\subsection{Odpadní zařízení pro občany}
\label{Odpadní zařízení pro občany}
Data od IPR obsahují seznam 36 sběrných dvorů s velmi detailním
popsání co jakou látku přijímá, otevírací dobu, zřizovatel. 

Z~OSM z~tabulky {\tt osm.czech\_point} byly záznamy vyfiltrovány podle
hodnot
\begin{verbatim}
    amenity = recycling
    recycling_type = centre
\end{verbatim}    
Bylo nalezeno celkem 11 záznamů (bodů), a z~toho bylo 6 bodů (z~IPR)
ve~vzdálenosti 100 metrů od bodů z OSM. Těchto 6 bodů bude vhodné
pouze přesunout. Zbylé body je nutné v databázi OSM nově vytvořit.

Z dostupných dat z tabulky byly navrženy tyto atributy
\begin{verbatim}
    amenity = recycling
    recycling_type = centre
    opening_hours =
    fee =
    operator =
    source =
    source:loc = IPR
\end{verbatim}
Navržené atributy by se naplnily z tabulky dat od IPR. Otevírací doba
by byla vložena do~{\tt opening\_hours~= }. Hodnoty v~tomto atributu
by byly formátovány dle doporučeného formátování.\footnote{http://wiki.openstreetmap.org/wiki/Key:opening\_hours}
U~atributu {\tt fee~= } bylo navrženo {\tt free of charge for the Prague citizens}, tedy zdarma pro~občany Prahy. Hodnota
{\tt operator~= } by byla ze~stejnojmenného sloupce z~dat od~IPR. 
Atribut {\tt source~= } by byl ze~sloupce {\tt POSKYT }.

Dále byly navrženy atributy
\begin{verbatim}
    recycling:paper = yes
    recycling:plastic = yes
    recycling:metal = yes
    recycling:wood = yes
    recycling: ... = yes
\end{verbatim}
Podle toho co zde lze recyklovat, dle hodnoty ve~sloupci
{\tt odpadprije }.


\subsection{Veřejné toalety}
\label{Veřejné toalety}
Datový soubor Veřejné toalety obsahuje jednu tabulku dat. V~této
tabulce je celkem 247 záznamů. U každého záznamu je uvedena lokalita a
provozovatel. Dále je uvedena doba, po kterou je přístupný, zdali je
nutné zaplatit poplatek, jestli je bezbariérový a zdroj těchto dat.

Záznamy odpovídajícíc toaletám byly v~OSM vyhledány podle atributu
\begin{verbatim}
    amenity = toilets
\end{verbatim}

a bylo zjištěno, toho času, 185 záznamů. Překrytí dat IPR a OSM bylo.
Buď k bodu z IPR nebyl v okruhu 5 metrů žádný bod z OSM, nebo
byl, ale měl jiné souřadnice. Také nastaly situace, kdy v OSM
datech byl jen jeden bod, ale z dat od IPR tam bylo více (záznamů) 
bodů. Bylo by tedy vhodné stávající data doplnit daty z IPR.

Na základě rešerše dat z IPR a doporučených atributů byly vybrány tyto
\begin{verbatim}
    acces = yes/permissive
    fee = no/yes
    wheelchair = yes/no
    opening_hours =
    operator =
    description =
    source = IPR
    source:location = IPR
\end{verbatim}
Pokud by hodnota byla {\tt fee~=~yes} připojil by se atribut
\begin{verbatim}
    fee:price =
\end{verbatim}
a jeho hodnota by byla cena v Kč.

Hodnota u {\tt description~= } by byla hodnota ze sloupce
{\tt lokalita}. U~atributu {\tt operator~= } byla navržena hodnota
{\tt Hlavní Město Praha}, ale jen u~bodů, které mají v tabulce
(data od IPR) ve sloupci {\tt Operator} hodnotu {\tt HMP }.
U zbývajících bodů (záznamů) by atribut neměl žádnou hodnotu.


\subsection{Parkovací automaty}
\label{Parkovací automaty}
Datová sada Parkovací automaty obsahovala celkem 452 záznamů (bodů)
parkovacích automatů. Je zde uveden pouze {\tt TYP} a Poskytovatel dat.
Nelze tedy určit zdali automat přijímá mince, bankovky (českou nebo
cizí měnu) a nebo platební karty.
\\*
Pro porovnání byly vyhledány body z OSM s atributy
\begin{verbatim}
    amenity = vendings_machine
    vending = parking_tickets
\end{verbatim}
Podle těchto parametrů bylo nalezeno pouze 8 bodových prvků v databázi
OSM. Pouze dva záznamy z~OSM jsou v~blízkosti k bodům z~IPR. Jeden je
vzdálený do 5 metrů a druhý do 20 metrů. Nelze tedy určit, zda oba (IPR
a OSM bod) odkazují na stejný objekt, nebo na dva různé. Proto se
původní body nebudou přesouvat, a ani rušit. Nové prvky půjde zpětně
rozlišit podle atributu {\tt source:loc~= }.
\\*
U nově přidaných uzlů byly navrženy tyto atributy
\begin{verbatim}
    amenity = vendings_machine
    vending = parking_tickets
    source = IPR
    source:loc = IPR
\end{verbatim}
Do~hodnoty {\tt source~= } by byla použita hodnota ze~sloupce
{\tt POSKYT}). Hodnota u~atributu {\tt source:loc~= } by byla použita
{\tt IPR}.

\subsection{Záchytná parkoviště P+R}
\label{Záchytná parkoviště P+R}
Datová sada Záchytná parkoviště P+R obsahoval 16 záznamů. V~tabulce
bohužel nejsou žádné jiné údaje, kromě polohy uložené v~geometrii a
poskytovatel dat ve~sloupci {\tt poskyt}.

Z dat OSM byly vyhledány body pomocí atributu
\begin{verbatim}
    amenity = park_ride
\end{verbatim}
Tento atribut je stále ve fázi schvalování, tudíž nebyly nalezeny
žádné záznamy. 
Proto bylo vyzkoušeno hledání pomocí obecnějšího atributu
\begin{verbatim}
    amenity = parking
\end{verbatim}
Na území Prahy bylo nalezeno, toho času, 182 záznamů (bodů). Bylo
vyzkoušeno hledání s přidaným atributem {\tt name~= }, který by
obsahoval P+R. Byl nalezen pouze jeden záznam.

Dále bylo vyzkoušeno hledání bodů s atributy
\begin{verbatim}
    amenity = parking
    park_ride = yes
\end{verbatim}
Na území Prahy byl nalezen pouze jeden prvek.

Je tedy navrženo importovat všechny prvky z databáze IPR.
Navržené atributy jsou
\begin{verbatim}
    amenity = parking
    park_ride = yes
    source = HMP-URM
    source:loc = IPR
\end{verbatim}

Výše zmíněné návrhy byly takto zveřejněny v diskuzi na stránkách
talk-cz.

\subsection{Budovy 3D - výšky budov}
\label{Budovy 3D - výšky budov}
V~komentářích k návrhům v~Talk-cz bylo dále doporučeno přidání
výšek budov ze~souboru dat Budovy 3D.
\begin{verbatim}
    building:height
\end{verbatim}
V návrhu bylo doporučeno označení pro zdroj tohoto importu, pro případné pozdější úpravy a opravy.
\begin{verbatim}
    source:building:level =
\end{verbatim}

Při rešerši dat uložených na školní geodatabázi bylo zjištěno,
že data stahovaná a ukládaná z OSM neobsahují potřebné údaje.
Přesněji ukládaná tabulka neobsahovala údaje o výšce budov
(sloupec {\tt height} a sloupec {\tt building\::height}).
Proto bylo nutné základní schéma změnit tak, aby se tyto údaje také 
ukládaly. Upravené schéma bylo uloženo a bylo nastaveno jako výchozí.

Bylo také vyhodnoceno, že bude vhodné, aby se ukládal údaj, jestli
není polygon vytvořen z relace budovy, protože v rámci této relace
(budovy) mají definovanou výšku ({\tt height}, nebo
{\tt building\::height} pouze její části, ze~kterých je složena.
K~tomuto rozeznání byl použit atribut {\tt builing\:part}.

V souboru dat Budovy 3D je dále uloženo 128 datových sad pro menší
oblasti. Tyto datové sady obsahují budovy jen z malé oblasti, protože
by datová sada budov z~celého území Prahy byla moc velká (až v~řádů
GB). Každá datová sada je přibližně obdélníková oblast z území Prahy a
má svůj originální název.

V~tomto příkladě popisu procesu práce bude pracováno s~datovou sadou
{\tt BD\_Prah73}. Po~vybrání této jedné datové sady (oblasti) byla
vytvořena minimální ohraničující oblast okolo ní a přidán přesah o
100~metrů na každou stranu. Tedy byl vytvořen obdélník okolo všech dat
(polygonů) v~jedné datové sadě. Byl zvolen tento postup, aby se z~dat
OSM nevyřadily polygony, které by nebyly celým objemem uvnitř.

Pomocí tohoto obdélníku byly vybrány budovy z OSM uvnitř tohoto
obdélníku a poté byla vytvořena nárazníková zóna (buffer) s hodnotou
3~m pro~každou budovu. Byl zvolen tento postup, protože budovy z IPR a
OSM nemusí mít přesně stejnou polohu a tvar (polygonů). Tento dočasný
datasest byl nazván {\tt bud73\_OSM}.

Pro velikost nárazníkové zóny kolem budov z OSM se provedla analýza.
Modelováno bylo s hodnotami po~0.5~m od~0.5~m až do vzdálenosti 5~m.
Při malé velikost nárazníkové zóny nastávala situace, že v~této zóně
nebyla nalezena žádná budova (polygon budovy) z datasetu IPR. Mohlo by
se zdát, že čím větší nárazníková zóna kolem budov z OSM, tím lépe.
Avšak, čím byla velikost nárazníkové zóny zvolena větší, tím více se
stalo, že v této nárazníkové zóně "uvízlo" více budov (polygonů)
z~datasetu IPR. Bylo tedy nutné zvolit "zlatou střední cestu" tj. 3 m.

Další možností jak se vyvarovat, aby v~nárazníkové zóně kolem budov
z~OSM "uvízlo více polygonů budov (z dataset IPR) je spouštět skript
vícekrát a postupně volit větší a větší nárazníkovou zónu. Skript je
napsán tak, aby vybral a vytvořil nárazníkovou zónu, jen těm budovám
(polygonům), které nemají atribut {\tt building:part} a ani
{\tt building:height}. Ovšem tento postup spouštět skript vícekrát je
samozřejmě také více časově náročné.

Dále byl proveden průzkum dat z~{\tt BD\_Prah73} datové sady
ze~souboru Budovy 3D (z IPR). Bylo zjištěno, že vždy polygony, které
vytvářejí 3D model jedné budovy, mají stejnou hodnotu ve~sloupci
{\tt id\_bud}. V~3D geometrii těchto polygonů budov jsou uloženy
všechny (x,y,z) souřadnice vrcholů, jež určují polygon. Polygony se
stejnou hodnotou {\tt id\_bud} lze dále dělit dle hodnot
ve~sloupci {\tt typ}.
\\*
Těmi to hodnotami jsou
\begin{verbatim}
    dilci plocha stresni kruhove plochy
    komin
    sikma stresni plocha
    svisla obvodova stena
    vikyr_stresni nadstavba
    vytah_vetrani_klimatizace
    vodorovna stresni plocha
    vyznacena vez na strese
    zakladna vikyre, stresni nadstavby
    zakladova deska
 .. a další
\end{verbatim}

Poté byly z~datové sady {\tt BD\_Prah73} vybrány všechny polygony,
které měly ve~sloupci {\tt typ} obsahují frázi {\tt stresni}.
Geometrie těchto vybraných 3D polygonů byly opraveny funkcí
{\tt ST\_MakeValid()} a následně "přetvořeny" na 2D polygony funkcí
{\tt ST\_Force\_2D()}. Vynechala se jen Z-ová souřadnice z~geometrie.
Kvůli některým situacím, kdy k~sobě polygony přesně nedoléhaly, bylo
nutné vytvořit z~nich nárazníkovou zónu (buffer) a jako dostačující
byla zvolena hodnota 0.1 m. Následně byly tyto polygony takzvaně
"rozpuštěny". Rozpuštění znamená, že polygony, které se dotýkají
minimálně jednou společnou stranou, se sloučí do~jednoho polygonu.
Rozpuštěny byly polygony navzájem jen ty, které měly stejnou hodnotu
ve~sloupci {\tt id\_bud}. Tím se docílilo, že každá budova bude
reprezentována jenom jedním 2D polygonem. Tato dočasná datová sada
byla nazvána {\tt bud73\_IPR}.

Následně byly pro každý polygon z~{\tt bud73\_OSM} hledány
všechny polygony z~{\tt bud73\_IPR}, které by byly uvnitř.
Na~tuto operaci byla použita funkce
\begin{verbatim}
    ST_Within( bud73_OSM.geom , bud73_OSM.geom )
\end{verbatim}
Tímto byly nalezeny prvky z dat IPR, které jsou "přibližně na stejném
místě" jako budovy z OSM. Do tabulky o dvou sloupcích byly uloženy
pouze údaje {\tt gid} pro budovy z OSM a {\tt id\_bud} polygonu budovy
(z dat IPR). Tato dočasná tabulka byla nazvána {\tt gid\_id}.

Dále tedy bylo možné k těmto prvkům ({\tt gid} OSM budov) přiřadit
výšku podle hodnoty {\tt id\_bud}, kterou mají všechny polygony
tvořící jednu budovu stejnou. Nejprve musely být vybrány pouze ty
záznamy {\tt gid}, které jsou v tabulce {\tt git\_id} pouze jednou.
To jest polygon budovy z~OSM, ve které byl nalezen pouze jeden polygon
z datasetu IPR. Následně k nim byly připojeny výšky podle hodnoty
{\tt id\_bud}. Pro nalezení výšky budovy \footnote{dle \url{http://wiki.openstreetmap.org/wiki/Cs:Simple_3D_Buildings}}
byly použity všechny polygony tvořící budovu. V~rámci jedné budovy
(podle stejné hodnoty v {\tt id\_bud}) byla výška budovy vypočtena
jako rozdíl nejvyššího bodu a nejnižšího bodu budovy.
Přesněji byla vypočtena takto
\begin{verbatim}
    select	id_bud,
            max(maxZ) - min(minZ) as Height
    from(   select	id_bud,
                    ST_ZMax(geom) as maxZ,
                    ST_ZMin(geom) as minZ
            from    ipr.bd3_prah73
        ) as max_min
    group by id_bud
\end{verbatim}
Ve skriptu je tato dočasná tabulka označena {\tt osm\_height}.

Nakonec se získané hodnoty výšek budov přidají do původního datasetu
(tabulky) {\tt praha\_building\_osm}. 
Dočasné tabulky {\tt bud73\_osm, bud73\_ipr, gid\_id} a
{\tt osm\_height} jsou nakonec vymazány.







\chapter{Závěr}
\label{5-zaver}

V~bakalářské práci bylo popsáno základní dělení prvků v~databázi
OpenStreetMap, licencování a hlavní zdroje dat. Dále byl vysvětlen termín
Opendata (otevřená data).
Při~řešení licenční otázky byl zjištěn problém. Licence dat
OSM a IPR nejsou kompatibilní. Nelze tedy v~současné době začlenit
data IPR do~databáze OSM. V~rámci práce byly popsány hlavní rozdíly mezi těmito licencemi.
%Problém byl řešen tak, že se vyčká na~změnu licence u~dat IPR, a
%v~tomto mezičase se import připraví, aby se poté mohl už jen provést.
Import dat je tedy možný provést až po vyřešení licenčního
problému. Tato práce nicméně import vybraných dat IPR do OSM připravuje.

Pro~snadnější práci byl vytvořen program, který umožňuje vyhledání
v datech IPR, dále jejich stažení a
následný import do~geodatabáze PostGIS. Program byl napsán v~jazyce
Python.


Všechna data zveřejněná IPR byla prozkoumána (v~programu QGIS)
a bylo posouzeno, zdali jsou či nejsou vhodná začlenit je do~databáze
OpenStreetMap. Po~vybrání vhodných dat pro začlenění do~databáze OSM byla
provedena rešerše aktuálního stavu v~databázi OSM.
Na základě toho vzešly čtyři datové sady vhodné pro~import.
Jmenovitě Parkovací automaty, Parkoviště P+R, Odpadní zařízení
pro~občany a Veřejné toalety. Následně z~nich byla vytvořena a upravena
data s~ohledem doporučených atributů OSM.
Na~diskuzním fóru Talk-cz byl tento záměr zveřejněn a v~diskuzi dále bylo
navrženo, zabývat se ještě importem výšek budov.


Po~změně licence u~dat zveřejňovaných IPR se samotný import může provést
v~programu {\tt osmosis}.
Jako zajímavé by se do~budoucna jevilo zpracovat 3D modely budov
z~datasetu Budovy 3D. Což by znamenalo navrhnout program (skript), který by vytvořil
modely budov v~OSM.


% Vysázení seznamu zkratek
\begin{seznamzkratek}{ABCDE}

    \novazkratka{OSM}
		{OSM}
		{OpenStreetMap (\textit{Open Street Map})}

    \novazkratka{IPR}
	    {IPR}
        {Institut plánování a rozvoje hlavního města Prahy}

    \novazkratka{RUIAN}
	    {RUIAN}
        {Registr územní identifikace, adres a nemovitostí}

    \novazkratka{PID}
	    {PID}
        {Pražská integrovaná doprava}

    \novazkratka{KČT}
	    {KČT}
        {Klub českých turistů}

    \novazkratka{CC-BY-SA}
		{CC-BY-SA}
		{Uveďte původ-Zachovejte licenci (\textit{Creative Commons Attribution-ShareAlike})}

    \novazkratka{ODbL}
		{ODbL}
		{(\textit{Open Database License})}

    \novazkratka{GIS}
	      {GIS}
 	      {Geografický informační systém  (\textit{Geographic Information System})}

    \novazkratka{GNSS}
		{GNSS}
		{Globální družicový polohový systém (\textit{Global Navigation Satellite System})}

    \novazkratka{GPS}
		{GPS}
		{Globální polohovací systém (\textit{Global Positioning System})}

    \novazkratka{GLONASS}
		{GLONASS}
		{Glonass (\textit{Globalnaja navigacionnaja sputnikovaja sistěma})}

    \novazkratka{WGS 84}
		{WGS 84}
		{Světový geodetický systém 1984 (\textit{World Geodetic System 1984})}

    \novazkratka{S-JTSK}
		{S-JTSK}
		{Systém jednotné trigonometrické sítě katastrální}

    \novazkratka{GC}
		{GC}
		{Geocaching (\textit{Geocaching})}

    \novazkratka{SQL}
		{SQL}
		{SQL (\textit{Structured Query Language})}

    \novazkratka{XML}
		{XML}
		{Rozšiřitelný značkovací jazyk (\textit{Extensible Markup Language})}

    \novazkratka{GDAL}
		{GDAL}
		{Geoprostorová datová knihovna (\textit{Geospatial Data Abstraction Library})}

    \novazkratka{QGIS}
        {QGIS}
        {Quantum GIS}

    \novazkratka{JOSM}
        {JOSM}
        {Java OpenStreetMap Editor}


\end{seznamzkratek}


% Literatura
\nocite{*}
\def\refname{Literatura}
\bibliographystyle{mystyle}
\bibliography{literatura}


% Začátek příloh
\prilohy

% Vysázení seznamu příloh
\seznampriloh

% Vložení souboru s přílohami
\chapter{Typy vykreslení OSM dat}
\label{vykresleni_dat}

\begin{figure}[H]
    \centering
    \includegraphics[width=11.5cm]{pictures/osm_standard.png} 
    \caption{Standardní mapa (Standard)}
    \label{fig:standard}
\end{figure}

\begin{figure}[H]
    \centering
    \includegraphics[width=11.5cm]{pictures/osm_cyclemap.png} 
    \caption{Cyklistická mapa (Cycle Map)}
    \label{fig:cycle}
\end{figure}

% new list

\begin{figure}[H]
    \centering
    \includegraphics[width=11.5cm]{pictures/osm_transport.png} 
    \caption{Transport Map}
    \label{fig:transport}
\end{figure}

\begin{figure}[H]
    \centering
    \includegraphics[width=11.5cm]{pictures/osm_mapquest.png} 
    \caption{MapQuest Open}
    \label{fig:mapquest}
\end{figure}

% new list

\begin{figure}[H]
    \centering
    \includegraphics[width=11.5cm]{pictures/osm_humanitarian.png} 
    \caption{Humanitární mapa (Humanitarian)}
    \label{fig:humanitarian}
\end{figure}

\begin{figure}[H]
    \centering
    \includegraphics[width=11.5cm]{pictures/F4.png} 
    \caption{Projekt F4 (zdroj \url{http://demo.f4map.com/})}
    \label{fig:F4}
\end{figure}

% new list

\chapter{Diagramy}
\label{digramy}

\begin{figure}[H]
  \centering
  \includegraphics[scale=0.70,angle=90]{pictures/WorkFlow.png}
  \caption{Diagram postupu práce.}
  \label{fig:diagram_workflow}
\end{figure}

\begin{figure}[H]
  \centering
  \includegraphics[scale=0.70,angle=90]{pictures/diagram_building_height.png}
  \caption{Diagram postupu práce pro získání výšky budov.}
  \label{fig:diagram_building}
\end{figure}

% new list

\chapter{Obsah CD}
\label{priloha-obsahCD}
\setlength{\unitlength}{.5mm}
\begin{picture}(250, 220)

  \put(  0, 212){\textbf{.}}

  \put(  1, 200){\line(0, 1){5}}
  \put(  1, 200){\line(1, 0){10} {\textbf{ src}}}  

      \put( 16, 190){\line(0, 1){8}}
      \put( 16, 190){\line(1, 0){10} {\textbf{ iprdownloader}}}
      \put(150, 190){ zdrojové soubory programu}

          \put( 29, 180){\line(0, 1){8}}
          \put( 29, 180){\line(1, 0){10} { IprBase.py}}
          \put( 29, 170){\line(0, 1){10}}
          \put( 29, 170){\line(1, 0){10} { IprPg.py}}
          \put( 29, 160){\line(0, 1){10}}
          \put( 29, 160){\line(1, 0){10} {\textbf{ Iprdownloader.py}}}

      \put( 16, 150){\line(0, 1){40}}
      \put( 16, 150){\line(1, 0){10} {\textbf{ pg}}}
      \put(150, 150){ zdrojové kódy shell skriptu}      

          \put( 29, 140){\line(0, 1){8}}
          \put( 29, 140){\line(1, 0){10} { pgis\_osm\_bp.style}}
          \put(150, 140){ vlastní schema dat z OSM}
          \put( 29, 130){\line(0, 1){10}}
          \put( 29, 130){\line(1, 0){10} { upgrade\_pgis\_osm\_bp.sql}}
          
      \put( 16, 120){\line(0, 1){30}}
      \put( 16, 120){\line(1, 0){10} {\textbf{ sql}}}
      \put(150, 120){ zdrojové kódy sql dump}
            
          \put( 29, 110){\line(0, 1){8}}
          \put( 29, 110){\line(1, 0){10} { buildins.sql}}
          \put( 29, 100){\line(0, 1){10}}
          \put( 29, 100){\line(1, 0){10} { park\_and\_ride.sql}}
          \put( 29,  90){\line(0, 1){10}}
          \put( 29,  90){\line(1, 0){10} { parkomat.sql}}
          \put( 29,  80){\line(0, 1){10}}
          \put( 29,  80){\line(1, 0){10} { recycling\_centre.sql}}
          \put( 29,  70){\line(0, 1){10}}
          \put( 29,  70){\line(1, 0){10} { toilets.sql}}          
          
  \put(  1,  60){\line(0, 1){140}}
  \put(  1,  60){\line(1, 0){10} {\textbf{ text}}}

      \put( 16,  50){\line(0, 1){8}}
      \put( 16,  50){\line(1, 0){10} {\textbf{ latex}}}
      \put(150,  50){ zdrojové soubory textu práce}
      \put( 16,  40){\line(0, 1){10}}
      \put( 16,  40){\line(1, 0){10} { martin-jakl-bp-2016.xls}}
      \put(150,  40){ anotace práce}
      \put( 16,  30){\line(0, 1){10}}
      \put( 16,  30){\line(1, 0){10} { martin-jakl-bp-2016.pdf}}
      \put(150,  30){ práce ve formátu PDF}
      \put( 16,  20){\line(0, 1){10}}
      \put( 16,  20){\line(1, 0){10} { zadani-oficialni.jpeg}}
      \put(150,  20){ naskenované oficiální zadání práce}
\end{picture}


% Konec dokumentu
\end{document}
